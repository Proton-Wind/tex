%\documentclass[b5j,10pt,dvipdfmx]{jsarticle}
\documentclass[b5j,9.5pt]{jsbook}
\usepackage{okumacro}
\usepackage{amsmath,amsthm,amssymb}
\usepackage{enumerate,multicol,}
\usepackage{ascmac,itembbox,emath,hako,scrpage,ulinej,emathP,emathMw,emathEy}
%\usepackage[draft]{graphicx}
\usepackage{graphicx}
\usepackage{picins}
\pagestyle{empty}
\setlength{\textwidth}{162mm}
\setlength{\textheight}{230mm}
\setlength{\oddsidemargin}{-15.4mm}
\setlength{\evensidemargin}{-15.4mm}
\setlength{\topmargin}{-17.4mm}
%\setlength{\columnseprule}{0.4pt}
\renewcommand\labelenumi{\fbox{\bfseries{\sffamily{\theenumi}}}}
\renewcommand{\labelenumii}{(\arabic{enumii})}
\renewcommand{\labelenumiii}{\bfseries{\カタカナ{enumiii}.}}
\def\bunsuushisuu#1#2{\raisebox{0.7zh}{$#1 \over #2$}}
\def\santaku{{\bfseries ア}~{\bfseries ウ}}
\def\yontaku{{\bfseries ア}~{\bfseries エ}}
\def\gotaku{{\bfseries ア}~{\bfseries オ}}
\def\rokutaku{{\bfseries ア}~{\bfseries カ}}
\def\nanataku{{\bfseries ア}~{\bfseries キ}}
\def\sbou#1#2{$_{\sf{#1}}$\kern-0.2pt \ulinej{#2}}%
\def\tanni#1{$〔\mathrm{\sf #1}〕\kern -2pt$}%
\def\sftanni#1{$\kern 2pt{\mathrm{\sf #1}}$}
\begin{document}
\AtBeginDvi{\special{papersize=\the\paperwidth,\the\paperheight}}
%\setcounter{page}{1}
\newpagestyle{custom}{(0mm,0pt)%
{\hfill}
 {%
 \begin{minipage}{16.5cm}
 {\bf {\large 一学期中間考査 物理(科学技術科・情報科)}}\hfill T2A,J2A宮野尾,T2BC,J2B舟橋 R04.5.17(火)1限\\
 \end{minipage}
}%
{\hfill}%
(\textwidth,0pt)}%
{(0mm,0pt)%
{\hfill}{\hfill}{\hfill}%
(0mm,0pt)}
\pagestyle{custom}
\begin{enumerate}
\item 以下の問いに答えよ。必要があれば,$\sqrt{2}=1.41$,$\sqrt{3}=1.73$を用いること。
	\begin{enumerate}
	\item
		\begin{mawarikomi}(0pt,15pt){130pt}{\input{./fig/1中fig01.tex}}
		右図のように平面上の点A,B,Cがある。AB間は4.0\sftanni{m},BC間は3.0\sftanni{m}で,AB$\perp $BCとする。
		図の矢印が示すように,点Aから点Bを経由して点Cまで移動する。
			\begin{enumerate}[m]
			\item このときの移動距離は何\sftanni{m}か。
			\item このときの変位は何\sftanni{m}か。
			\end{enumerate}
		\end{mawarikomi}
	\vspace{10mm}
	\item 雨が鉛直に10\sftanni{m/s}の速さで降る中を,電車がまっすぐな線路上を一定の速さで水平に走っている。このとき,電車内の人が見る雨滴の落下方向は,鉛直方向と45$\Deg$をなしていた。電車の速さを求めよ。
	\item 東向き10\sftanni{m/s}で進む自転車が,10\sftanni{s}後に南向き10\sftanni{m/s}となった。この間における自動車の平均の加速度の向きと,その大きさ$\bar{a}$\tanni{m/s}を求めよ。
	\item 地面より44.1\sftanni{m}の高さから,小球を3.00\sftanni{m/s}で水平に投げ出した。重力加速度の大きさを9.80\sftanni{m/s^2}とする。
		\begin{enumerate}[m]
		\item 地面に到達するのは,投げ出してから何\sftanni{s}後か。
		\item 投げ出した所の真下の点から,小球の落下地点までの水平距離$\ell $\tanni{m}を求めよ。
		\end{enumerate}
	\item
		\begin{mawarikomi}(0pt,0pt){130pt}{\input{./fig/1中fig03.tex}}
		図のように,軽い棒に大きさ2.0\sftanni{N}の力がはたらいている。このとき,点P,点Qのまわりの力のモーメント$M_\mathrm{P}$,$M_\mathrm{Q}$\tanni{N\cdot m}をそれぞれ求めよ。反時計回りを正とする。
		\end{mawarikomi}
	\vspace{10mm}
	\item
		\begin{mawarikomi}(0pt,0pt){130pt}{\input{./fig/1中fig04.tex}}
		図のような厚さが一様な板上の点Pに,図の矢印の方向へ5.0\sftanni{N}を加えるとき,O点回りのモーメントを求めよ。OP間の距離を2.0\sftanni{m},反時計回りを正とする。
		\end{mawarikomi}
	\end{enumerate}
	\vfill
	\vfill
	\item
		\begin{mawarikomi}(0pt,0pt){230pt}{\input{./fig/1中fig02.tex}}
		流れの速さが1.5\sftanni{m/s},川幅50\sftanni{m}のまっすぐな川がある。この川を静水上を2.5\sftanni{m/s}の速さで進むことができる舟で移動する。
			\begin{enumerate}
			\item {\bf 図1}のように,この舟を川の流れ対して常に直角に向けて川を渡るときに要する時間は何\sftanni{s}か。
			\item {\bf 図2}のように,この舟で川を直角に横切りたい。へさきを向けるべき図の角度$\theta $について,
			$\tan{\theta }$を求めよ。
			\item \maru{2}のとき,川を横切るのに要する時間は何\sftanni{s}か。
			\end{enumerate}
		\end{mawarikomi}
	\vfill
\newpage
	\item
		\begin{mawarikomi}(0pt,0pt){160pt}{\input{./fig/1中fig05.tex}}
		北向き5.0\sftanni{m/s}で自転車Aが走行している。以下の問いに答えよ。必要があれば,$\sqrt{2}=1.41$,$\sqrt{3}=1.73$を用いること。
			\begin{enumerate}
			\item 自転車Bから見て,自転車Aは北向き12\sftanni{m/s}で走行しているように見えた。自転車Bの走行している方向と,速さを求めよ。
			\item 自転車Aから見て,自転車Cは東向き5.0\sftanni{m/s}で走行しているように見えた。自転車Cの走行している方向と,速さを求めよ。
			\item 自転車Bから見た,自転車Cの速さを求めよ。
			\end{enumerate}
		\end{mawarikomi}
	\vfill
	\item
		\begin{mawarikomi}(0pt,0pt){160pt}{\input{./fig/1中fig06.tex}}
		地上29.4\sftanni{m}の高さの\ruby{崖}{がけ}の上から,小球を水平から30\Deg 上方に初速度9.80\sftanni{m/s}で投げた。重力加速度の大きさを9.80\sftanni{m/s}とする。
			\begin{enumerate}
			\item 投げてから小球が最高点に達するまでの時間$t_1$\tanni{s}を求めよ。
			\item 崖下の地面から,小球が達する最高点までの高さ$H$\tanni{m}を求めよ。
			\item 投げてから崖下の地面に達するまでの時間$t_2$\tanni{s}を求めよ。
			\item 小球が崖下の地面落下した点と,投げた点の真下との間の水平距離$\ell $\tanni{m}を求めよ。ただし,$\sqrt{3}=1.73$とする。
			\end{enumerate}
		\end{mawarikomi}
	\vfill
	\item
		\begin{mawarikomi}(0pt,0pt){175pt}{\input{./fig/1中fig07.tex}}
		右図のように,水平方向右向きに$x$軸,鉛直方向上向きに$y$軸をとる。原点にある小球1を,初速度の大きさ$v_0$\tanni{m/s},
		$x$軸の正の向きとなす角$\theta $で投げ出すと同時に,点P$(x_0$\tanni{m},$y_0$\tanni{m})にある小球2静かに落下させた(ただし,$x_0>0$,$y_0>0$)。
		重力加速度の大きさを$g$\tanni{m/s^2}とする。
			\begin{enumerate}
			\item 小球1の初速度の$x$成分$v_{x0}$\tanni{m/s},$y$成分$y_{y0}$\tanni{m/s}をそれぞれ求めよ。
			\item 小球1が点Pの真下の点を通過するまでの時間$t$\tanni{s}を求めよ。
			\item (2)のときの,小球1の$y$座標$y_1$\tanni{m}と小球2の$y_2$\tanni{m}をそれぞれ求めよ。
			\item 角$\theta $がある値$\theta _0$のとき,小球1と小球2が衝突したとする。このとき,$\tan{\theta _0}$を求めよ。
			\end{enumerate}
		\end{mawarikomi}
	\vfill
\newpage
	\item
		\begin{mawarikomi}(0pt,0pt){140pt}{\input{./fig/1中fig08.tex}}
		長さ$\ell $\tanni{m}の一様な軽い棒ABがあり,A端は摩擦のある壁に接し,他端Bは,水平に張られた軽い糸1で壁に接続されて,
		{\bf 図1}に示す状態で支えられている。A端から$\bunsuu{2}{3}\ell $\tanni{m}となる点Pに軽い糸2を介して重さ$W$\tanni{N}のおもりを下げた。
		ただし,棒は壁に垂直な壁に垂直な鉛直面内にある。
			\begin{enumerate}
			\item 糸1が棒におよぼす張力を$\overrightarrow{T_\mathrm{B}}$,おもりの重力を$\overrightarrow{W}$とすると,棒が壁から受ける抗力$\overrightarrow{R}$(垂直抗力と静止摩擦力の合力)
			を表すベクトルとして正しいものは,{\bf 図2}の\maru{1}~\maru{4}のうちどれか。
			\item 糸1の張力の大きさを$T_\mathrm{B}$\tanni{N},糸2の張力の大きさを$W$\tanni{N},壁から棒が受ける垂直抗力の大きさを$N$\tanni{N},静止摩擦力の大きさを$F$\tanni{N}として,
			水平方向と鉛直方向のつり合いの式をそれぞれ書け。
			\item A端まわりの力のモーメントのつり合いの式を(2)の$\ell $,$T_\mathrm{B}$,$W$を用いて表した式として正しいものを,次の\yontaku から1つ選び記号で答えよ。
				\begin{enumerate}
				\item $T_\mathrm{B}\cdot \ell \cos{60\Deg }=W\cdot \bunsuu{2}{3}\ell \sin{60\Deg }$
				\item $T_\mathrm{B}\cdot \ell \sin{60\Deg }=W\cdot \bunsuu{2}{3}\ell \cos{60\Deg }$
				\item $T_\mathrm{B}\cdot \ell \cos{60\Deg }=W\cdot \bunsuu{2}{3}\ell \cos{60\Deg }$
				\item $T_\mathrm{B}\cdot \ell \sin{60\Deg }=W\cdot \bunsuu{2}{3}\ell \sin{60\Deg }$
				\end{enumerate}
			\end{enumerate}
		\end{mawarikomi}
\vfill
	\item 右図のように軽い棒に力を加えた。(1)から(3)は図のO端から合力の作用線の位置までの距離を,(4)は偶力のモーメントを求めよ。
		\begin{center}
		\input{./fig/1中fig09.tex}
		\end{center}
\vfill
\end{enumerate}
\newpage
〔計算用紙〕
\vfill
\end{document}
