%\documentclass[b5j,10pt,dvipdfmx]{jsarticle}
\documentclass[b5j,9.5pt]{jsbook}
\usepackage{okumacro}
\usepackage{amsmath,amsthm,amssymb}
\usepackage{enumerate,multicol,}
\usepackage{ascmac,itembbox,emath,hako,scrpage,ulinej,emathP,emathMw,emathEy}
%\usepackage[draft]{graphicx}
\usepackage{graphicx}
\usepackage{picins}
\pagestyle{empty}
\setlength{\textwidth}{162mm}
\setlength{\textheight}{230mm}
\setlength{\oddsidemargin}{-15.4mm}
\setlength{\evensidemargin}{-15.4mm}
\setlength{\topmargin}{-17.4mm}
%\setlength{\columnseprule}{0.4pt}
\renewcommand\labelenumi{\fbox{\bfseries{\sffamily{\theenumi}}}}
\renewcommand{\labelenumii}{(\arabic{enumii})}
\renewcommand{\labelenumiii}{\bfseries{\カタカナ{enumiii}.}}
\def\bunsuushisuu#1#2{\raisebox{0.7zh}{$#1 \over #2$}}
\def\santaku{{\bfseries ア}~{\bfseries ウ}}
\def\yontaku{{\bfseries ア}~{\bfseries エ}}
\def\gotaku{{\bfseries ア}~{\bfseries オ}}
\def\rokutaku{{\bfseries ア}~{\bfseries カ}}
\def\nanataku{{\bfseries ア}~{\bfseries キ}}
\def\sbou#1#2{$_{\sf{#1}}$\kern-0.2pt \ulinej{#2}}%
\def\tanni#1{$〔\mathrm{\sf #1}〕\kern -2pt$}%
\def\sftanni#1{$\kern 2pt{\mathrm{\sf #1}}$}
\begin{document}
\AtBeginDvi{\special{papersize=\the\paperwidth,\the\paperheight}}
%\setcounter{page}{1}
\newpagestyle{custom}{(0mm,0pt)%
{\hfill}
 {%
 \begin{minipage}{16.5cm}
 {\bf {\large 一学期中間考査 物理(科学技術科・情報科)}}\hfill T2A,J2A宮野尾,T2BC,J2B舟橋 R04.5.17(火)1限\\
 \end{minipage}
}%
{\hfill}%
(\textwidth,0pt)}%
{(0mm,0pt)%
{\hfill}{\hfill}{\hfill}%
(0mm,0pt)}
\pagestyle{custom}
\begin{enumerate}
\item 以下の問いに答えよ。必要があれば,$\sqrt{2}=1.41$,$\sqrt{3}=1.73$を用いること。
	\begin{enumerate}
	\item
		\begin{mawarikomi}(0pt,15pt){130pt}{\input{./fig/1中fig01.tex}}
		右図のように平面上の点A,B,Cがある。AB間は4.0\sftanni{m},BC間は3.0\sftanni{m}で,AB$\perp $BCとする。
		図の矢印が示すように,点Aから点Bを経由して点Cまで移動する。
			\begin{enumerate}[m]
			\item このときの移動距離は何\sftanni{m}か。
			\item このときの変位は何\sftanni{m}か。
			\end{enumerate}
		\end{mawarikomi}
	\vspace{10mm}
	\item 雨が鉛直に10\sftanni{m/s}の速さで降る中を,電車がまっすぐな線路上を一定の速さで水平に走っている。このとき,電車内の人が見る雨滴の落下方向は,鉛直方向と45$\Deg$をなしていた。電車の速さを求めよ。
	\item 東向き10\sftanni{m/s}で進む自転車が,10\sftanni{s}後に南向き10\sftanni{m/s}となった。この間における自動車の平均の加速度の向きと,その大きさ$\bar{a}$\tanni{m/s}を求めよ。
	\item 地面より44.1\sftanni{m}の高さから,小球を3.00\sftanni{m/s}で水平に投げ出した。重力加速度の大きさを9.80\sftanni{m/s^2}とする。
		\begin{enumerate}[m]
		\item 地面に到達するのは,投げ出してから何\sftanni{s}後か。
		\item 投げ出した所の真下の点から,小球の落下地点までの水平距離$\ell $\tanni{m}を求めよ。
		\end{enumerate}
	\item
		\begin{mawarikomi}(0pt,0pt){130pt}{%WinTpicVersion4.32a
{\unitlength 0.1in%
\begin{picture}(20.0000,8.8000)(4.0000,-12.5000)%
% BOX 2 0 3 0 Black White  
% 2 400 800 2400 910
% 
\special{pn 8}%
\special{pa 400 800}%
\special{pa 2400 800}%
\special{pa 2400 910}%
\special{pa 400 910}%
\special{pa 400 800}%
\special{pa 2400 800}%
\special{fp}%
% DOT 1 2 3 0 Black White  
% 3 500 860 1600 860 2300 860
% 
\special{pn 4}%
\special{sh 1}%
\special{ar 500 860 10 10 0 6.2831853}%
\special{sh 1}%
\special{ar 1600 860 10 10 0 6.2831853}%
\special{sh 1}%
\special{ar 2300 860 10 10 0 6.2831853}%
% VECTOR 1 0 3 0 Black White  
% 2 1600 860 1600 460
% 
\special{pn 13}%
\special{pa 1600 860}%
\special{pa 1600 460}%
\special{fp}%
\special{sh 1}%
\special{pa 1600 460}%
\special{pa 1580 527}%
\special{pa 1600 513}%
\special{pa 1620 527}%
\special{pa 1600 460}%
\special{fp}%
% LINE 3 1 3 0 Black White  
% 2 500 850 500 1250
% 
\special{pn 4}%
\special{pa 500 850}%
\special{pa 500 1250}%
\special{da 0.070}%
% LINE 3 1 3 0 Black White  
% 2 2300 850 2300 1250
% 
\special{pn 4}%
\special{pa 2300 850}%
\special{pa 2300 1250}%
\special{da 0.070}%
% LINE 3 1 3 0 Black White  
% 2 1600 1250 1600 850
% 
\special{pn 4}%
\special{pa 1600 1250}%
\special{pa 1600 850}%
\special{da 0.070}%
% VECTOR 3 0 3 0 Black White  
% 4 1000 1100 500 1100 1000 1100 1600 1100
% 
\special{pn 4}%
\special{pa 1000 1100}%
\special{pa 500 1100}%
\special{fp}%
\special{sh 1}%
\special{pa 500 1100}%
\special{pa 567 1120}%
\special{pa 553 1100}%
\special{pa 567 1080}%
\special{pa 500 1100}%
\special{fp}%
\special{pa 1000 1100}%
\special{pa 1600 1100}%
\special{fp}%
\special{sh 1}%
\special{pa 1600 1100}%
\special{pa 1533 1080}%
\special{pa 1547 1100}%
\special{pa 1533 1120}%
\special{pa 1600 1100}%
\special{fp}%
% VECTOR 3 0 3 0 Black White  
% 4 2000 1100 1600 1100 2000 1100 2300 1100
% 
\special{pn 4}%
\special{pa 2000 1100}%
\special{pa 1600 1100}%
\special{fp}%
\special{sh 1}%
\special{pa 1600 1100}%
\special{pa 1667 1120}%
\special{pa 1653 1100}%
\special{pa 1667 1080}%
\special{pa 1600 1100}%
\special{fp}%
\special{pa 2000 1100}%
\special{pa 2300 1100}%
\special{fp}%
\special{sh 1}%
\special{pa 2300 1100}%
\special{pa 2233 1080}%
\special{pa 2247 1100}%
\special{pa 2233 1120}%
\special{pa 2300 1100}%
\special{fp}%
% STR 2 0 3 0 Black White  
% 4 1000 1000 1000 1100 5 0 1 0
% 5.5{\sf m}
\put(10.0000,-11.0000){\makebox(0,0){{\colorbox[named]{White}{5.5{\sf m}}}}}%
% STR 2 0 3 0 Black White  
% 4 2000 1000 2000 1100 5 0 1 0
% 3.5{\sf m}
\put(20.0000,-11.0000){\makebox(0,0){{\colorbox[named]{White}{3.5{\sf m}}}}}%
% STR 2 0 3 0 Black White  
% 4 1720 400 1720 500 2 0 0 0
% 2.0{\sf N}
\put(17.2000,-5.0000){\makebox(0,0)[lb]{2.0{\sf N}}}%
% STR 2 0 3 0 Black White  
% 4 500 600 500 700 5 0 0 0
% P
\put(5.0000,-7.0000){\makebox(0,0){P}}%
% STR 2 0 3 0 Black White  
% 4 2300 600 2300 700 5 0 0 0
% Q
\put(23.0000,-7.0000){\makebox(0,0){Q}}%
\end{picture}}%
}
		図のように,軽い棒に大きさ2.0\sftanni{N}の力がはたらいている。このとき,点P,点Qのまわりの力のモーメント$M_\mathrm{P}$,$M_\mathrm{Q}$\tanni{N\cdot m}をそれぞれ求めよ。反時計回りを正とする。
		\end{mawarikomi}
	\vspace{10mm}
	\item
		\begin{mawarikomi}(0pt,0pt){130pt}{%WinTpicVersion4.32a
{\unitlength 0.1in%
\begin{picture}(16.0000,13.7000)(3.0000,-17.0000)%
% ELLIPSE 2 0 3 0 Black White  
% 4 1100 1100 1900 500 1900 500 1900 500
% 
\special{pn 8}%
\special{ar 1100 1100 800 600 0.0000000 6.2831853}%
% STR 2 0 3 0 Black White  
% 4 500 1200 500 1300 5 0 0 0
% $\odot$
\put(5.0000,-13.0000){\makebox(0,0){$\odot$}}%
% LINE 2 0 3 0 Black White  
% 2 500 1300 1500 720
% 
\special{pn 8}%
\special{pa 500 1300}%
\special{pa 1500 720}%
\special{fp}%
% VECTOR 1 0 3 0 Black White  
% 2 1500 730 1500 330
% 
\special{pn 13}%
\special{pa 1500 730}%
\special{pa 1500 330}%
\special{fp}%
\special{sh 1}%
\special{pa 1500 330}%
\special{pa 1480 397}%
\special{pa 1500 383}%
\special{pa 1520 397}%
\special{pa 1500 330}%
\special{fp}%
% LINE 3 1 3 0 Black White  
% 2 1500 700 1500 1700
% 
\special{pn 4}%
\special{pa 1500 700}%
\special{pa 1500 1700}%
\special{da 0.070}%
% DOT 1 0 3 0 Black White  
% 1 1500 730
% 
\special{pn 4}%
\special{sh 1}%
\special{ar 1500 730 10 10 0 6.2831853}%
% CIRCLE 2 0 3 0 Black White  
% 4 1500 730 1700 730 1100 950 1500 950
% 
\special{pn 8}%
\special{ar 1500 730 200 200 1.5707963 2.6387494}%
% STR 2 0 3 0 Black White  
% 4 1300 930 1300 1030 5 0 0 0
% 60$^\circ $
\put(13.0000,-10.3000){\makebox(0,0){60$^\circ $}}%
% STR 2 0 3 0 Black White  
% 4 1590 400 1590 500 2 0 0 0
% 5.0{\sf N}
\put(15.9000,-5.0000){\makebox(0,0)[lb]{5.0{\sf N}}}%
% CIRCLE 2 2 3 0 Black White  
% 4 1390 1700 1410 740 1490 740 90 1090
% 
\special{pn 8}%
\special{pn 8}%
\special{pa 521 1292}%
\special{pa 524 1285}%
\special{fp}%
\special{pa 541 1252}%
\special{pa 545 1244}%
\special{fp}%
\special{pa 563 1212}%
\special{pa 567 1205}%
\special{fp}%
\special{pa 587 1173}%
\special{pa 592 1167}%
\special{fp}%
\special{pa 613 1136}%
\special{pa 618 1130}%
\special{fp}%
\special{pa 640 1100}%
\special{pa 645 1094}%
\special{fp}%
\special{pa 669 1066}%
\special{pa 675 1060}%
\special{fp}%
\special{pa 700 1033}%
\special{pa 706 1026}%
\special{fp}%
\special{pa 733 1000}%
\special{pa 739 995}%
\special{fp}%
\special{pa 766 970}%
\special{pa 773 965}%
\special{fp}%
\special{pa 801 941}%
\special{pa 808 937}%
\special{fp}%
\special{pa 838 915}%
\special{pa 844 910}%
\special{fp}%
\special{pa 875 889}%
\special{pa 882 885}%
\special{fp}%
\special{pa 914 866}%
\special{pa 921 862}%
\special{fp}%
\special{pa 954 845}%
\special{pa 961 841}%
\special{fp}%
\special{pa 995 825}%
\special{pa 1002 822}%
\special{fp}%
\special{pa 1037 807}%
\special{pa 1044 804}%
\special{fp}%
\special{pa 1079 792}%
\special{pa 1087 789}%
\special{fp}%
\special{pa 1122 778}%
\special{pa 1130 776}%
\special{fp}%
\special{pa 1166 767}%
\special{pa 1174 764}%
\special{fp}%
\special{pa 1210 757}%
\special{pa 1218 755}%
\special{fp}%
\special{pa 1255 749}%
\special{pa 1263 748}%
\special{fp}%
\special{pa 1300 744}%
\special{pa 1308 743}%
\special{fp}%
\special{pa 1345 741}%
\special{pa 1353 741}%
\special{fp}%
\special{pa 1391 740}%
\special{pa 1399 740}%
\special{fp}%
\special{pa 1436 741}%
\special{pa 1444 742}%
\special{fp}%
\special{pa 1481 745}%
\special{pa 1489 745}%
\special{fp}%
% STR 2 0 3 0 Black White  
% 4 790 800 790 900 5 0 1 0
% 2.0{\sf m}
\put(7.9000,-9.0000){\makebox(0,0){{\colorbox[named]{White}{2.0{\sf m}}}}}%
% STR 2 0 3 0 Black White  
% 4 580 1280 580 1380 5 0 0 0
% O
\put(5.8000,-13.8000){\makebox(0,0){O}}%
% STR 2 0 3 0 Black White  
% 4 1580 680 1580 780 5 0 0 0
% P
\put(15.8000,-7.8000){\makebox(0,0){P}}%
\end{picture}}%
}
		図のような厚さが一様な板上の点Pに,図の矢印の方向へ5.0\sftanni{N}を加えるとき,O点回りのモーメントを求めよ。OP間の距離を2.0\sftanni{m},反時計回りを正とする。
		\end{mawarikomi}
	\end{enumerate}
	\vfill
	\vfill
	\item
		\begin{mawarikomi}(0pt,0pt){230pt}{\input{./fig/1中fig02.tex}}
		流れの速さが1.5\sftanni{m/s},川幅50\sftanni{m}のまっすぐな川がある。この川を静水上を2.5\sftanni{m/s}の速さで進むことができる舟で移動する。
			\begin{enumerate}
			\item {\bf 図1}のように,この舟を川の流れ対して常に直角に向けて川を渡るときに要する時間は何\sftanni{s}か。
			\item {\bf 図2}のように,この舟で川を直角に横切りたい。へさきを向けるべき図の角度$\theta $について,
			$\tan{\theta }$を求めよ。
			\item \maru{2}のとき,川を横切るのに要する時間は何\sftanni{s}か。
			\end{enumerate}
		\end{mawarikomi}
	\vfill
\newpage
	\item
		\begin{mawarikomi}(0pt,0pt){160pt}{%WinTpicVersion4.32a
{\unitlength 0.1in%
\begin{picture}(22.0700,22.0000)(2.6000,-26.3500)%
% VECTOR 2 0 3 0 Black White  
% 4 1400 1600 2400 1600 1400 1600 400 1600
% 
\special{pn 8}%
\special{pa 1400 1600}%
\special{pa 2400 1600}%
\special{fp}%
\special{sh 1}%
\special{pa 2400 1600}%
\special{pa 2333 1580}%
\special{pa 2347 1600}%
\special{pa 2333 1620}%
\special{pa 2400 1600}%
\special{fp}%
\special{pa 1400 1600}%
\special{pa 400 1600}%
\special{fp}%
\special{sh 1}%
\special{pa 400 1600}%
\special{pa 467 1620}%
\special{pa 453 1600}%
\special{pa 467 1580}%
\special{pa 400 1600}%
\special{fp}%
% VECTOR 2 0 3 0 Black White  
% 2 1400 1600 1400 600
% 
\special{pn 8}%
\special{pa 1400 1600}%
\special{pa 1400 600}%
\special{fp}%
\special{sh 1}%
\special{pa 1400 600}%
\special{pa 1380 667}%
\special{pa 1400 653}%
\special{pa 1420 667}%
\special{pa 1400 600}%
\special{fp}%
% VECTOR 1 0 3 0 Black White  
% 2 1400 1600 1400 1000
% 
\special{pn 13}%
\special{pa 1400 1600}%
\special{pa 1400 1000}%
\special{fp}%
\special{sh 1}%
\special{pa 1400 1000}%
\special{pa 1380 1067}%
\special{pa 1400 1053}%
\special{pa 1420 1067}%
\special{pa 1400 1000}%
\special{fp}%
% STR 2 0 3 0 Black White  
% 4 1400 400 1400 500 5 0 0 0
% �k
\put(14.0000,-5.0000){\makebox(0,0){�k}}%
% STR 2 0 3 0 Black White  
% 4 2500 1500 2500 1600 5 0 0 0
% ��
\put(25.0000,-16.0000){\makebox(0,0){��}}%
% VECTOR 2 0 3 0 Black White  
% 2 1400 1600 1400 2600
% 
\special{pn 8}%
\special{pa 1400 1600}%
\special{pa 1400 2600}%
\special{fp}%
\special{sh 1}%
\special{pa 1400 2600}%
\special{pa 1420 2533}%
\special{pa 1400 2547}%
\special{pa 1380 2533}%
\special{pa 1400 2600}%
\special{fp}%
% STR 2 0 3 0 Black White  
% 4 1400 2600 1400 2700 5 0 0 0
% ��
\put(14.0000,-27.0000){\makebox(0,0){��}}%
% STR 2 0 3 0 Black White  
% 4 320 1500 320 1600 5 0 0 0
% ��
\put(3.2000,-16.0000){\makebox(0,0){��}}%
% STR 2 0 3 0 Black White  
% 4 1370 1530 1370 1630 4 0 0 0
% O
\put(13.7000,-16.3000){\makebox(0,0)[rt]{O}}%
% STR 2 0 3 0 Black White  
% 4 1440 900 1440 1000 2 0 0 0
% ���]��A
\put(14.4000,-10.0000){\makebox(0,0)[lb]{���]��A}}%
% STR 2 0 3 0 Black White  
% 4 1320 1100 1320 1200 3 0 0 0
% 5.0{\sf m/s}
\put(13.2000,-12.0000){\makebox(0,0)[rb]{5.0{\sf m/s}}}%
\end{picture}}%
}
		北向き5.0\sftanni{m/s}で自転車Aが走行している。以下の問いに答えよ。必要があれば,$\sqrt{2}=1.41$,$\sqrt{3}=1.73$を用いること。
			\begin{enumerate}
			\item 自転車Bから見て,自転車Aは北向き12\sftanni{m/s}で走行しているように見えた。自転車Bの走行している方向と,速さを求めよ。
			\item 自転車Aから見て,自転車Cは東向き5.0\sftanni{m/s}で走行しているように見えた。自転車Cの走行している方向と,速さを求めよ。
			\item 自転車Bから見た,自転車Cの速さを求めよ。
			\end{enumerate}
		\end{mawarikomi}
	\vfill
	\item
		\begin{mawarikomi}(0pt,0pt){160pt}{\input{./fig/1中fig06.tex}}
		地上29.4\sftanni{m}の高さの\ruby{崖}{がけ}の上から,小球を水平から30\Deg 上方に初速度9.80\sftanni{m/s}で投げた。重力加速度の大きさを9.80\sftanni{m/s}とする。
			\begin{enumerate}
			\item 投げてから小球が最高点に達するまでの時間$t_1$\tanni{s}を求めよ。
			\item 崖下の地面から,小球が達する最高点までの高さ$H$\tanni{m}を求めよ。
			\item 投げてから崖下の地面に達するまでの時間$t_2$\tanni{s}を求めよ。
			\item 小球が崖下の地面落下した点と,投げた点の真下との間の水平距離$\ell $\tanni{m}を求めよ。ただし,$\sqrt{3}=1.73$とする。
			\end{enumerate}
		\end{mawarikomi}
	\vfill
	\item
		\begin{mawarikomi}(0pt,0pt){175pt}{%WinTpicVersion4.32a
{\unitlength 0.1in%
\begin{picture}(24.8300,17.5800)(3.1000,-24.7600)%
% VECTOR 2 0 3 0 Black White  
% 2 536 2438 2793 2438
% 
\special{pn 8}%
\special{pa 536 2438}%
\special{pa 2793 2438}%
\special{fp}%
\special{sh 1}%
\special{pa 2793 2438}%
\special{pa 2726 2418}%
\special{pa 2740 2438}%
\special{pa 2726 2458}%
\special{pa 2793 2438}%
\special{fp}%
% VECTOR 2 0 3 0 Black White  
% 2 536 2438 536 783
% 
\special{pn 8}%
\special{pa 536 2438}%
\special{pa 536 783}%
\special{fp}%
\special{sh 1}%
\special{pa 536 783}%
\special{pa 516 850}%
\special{pa 536 836}%
\special{pa 556 850}%
\special{pa 536 783}%
\special{fp}%
% STR 2 0 3 0 Black White  
% 4 513 2385 513 2460 4 0 0 0
% O
\put(5.1300,-24.6000){\makebox(0,0)[rt]{O}}%
% VECTOR 1 0 3 0 Black White  
% 2 536 2438 987 2137
% 
\special{pn 13}%
\special{pa 536 2438}%
\special{pa 987 2137}%
\special{fp}%
\special{sh 1}%
\special{pa 987 2137}%
\special{pa 920 2157}%
\special{pa 943 2167}%
\special{pa 943 2191}%
\special{pa 987 2137}%
\special{fp}%
% FUNC 2 2 3 0 Black White  
% 9 536 1837 2492 2438 1890 2438 3244 2438 1890 1987 536 1837 2491 2438 0 2 0 0
% -x^2+1
\special{pn 8}%
\special{pn 8}%
\special{pa 536 2438}%
\special{pa 543 2433}%
\special{fp}%
\special{pa 574 2413}%
\special{pa 581 2408}%
\special{fp}%
\special{pa 613 2388}%
\special{pa 620 2384}%
\special{fp}%
\special{pa 651 2364}%
\special{pa 658 2360}%
\special{fp}%
\special{pa 690 2341}%
\special{pa 697 2337}%
\special{fp}%
\special{pa 730 2318}%
\special{pa 737 2314}%
\special{fp}%
\special{pa 770 2296}%
\special{pa 777 2292}%
\special{fp}%
\special{pa 810 2274}%
\special{pa 817 2270}%
\special{fp}%
\special{pa 850 2253}%
\special{pa 857 2250}%
\special{fp}%
\special{pa 891 2233}%
\special{pa 898 2229}%
\special{fp}%
\special{pa 932 2213}%
\special{pa 939 2209}%
\special{fp}%
\special{pa 973 2194}%
\special{pa 981 2191}%
\special{fp}%
\special{pa 1015 2175}%
\special{pa 1022 2172}%
\special{fp}%
\special{pa 1057 2158}%
\special{pa 1064 2154}%
\special{fp}%
\special{pa 1099 2141}%
\special{pa 1107 2138}%
\special{fp}%
\special{pa 1142 2125}%
\special{pa 1149 2122}%
\special{fp}%
\special{pa 1185 2109}%
\special{pa 1192 2107}%
\special{fp}%
\special{pa 1228 2095}%
\special{pa 1236 2093}%
\special{fp}%
\special{pa 1271 2081}%
\special{pa 1279 2079}%
\special{fp}%
\special{pa 1315 2068}%
\special{pa 1323 2066}%
\special{fp}%
\special{pa 1359 2056}%
\special{pa 1367 2055}%
\special{fp}%
\special{pa 1403 2045}%
\special{pa 1411 2044}%
\special{fp}%
\special{pa 1447 2036}%
\special{pa 1455 2034}%
\special{fp}%
\special{pa 1492 2026}%
\special{pa 1500 2024}%
\special{fp}%
\special{pa 1537 2018}%
\special{pa 1545 2016}%
\special{fp}%
\special{pa 1582 2011}%
\special{pa 1590 2009}%
\special{fp}%
\special{pa 1627 2004}%
\special{pa 1635 2003}%
\special{fp}%
\special{pa 1672 1999}%
\special{pa 1680 1998}%
\special{fp}%
\special{pa 1717 1995}%
\special{pa 1725 1994}%
\special{fp}%
\special{pa 1762 1991}%
\special{pa 1770 1991}%
\special{fp}%
\special{pa 1808 1989}%
\special{pa 1816 1988}%
\special{fp}%
\special{pa 1853 1987}%
\special{pa 1861 1987}%
\special{fp}%
\special{pa 1899 1987}%
\special{pa 1907 1987}%
\special{fp}%
\special{pa 1945 1988}%
\special{pa 1953 1988}%
\special{fp}%
\special{pa 1990 1989}%
\special{pa 1998 1990}%
\special{fp}%
\special{pa 2035 1992}%
\special{pa 2043 1993}%
\special{fp}%
\special{pa 2081 1996}%
\special{pa 2089 1997}%
\special{fp}%
\special{pa 2126 2001}%
\special{pa 2134 2002}%
\special{fp}%
\special{pa 2171 2006}%
\special{pa 2179 2008}%
\special{fp}%
\special{pa 2216 2013}%
\special{pa 2224 2015}%
\special{fp}%
\special{pa 2261 2021}%
\special{pa 2269 2023}%
\special{fp}%
\special{pa 2306 2029}%
\special{pa 2314 2031}%
\special{fp}%
\special{pa 2350 2039}%
\special{pa 2358 2041}%
\special{fp}%
\special{pa 2394 2050}%
\special{pa 2402 2051}%
\special{fp}%
\special{pa 2439 2061}%
\special{pa 2446 2063}%
\special{fp}%
\special{pa 2482 2073}%
\special{pa 2490 2076}%
\special{fp}%
% LINE 2 1 3 0 Black White  
% 2 2492 2438 2492 1084
% 
\special{pn 8}%
\special{pa 2492 2438}%
\special{pa 2492 1084}%
\special{da 0.070}%
% CIRCLE 2 0 0 0 Black Black  
% 4 2492 1084 2530 1084 2530 1084 2530 1084
% 
\special{sh 1.000}%
\special{ia 2492 1084 38 38 0.0000000 6.2831853}%
\special{pn 8}%
\special{ar 2492 1084 38 38 0.0000000 6.2831853}%
% CIRCLE 2 0 2 0 Black White  
% 4 536 2438 574 2438 574 2438 574 2438
% 
\special{sh 0}%
\special{ia 536 2438 38 38 0.0000000 6.2831853}%
\special{pn 8}%
\special{ar 536 2438 38 38 0.0000000 6.2831853}%
% STR 2 0 3 0 Black White  
% 4 460 707 460 783 5 0 0 0
% $y$
\put(4.6000,-7.8300){\makebox(0,0){$y$}}%
% STR 2 0 3 0 Black White  
% 4 2717 2445 2717 2521 5 0 0 0
% $x$
\put(27.1700,-25.2100){\makebox(0,0){$x$}}%
% STR 2 0 3 0 Black White  
% 4 2492 2445 2492 2521 5 0 0 0
% $x_0$
\put(24.9200,-25.2100){\makebox(0,0){$x_0$}}%
% LINE 2 1 3 0 Black White  
% 2 2492 1084 536 1084
% 
\special{pn 8}%
\special{pa 2492 1084}%
\special{pa 536 1084}%
\special{da 0.070}%
% STR 2 0 3 0 Black White  
% 4 460 1009 460 1084 5 0 0 0
% $y_0$
\put(4.6000,-10.8400){\makebox(0,0){$y_0$}}%
% CIRCLE 2 0 3 0 Black White  
% 4 536 2438 761 2438 2492 2438 2492 1084
% 
\special{pn 8}%
\special{ar 536 2438 225 225 5.6776938 6.2831853}%
% STR 2 0 3 0 Black White  
% 4 792 2333 792 2408 2 0 0 0
% $\theta $
\put(7.9200,-24.0800){\makebox(0,0)[lb]{$\theta $}}%
% STR 2 0 3 0 Black White  
% 4 2379 919 2379 993 2 0 0 0
% ����2
\put(23.7900,-9.9300){\makebox(0,0)[lb]{����2}}%
% STR 2 0 3 0 Black White  
% 4 544 2112 544 2187 2 0 0 0
% ����1
\put(5.4400,-21.8700){\makebox(0,0)[lb]{����1}}%
% STR 2 0 3 0 Black White  
% 4 987 2039 987 2114 2 0 0 0
% $v_0$
\put(9.8700,-21.1400){\makebox(0,0)[lb]{$v_0$}}%
\end{picture}}%
}
		右図のように,水平方向右向きに$x$軸,鉛直方向上向きに$y$軸をとる。原点にある小球1を,初速度の大きさ$v_0$\tanni{m/s},
		$x$軸の正の向きとなす角$\theta $で投げ出すと同時に,点P$(x_0$\tanni{m},$y_0$\tanni{m})にある小球2静かに落下させた(ただし,$x_0>0$,$y_0>0$)。
		重力加速度の大きさを$g$\tanni{m/s^2}とする。
			\begin{enumerate}
			\item 小球1の初速度の$x$成分$v_{x0}$\tanni{m/s},$y$成分$y_{y0}$\tanni{m/s}をそれぞれ求めよ。
			\item 小球1が点Pの真下の点を通過するまでの時間$t$\tanni{s}を求めよ。
			\item (2)のときの,小球1の$y$座標$y_1$\tanni{m}と小球2の$y_2$\tanni{m}をそれぞれ求めよ。
			\item 角$\theta $がある値$\theta _0$のとき,小球1と小球2が衝突したとする。このとき,$\tan{\theta _0}$を求めよ。
			\end{enumerate}
		\end{mawarikomi}
	\vfill
\newpage
	\item
		\begin{mawarikomi}(0pt,0pt){140pt}{%WinTpicVersion4.32a
{\unitlength 0.1in%
\begin{picture}(17.4100,39.9500)(2.4300,-41.9500)%
% LINE 2 0 3 0 Black White  
% 2 395 200 395 2000
% 
\special{pn 8}%
\special{pa 395 200}%
\special{pa 395 2000}%
\special{fp}%
% LINE 2 0 3 0 Black White  
% 2 395 650 1945 650
% 
\special{pn 8}%
\special{pa 395 650}%
\special{pa 1945 650}%
\special{fp}%
% POLYGON 2 0 3 0 Black White  
% 5 400 1547 425 1590 1984 690 1959 646 400 1547
% 
\special{pn 8}%
\special{pa 400 1547}%
\special{pa 425 1590}%
\special{pa 1984 690}%
\special{pa 1959 646}%
\special{pa 400 1547}%
\special{pa 425 1590}%
\special{fp}%
% DOT 0 0 3 0 Black White  
% 1 1440 984
% 
\special{pn 4}%
\special{sh 1}%
\special{ar 1440 984 16 16 0 6.2831853}%
% LINE 3 0 3 0 Black White  
% 62 295 260 355 200 295 320 395 220 295 380 395 280 295 440 395 340 295 500 395 400 295 560 395 460 295 620 395 520 295 680 395 580 295 740 395 640 295 800 395 700 295 860 395 760 295 920 395 820 295 980 395 880 295 1040 395 940 295 1100 395 1000 295 1160 395 1060 295 1220 395 1120 295 1280 395 1180 295 1340 395 1240 295 1400 395 1300 295 1460 395 1360 295 1520 395 1420 295 1580 395 1480 295 1640 395 1540 295 1700 395 1600 295 1760 395 1660 295 1820 395 1720 295 1880 395 1780 295 1940 395 1840 295 2000 395 1900 355 2000 395 1960
% 
\special{pn 4}%
\special{pa 295 260}%
\special{pa 355 200}%
\special{fp}%
\special{pa 295 320}%
\special{pa 395 220}%
\special{fp}%
\special{pa 295 380}%
\special{pa 395 280}%
\special{fp}%
\special{pa 295 440}%
\special{pa 395 340}%
\special{fp}%
\special{pa 295 500}%
\special{pa 395 400}%
\special{fp}%
\special{pa 295 560}%
\special{pa 395 460}%
\special{fp}%
\special{pa 295 620}%
\special{pa 395 520}%
\special{fp}%
\special{pa 295 680}%
\special{pa 395 580}%
\special{fp}%
\special{pa 295 740}%
\special{pa 395 640}%
\special{fp}%
\special{pa 295 800}%
\special{pa 395 700}%
\special{fp}%
\special{pa 295 860}%
\special{pa 395 760}%
\special{fp}%
\special{pa 295 920}%
\special{pa 395 820}%
\special{fp}%
\special{pa 295 980}%
\special{pa 395 880}%
\special{fp}%
\special{pa 295 1040}%
\special{pa 395 940}%
\special{fp}%
\special{pa 295 1100}%
\special{pa 395 1000}%
\special{fp}%
\special{pa 295 1160}%
\special{pa 395 1060}%
\special{fp}%
\special{pa 295 1220}%
\special{pa 395 1120}%
\special{fp}%
\special{pa 295 1280}%
\special{pa 395 1180}%
\special{fp}%
\special{pa 295 1340}%
\special{pa 395 1240}%
\special{fp}%
\special{pa 295 1400}%
\special{pa 395 1300}%
\special{fp}%
\special{pa 295 1460}%
\special{pa 395 1360}%
\special{fp}%
\special{pa 295 1520}%
\special{pa 395 1420}%
\special{fp}%
\special{pa 295 1580}%
\special{pa 395 1480}%
\special{fp}%
\special{pa 295 1640}%
\special{pa 395 1540}%
\special{fp}%
\special{pa 295 1700}%
\special{pa 395 1600}%
\special{fp}%
\special{pa 295 1760}%
\special{pa 395 1660}%
\special{fp}%
\special{pa 295 1820}%
\special{pa 395 1720}%
\special{fp}%
\special{pa 295 1880}%
\special{pa 395 1780}%
\special{fp}%
\special{pa 295 1940}%
\special{pa 395 1840}%
\special{fp}%
\special{pa 295 2000}%
\special{pa 395 1900}%
\special{fp}%
\special{pa 355 2000}%
\special{pa 395 1960}%
\special{fp}%
% CIRCLE 2 0 3 0 Black White  
% 4 395 1540 595 1540 1815 720 395 720
% 
\special{pn 8}%
\special{ar 395 1540 200 200 4.7123890 5.7595006}%
% STR 2 0 3 0 Black White  
% 4 475 1210 475 1310 2 0 0 0
% 60\Deg
\put(4.7500,-13.1000){\makebox(0,0)[lb]{60\Deg}}%
% LINE 3 0 3 0 Black White  
% 4 395 750 495 750 495 750 495 650
% 
\special{pn 4}%
\special{pa 395 750}%
\special{pa 495 750}%
\special{fp}%
\special{pa 495 750}%
\special{pa 495 650}%
\special{fp}%
% LINE 2 0 3 0 Black White  
% 2 1435 970 1435 1520
% 
\special{pn 8}%
\special{pa 1435 970}%
\special{pa 1435 1520}%
\special{fp}%
% CIRCLE 2 0 2 0 Black White  
% 4 1435 1520 1495 1520 1495 1520 1495 1520
% 
\special{sh 0}%
\special{ia 1435 1520 60 60 0.0000000 6.2831853}%
\special{pn 8}%
\special{ar 1435 1520 60 60 0.0000000 6.2831853}%
% DOT 0 0 3 0 Black White  
% 1 395 1540
% 
\special{pn 4}%
\special{sh 1}%
\special{ar 395 1540 16 16 0 6.2831853}%
% STR 2 0 3 0 Black White  
% 4 1965 510 1965 610 2 0 0 0
% B
\put(19.6500,-6.1000){\makebox(0,0)[lb]{B}}%
% STR 2 0 3 0 Black White  
% 4 295 1490 295 1590 5 0 1 0
% A
\put(2.9500,-15.9000){\makebox(0,0){{\colorbox[named]{White}{A}}}}%
% STR 2 0 3 0 Black White  
% 4 1435 770 1435 870 5 0 0 0
% P
\put(14.3500,-8.7000){\makebox(0,0){P}}%
% STR 2 0 3 0 Black White  
% 4 1515 1480 1515 1580 2 0 0 0
% �d��$W$�k{\sf N}�l
\put(15.1500,-15.8000){\makebox(0,0)[lb]{�d��$W$�k{\sf N}�l}}%
% CIRCLE 3 0 3 0 Black White  
% 4 725 900 1325 1290 325 1690 1725 1030
% 
\special{pn 4}%
\special{ar 725 900 716 716 0.1292750 2.0394944}%
% STR 2 0 3 0 Black White  
% 4 695 1690 695 1790 2 0 1 0
% $\bunsuu{2}{3}\ell $�k{\sf m}�l
\put(6.9500,-17.9000){\makebox(0,0)[lb]{{\colorbox[named]{White}{$\bunsuu{2}{3}\ell $�k{\sf m}�l}}}}%
% STR 2 0 3 0 Black White  
% 4 1025 510 1025 610 2 0 0 0
% ��1
\put(10.2500,-6.1000){\makebox(0,0)[lb]{��1}}%
% STR 2 0 3 0 Black White  
% 4 1465 1190 1465 1290 2 0 0 0
% ��2
\put(14.6500,-12.9000){\makebox(0,0)[lb]{��2}}%
% LINE 2 0 3 0 Black White  
% 2 391 2200 391 3791
% 
\special{pn 8}%
\special{pa 391 2200}%
\special{pa 391 3791}%
\special{fp}%
% LINE 2 0 3 0 Black White  
% 2 391 2597 1761 2597
% 
\special{pn 8}%
\special{pa 391 2597}%
\special{pa 1761 2597}%
\special{fp}%
% POLYGON 2 0 3 0 Black White  
% 5 396 3391 418 3429 1796 2633 1774 2594 396 3391
% 
\special{pn 8}%
\special{pa 396 3391}%
\special{pa 418 3429}%
\special{pa 1796 2633}%
\special{pa 1774 2594}%
\special{pa 396 3391}%
\special{pa 418 3429}%
\special{fp}%
% DOT 0 0 3 0 Black White  
% 1 1315 2893
% 
\special{pn 4}%
\special{sh 1}%
\special{ar 1315 2893 16 16 0 6.2831853}%
% LINE 3 0 3 0 Black White  
% 62 302 2253 355 2200 302 2305 391 2217 302 2359 391 2271 302 2412 391 2324 302 2465 391 2377 302 2518 391 2429 302 2572 391 2483 302 2624 391 2536 302 2677 391 2589 302 2730 391 2641 302 2784 391 2695 302 2836 391 2748 302 2889 391 2801 302 2942 391 2854 302 2996 391 2908 302 3049 391 2960 302 3101 391 3013 302 3154 391 3066 302 3208 391 3120 302 3261 391 3173 302 3314 391 3225 302 3366 391 3278 302 3420 391 3332 302 3473 391 3385 302 3526 391 3437 302 3578 391 3490 302 3632 391 3544 302 3685 391 3597 302 3738 391 3650 302 3791 391 3702 355 3791 391 3756
% 
\special{pn 4}%
\special{pa 302 2253}%
\special{pa 355 2200}%
\special{fp}%
\special{pa 302 2305}%
\special{pa 391 2217}%
\special{fp}%
\special{pa 302 2359}%
\special{pa 391 2271}%
\special{fp}%
\special{pa 302 2412}%
\special{pa 391 2324}%
\special{fp}%
\special{pa 302 2465}%
\special{pa 391 2377}%
\special{fp}%
\special{pa 302 2518}%
\special{pa 391 2429}%
\special{fp}%
\special{pa 302 2572}%
\special{pa 391 2483}%
\special{fp}%
\special{pa 302 2624}%
\special{pa 391 2536}%
\special{fp}%
\special{pa 302 2677}%
\special{pa 391 2589}%
\special{fp}%
\special{pa 302 2730}%
\special{pa 391 2641}%
\special{fp}%
\special{pa 302 2784}%
\special{pa 391 2695}%
\special{fp}%
\special{pa 302 2836}%
\special{pa 391 2748}%
\special{fp}%
\special{pa 302 2889}%
\special{pa 391 2801}%
\special{fp}%
\special{pa 302 2942}%
\special{pa 391 2854}%
\special{fp}%
\special{pa 302 2996}%
\special{pa 391 2908}%
\special{fp}%
\special{pa 302 3049}%
\special{pa 391 2960}%
\special{fp}%
\special{pa 302 3101}%
\special{pa 391 3013}%
\special{fp}%
\special{pa 302 3154}%
\special{pa 391 3066}%
\special{fp}%
\special{pa 302 3208}%
\special{pa 391 3120}%
\special{fp}%
\special{pa 302 3261}%
\special{pa 391 3173}%
\special{fp}%
\special{pa 302 3314}%
\special{pa 391 3225}%
\special{fp}%
\special{pa 302 3366}%
\special{pa 391 3278}%
\special{fp}%
\special{pa 302 3420}%
\special{pa 391 3332}%
\special{fp}%
\special{pa 302 3473}%
\special{pa 391 3385}%
\special{fp}%
\special{pa 302 3526}%
\special{pa 391 3437}%
\special{fp}%
\special{pa 302 3578}%
\special{pa 391 3490}%
\special{fp}%
\special{pa 302 3632}%
\special{pa 391 3544}%
\special{fp}%
\special{pa 302 3685}%
\special{pa 391 3597}%
\special{fp}%
\special{pa 302 3738}%
\special{pa 391 3650}%
\special{fp}%
\special{pa 302 3791}%
\special{pa 391 3702}%
\special{fp}%
\special{pa 355 3791}%
\special{pa 391 3756}%
\special{fp}%
% LINE 2 0 3 0 Black White  
% 2 1311 2881 1311 3366
% 
\special{pn 8}%
\special{pa 1311 2881}%
\special{pa 1311 3366}%
\special{fp}%
% CIRCLE 2 0 2 0 Black White  
% 4 1311 3366 1363 3366 1363 3366 1363 3366
% 
\special{sh 0}%
\special{ia 1311 3366 52 52 0.0000000 6.2831853}%
\special{pn 8}%
\special{ar 1311 3366 52 52 0.0000000 6.2831853}%
% DOT 0 0 3 0 Black White  
% 1 391 3385
% 
\special{pn 4}%
\special{sh 1}%
\special{ar 391 3385 16 16 0 6.2831853}%
% STR 2 0 3 0 Black White  
% 4 1779 2473 1779 2563 2 0 0 0
% B
\put(17.7900,-25.6300){\makebox(0,0)[lb]{B}}%
% STR 2 0 3 0 Black White  
% 4 278 3341 278 3429 5 0 1 0
% A
\put(2.7800,-34.2900){\makebox(0,0){{\colorbox[named]{White}{A}}}}%
% STR 2 0 3 0 Black White  
% 4 1311 2704 1311 2792 5 0 0 0
% P
\put(13.1100,-27.9200){\makebox(0,0){P}}%
% VECTOR 1 0 3 0 Black White  
% 2 1761 2597 1054 2597
% 
\special{pn 13}%
\special{pa 1761 2597}%
\special{pa 1054 2597}%
\special{fp}%
\special{sh 1}%
\special{pa 1054 2597}%
\special{pa 1121 2617}%
\special{pa 1107 2597}%
\special{pa 1121 2577}%
\special{pa 1054 2597}%
\special{fp}%
% VECTOR 1 0 3 0 Black White  
% 2 391 3385 1098 2774
% 
\special{pn 13}%
\special{pa 391 3385}%
\special{pa 1098 2774}%
\special{fp}%
\special{sh 1}%
\special{pa 1098 2774}%
\special{pa 1034 2802}%
\special{pa 1058 2809}%
\special{pa 1061 2833}%
\special{pa 1098 2774}%
\special{fp}%
% VECTOR 1 0 3 0 Black White  
% 2 1319 2881 1319 3500
% 
\special{pn 13}%
\special{pa 1319 2881}%
\special{pa 1319 3500}%
\special{fp}%
\special{sh 1}%
\special{pa 1319 3500}%
\special{pa 1339 3433}%
\special{pa 1319 3447}%
\special{pa 1299 3433}%
\special{pa 1319 3500}%
\special{fp}%
% VECTOR 1 0 3 0 Black White  
% 2 391 3385 745 2774
% 
\special{pn 13}%
\special{pa 391 3385}%
\special{pa 745 2774}%
\special{fp}%
\special{sh 1}%
\special{pa 745 2774}%
\special{pa 694 2822}%
\special{pa 718 2820}%
\special{pa 729 2842}%
\special{pa 745 2774}%
\special{fp}%
% VECTOR 1 0 3 0 Black White  
% 2 391 3385 1098 3605
% 
\special{pn 13}%
\special{pa 391 3385}%
\special{pa 1098 3605}%
\special{fp}%
\special{sh 1}%
\special{pa 1098 3605}%
\special{pa 1040 3566}%
\special{pa 1047 3589}%
\special{pa 1028 3604}%
\special{pa 1098 3605}%
\special{fp}%
% VECTOR 1 0 3 0 Black White  
% 2 391 3385 745 4189
% 
\special{pn 13}%
\special{pa 391 3385}%
\special{pa 745 4189}%
\special{fp}%
\special{sh 1}%
\special{pa 745 4189}%
\special{pa 736 4120}%
\special{pa 724 4140}%
\special{pa 700 4136}%
\special{pa 745 4189}%
\special{fp}%
% STR 2 0 3 0 Black White  
% 4 1302 2483 1302 2572 2 0 0 0
% $\overrightarrow{T_\mathrm{B}}$
\put(13.0200,-25.7200){\makebox(0,0)[lb]{$\overrightarrow{T_\mathrm{B}}$}}%
% STR 2 0 3 0 Black White  
% 4 1355 3190 1355 3278 2 0 0 0
% $W$
\put(13.5500,-32.7800){\makebox(0,0)[lb]{$W$}}%
% STR 2 0 3 0 Black White  
% 4 1355 3190 1355 3278 2 0 0 0
% $\overrightarrow{W}$
\put(13.5500,-32.7800){\makebox(0,0)[lb]{$\overrightarrow{W}$}}%
% STR 2 0 3 0 Black White  
% 4 638 2668 638 2757 2 0 0 0
% \maru{1}
\put(6.3800,-27.5700){\makebox(0,0)[lb]{\maru{1}}}%
% STR 2 0 3 0 Black White  
% 4 1106 2798 1106 2887 2 0 0 0
% \maru{2}
\put(11.0600,-28.8700){\makebox(0,0)[lb]{\maru{2}}}%
% STR 2 0 3 0 Black White  
% 4 1046 3455 1046 3544 2 0 0 0
% \maru{3}
\put(10.4600,-35.4400){\makebox(0,0)[lb]{\maru{3}}}%
% STR 2 0 3 0 Black White  
% 4 771 4101 771 4189 2 0 0 0
% \maru{4}
\put(7.7100,-41.8900){\makebox(0,0)[lb]{\maru{4}}}%
% STR 2 0 3 0 Black White  
% 4 1495 1980 1495 2080 5 0 0 0
% {\bf �}�P}
\put(14.9500,-20.8000){\makebox(0,0){{\bf �}�P}}}%
% STR 2 0 3 0 Black White  
% 4 1486 4138 1486 4260 5 0 0 0
% {\bf �}�Q}
\put(14.8600,-42.6000){\makebox(0,0){{\bf �}�Q}}}%
\end{picture}}%
}
		長さ$\ell $\tanni{m}の一様な軽い棒ABがあり,A端は摩擦のある壁に接し,他端Bは,水平に張られた軽い糸1で壁に接続されて,
		{\bf 図1}に示す状態で支えられている。A端から$\bunsuu{2}{3}\ell $\tanni{m}となる点Pに軽い糸2を介して重さ$W$\tanni{N}のおもりを下げた。
		ただし,棒は壁に垂直な壁に垂直な鉛直面内にある。
			\begin{enumerate}
			\item 糸1が棒におよぼす張力を$\overrightarrow{T_\mathrm{B}}$,おもりの重力を$\overrightarrow{W}$とすると,棒が壁から受ける抗力$\overrightarrow{R}$(垂直抗力と静止摩擦力の合力)
			を表すベクトルとして正しいものは,{\bf 図2}の\maru{1}~\maru{4}のうちどれか。
			\item 糸1の張力の大きさを$T_\mathrm{B}$\tanni{N},糸2の張力の大きさを$W$\tanni{N},壁から棒が受ける垂直抗力の大きさを$N$\tanni{N},静止摩擦力の大きさを$F$\tanni{N}として,
			水平方向と鉛直方向のつり合いの式をそれぞれ書け。
			\item A端まわりの力のモーメントのつり合いの式を(2)の$\ell $,$T_\mathrm{B}$,$W$を用いて表した式として正しいものを,次の\yontaku から1つ選び記号で答えよ。
				\begin{enumerate}
				\item $T_\mathrm{B}\cdot \ell \cos{60\Deg }=W\cdot \bunsuu{2}{3}\ell \sin{60\Deg }$
				\item $T_\mathrm{B}\cdot \ell \sin{60\Deg }=W\cdot \bunsuu{2}{3}\ell \cos{60\Deg }$
				\item $T_\mathrm{B}\cdot \ell \cos{60\Deg }=W\cdot \bunsuu{2}{3}\ell \cos{60\Deg }$
				\item $T_\mathrm{B}\cdot \ell \sin{60\Deg }=W\cdot \bunsuu{2}{3}\ell \sin{60\Deg }$
				\end{enumerate}
			\end{enumerate}
		\end{mawarikomi}
\vfill
	\item 右図のように軽い棒に力を加えた。(1)から(3)は図のO端から合力の作用線の位置までの距離を,(4)は偶力のモーメントを求めよ。
		\begin{center}
		\input{./fig/1中fig09.tex}
		\end{center}
\vfill
\end{enumerate}
\newpage
〔計算用紙ああ〕
\vfill
\end{document}
