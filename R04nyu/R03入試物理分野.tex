\item
\begin{mawarikomi}{260pt}{%WinTpicVersion4.32a
{\unitlength 0.1in%
\begin{picture}(34.2000,67.3500)(33.8000,-71.3500)%
% VECTOR 2 0 3 0 Black White  
% 2 3800 1200 6800 1200
% 
\special{pn 8}%
\special{pa 3800 1200}%
\special{pa 6800 1200}%
\special{fp}%
\special{sh 1}%
\special{pa 6800 1200}%
\special{pa 6733 1180}%
\special{pa 6747 1200}%
\special{pa 6733 1220}%
\special{pa 6800 1200}%
\special{fp}%
% LINE 2 2 3 0 Black White  
% 2 5400 400 5400 2000
% 
\special{pn 8}%
\special{pa 5400 400}%
\special{pa 5400 2000}%
\special{dt 0.045}%
% BOX 2 0 3 0 Black White  
% 2 3800 3600 3750 6800
% 
\special{pn 8}%
\special{pa 3800 3600}%
\special{pa 3750 3600}%
\special{pa 3750 6800}%
\special{pa 3800 6800}%
\special{pa 3800 3600}%
\special{pa 3750 3600}%
\special{fp}%
% VECTOR 2 0 3 0 Black White  
% 2 3800 6800 6800 6800
% 
\special{pn 8}%
\special{pa 3800 6800}%
\special{pa 6800 6800}%
\special{fp}%
\special{sh 1}%
\special{pa 6800 6800}%
\special{pa 6733 6780}%
\special{pa 6747 6800}%
\special{pa 6733 6820}%
\special{pa 6800 6800}%
\special{fp}%
% LINE 0 0 3 0 Black White  
% 4 5400 6800 5400 6800 5400 6800 5400 5200
% 
\special{pn 20}%
\special{pa 5400 6800}%
\special{pa 5400 6800}%
\special{fp}%
\special{pa 5400 6800}%
\special{pa 5400 5200}%
\special{fp}%
% DOT 0 2 3 0 Black White  
% 5 5400 1200 5400 800 5400 400 5400 1600 5400 2000
% 
\special{pn 4}%
\special{sh 1}%
\special{ar 5400 1200 16 16 0 6.2831853}%
\special{sh 1}%
\special{ar 5400 800 16 16 0 6.2831853}%
\special{sh 1}%
\special{ar 5400 400 16 16 0 6.2831853}%
\special{sh 1}%
\special{ar 5400 1600 16 16 0 6.2831853}%
\special{sh 1}%
\special{ar 5400 2000 16 16 0 6.2831853}%
% STR 2 0 3 0 Black White  
% 4 6200 3060 6200 3160 5 0 0 0
% $\times $
\put(62.0000,-31.6000){\makebox(0,0){$\times $}}%
% STR 2 0 3 0 Black White  
% 4 6200 1100 6200 1200 5 0 0 0
% $\times $
\put(62.0000,-12.0000){\makebox(0,0){$\times $}}%
% STR 2 0 3 0 Black White  
% 4 6230 1060 6230 1160 2 0 0 0
% P
\put(62.3000,-11.6000){\makebox(0,0)[lb]{P}}%
% STR 2 0 3 0 Black White  
% 4 6230 3060 6230 3160 2 0 0 0
% P
\put(62.3000,-31.6000){\makebox(0,0)[lb]{P}}%
% STR 2 0 3 0 Black White  
% 4 6800 1200 6800 1300 5 0 0 0
% $x$
\put(68.0000,-13.0000){\makebox(0,0){$x$}}%
% CIRCLE 3 0 3 0 Black White  
% 4 5200 1800 5400 2000 5400 2000 5400 1600
% 
\special{pn 4}%
\special{ar 5200 1800 283 283 5.4977871 0.7853982}%
% STR 2 0 3 0 Black White  
% 4 5520 1700 5520 1800 5 0 1 0
% 10cm
\put(55.2000,-18.0000){\makebox(0,0){{\colorbox[named]{White}{10cm}}}}%
% CIRCLE 3 0 3 0 Black White  
% 4 5195 1400 5395 1600 5395 1600 5395 1200
% 
\special{pn 4}%
\special{ar 5195 1400 283 283 5.4977871 0.7853982}%
% STR 2 0 3 0 Black White  
% 4 5515 1300 5515 1400 5 0 1 0
% 10cm
\put(55.1500,-14.0000){\makebox(0,0){{\colorbox[named]{White}{10cm}}}}%
% CIRCLE 3 0 3 0 Black White  
% 4 5195 1000 5395 1200 5395 1200 5395 800
% 
\special{pn 4}%
\special{ar 5195 1000 283 283 5.4977871 0.7853982}%
% STR 2 0 3 0 Black White  
% 4 5515 900 5515 1000 5 0 1 0
% 10cm
\put(55.1500,-10.0000){\makebox(0,0){{\colorbox[named]{White}{10cm}}}}%
% CIRCLE 3 0 3 0 Black White  
% 4 5195 600 5395 800 5395 800 5395 400
% 
\special{pn 4}%
\special{ar 5195 600 283 283 5.4977871 0.7853982}%
% STR 2 0 3 0 Black White  
% 4 5515 500 5515 600 5 0 1 0
% 10cm
\put(55.1500,-6.0000){\makebox(0,0){{\colorbox[named]{White}{10cm}}}}%
% CIRCLE 3 0 3 0 Black White  
% 4 5790 600 6190 1200 5390 1200 6190 1200
% 
\special{pn 4}%
\special{ar 5790 600 721 721 0.9827937 2.1587989}%
% STR 2 0 3 0 Black White  
% 4 5900 1230 5900 1330 5 0 1 0
% 20cm
\put(59.0000,-13.3000){\makebox(0,0){{\colorbox[named]{White}{20cm}}}}%
% CIRCLE 3 0 3 0 Black White  
% 4 4600 0 3800 1200 3800 1200 5400 1200
% 
\special{pn 4}%
\special{ar 4600 0 1442 1442 0.9827937 2.1587989}%
% STR 2 0 3 0 Black White  
% 4 4600 1350 4600 1450 5 0 1 0
% 40cm
\put(46.0000,-14.5000){\makebox(0,0){{\colorbox[named]{White}{40cm}}}}%
% LINE 2 2 3 0 Black White  
% 2 6200 3200 6200 6800
% 
\special{pn 8}%
\special{pa 6200 3200}%
\special{pa 6200 6800}%
\special{dt 0.045}%
% CIRCLE 3 0 3 0 Black White  
% 4 2000 5000 6200 3200 6200 6800 6200 3200
% 
\special{pn 4}%
\special{ar 2000 5000 4569 4569 5.8782935 0.4048918}%
% STR 2 0 3 0 Black White  
% 4 6570 4900 6570 5000 5 0 1 0
% 90cm
\put(65.7000,-50.0000){\makebox(0,0){{\colorbox[named]{White}{90cm}}}}%
% VECTOR 2 0 3 0 Black White  
% 2 3800 6800 3800 3000
% 
\special{pn 8}%
\special{pa 3800 6800}%
\special{pa 3800 3000}%
\special{fp}%
\special{sh 1}%
\special{pa 3800 3000}%
\special{pa 3780 3067}%
\special{pa 3800 3053}%
\special{pa 3820 3067}%
\special{pa 3800 3000}%
\special{fp}%
% STR 2 0 3 0 Black White  
% 4 3690 3000 3690 3100 5 0 0 0
% $y$
\put(36.9000,-31.0000){\makebox(0,0){$y$}}%
% STR 2 0 3 0 Black White  
% 4 6800 6820 6800 6920 5 0 0 0
% $x$
\put(68.0000,-69.2000){\makebox(0,0){$x$}}%
% STR 2 0 3 0 Black White  
% 4 5200 2100 5200 2200 5 0 0 0
% ��ʐ}
\put(52.0000,-22.0000){\makebox(0,0){��ʐ}}}%
% STR 2 0 3 0 Black White  
% 4 5200 7100 5200 7200 5 0 0 0
% ���ʐ}
\put(52.0000,-72.0000){\makebox(0,0){���ʐ}}}%
% STR 2 0 3 0 Black White  
% 4 3600 5100 3600 5200 5 0 0 0
% ��
\put(36.0000,-52.0000){\makebox(0,0){��}}%
% STR 2 0 3 0 Black White  
% 4 3600 1100 3600 1200 5 0 0 0
% ��
\put(36.0000,-12.0000){\makebox(0,0){��}}%
% STR 2 0 3 0 Black White  
% 4 5400 5010 5400 5110 5 0 0 0
% �_
\put(54.0000,-51.1000){\makebox(0,0){�_}}%
% CIRCLE 3 0 3 0 Black White  
% 4 6200 6000 5400 5200 5400 5200 5400 6800
% 
\special{pn 4}%
\special{ar 6200 6000 1131 1131 2.3561945 3.9269908}%
% STR 2 0 3 0 Black White  
% 4 5080 5900 5080 6000 5 0 1 0
% 40cm
\put(50.8000,-60.0000){\makebox(0,0){{\colorbox[named]{White}{40cm}}}}%
% CIRCLE 3 0 3 0 Black White  
% 4 1500 5200 3750 3600 3750 6800 3750 3600
% 
\special{pn 4}%
\special{ar 1500 5200 2761 2761 5.6650411 0.6181442}%
% STR 2 0 3 0 Black White  
% 4 4240 5100 4240 5200 5 0 1 0
% 80cm
\put(42.4000,-52.0000){\makebox(0,0){{\colorbox[named]{White}{80cm}}}}%
% BOX 2 0 3 0 Black White  
% 2 3750 600 3800 1800
% 
\special{pn 8}%
\special{pa 3750 600}%
\special{pa 3800 600}%
\special{pa 3800 1800}%
\special{pa 3750 1800}%
\special{pa 3750 600}%
\special{pa 3800 600}%
\special{fp}%
% CIRCLE 3 0 3 0 Black White  
% 4 4000 1500 3750 1200 3750 1200 3750 1800
% 
\special{pn 4}%
\special{ar 4000 1500 391 391 2.2655346 4.0176507}%
% STR 2 0 3 0 Black White  
% 4 3510 1400 3510 1500 5 0 1 0
% 15cm
\put(35.1000,-15.0000){\makebox(0,0){{\colorbox[named]{White}{15cm}}}}%
% CIRCLE 3 0 3 0 Black White  
% 4 3995 900 3745 600 3745 600 3745 1200
% 
\special{pn 4}%
\special{ar 3995 900 391 391 2.2655346 4.0176507}%
% STR 2 0 3 0 Black White  
% 4 3505 800 3505 900 5 0 1 0
% 15cm
\put(35.0500,-9.0000){\makebox(0,0){{\colorbox[named]{White}{15cm}}}}%
% STR 2 0 3 0 Black White  
% 4 3835 1070 3835 1170 2 0 0 0
% O
\put(38.3500,-11.7000){\makebox(0,0)[lb]{O}}%
% STR 2 0 3 0 Black White  
% 4 3785 6730 3785 6830 4 0 0 0
% O
\put(37.8500,-68.3000){\makebox(0,0)[rt]{O}}%
% CIRCLE 3 0 3 0 Black White  
% 4 5790 6200 6190 6800 5390 6800 6190 6800
% 
\special{pn 4}%
\special{ar 5790 6200 721 721 0.9827937 2.1587989}%
% STR 2 0 3 0 Black White  
% 4 5800 6830 5800 6930 5 0 1 0
% 20cm
\put(58.0000,-69.3000){\makebox(0,0){{\colorbox[named]{White}{20cm}}}}%
% CIRCLE 3 0 3 0 Black White  
% 4 4600 5600 3800 6800 3800 6800 5400 6800
% 
\special{pn 4}%
\special{ar 4600 5600 1442 1442 0.9827937 2.1587989}%
% STR 2 0 3 0 Black White  
% 4 4600 6950 4600 7050 5 0 1 0
% 40cm
\put(46.0000,-70.5000){\makebox(0,0){{\colorbox[named]{White}{40cm}}}}%
% STR 2 0 3 0 Black White  
% 4 6200 4100 6200 4200 5 0 0 0
% $\times $
\put(62.0000,-42.0000){\makebox(0,0){$\times $}}%
% STR 2 0 3 0 Black White  
% 4 6600 3060 6600 3160 5 0 0 0
% $\times $
\put(66.0000,-31.6000){\makebox(0,0){$\times $}}%
% STR 2 0 3 0 Black White  
% 4 6630 3060 6630 3160 2 0 0 0
% Q
\put(66.3000,-31.6000){\makebox(0,0)[lb]{Q}}%
% LINE 2 2 3 0 Black White  
% 2 6630 3160 6200 3160
% 
\special{pn 8}%
\special{pa 6630 3160}%
\special{pa 6200 3160}%
\special{dt 0.045}%
% STR 2 0 3 0 Black White  
% 4 6230 4060 6230 4160 2 0 0 0
% R
\put(62.3000,-41.6000){\makebox(0,0)[lb]{R}}%
% STR 2 0 3 0 Black White  
% 4 6630 1060 6630 1160 2 0 0 0
% Q
\put(66.3000,-11.6000){\makebox(0,0)[lb]{Q}}%
% STR 2 0 3 0 Black White  
% 4 6600 1100 6600 1200 5 0 0 0
% $\times $
\put(66.0000,-12.0000){\makebox(0,0){$\times $}}%
\end{picture}}%
}
{\bf 幅30cm,高さ80cmの鏡を右の上面図,側面図のように,鏡の下端中央を原点Oとして配置した。鏡から離れる向きに$x$軸,
鏡の高さ方向に$y$軸をとる。鏡から$x$軸方向に40cmの位置に,10cm間隔で細い棒を5本,鏡と平行に並べた。
中央の棒から$x$軸正の方向へ20cm離れた地点から,$y$軸方向へ90cmの高さの点をPとし,この位置から鏡に映った細い棒を観察する。
以下の問に答えなさい。
}
	\begin{Enumerate}
	\item 鏡に映った細い棒は何本か。
	\item 鏡に映った像を,点Pと高さは同じで,鏡からより離れた位置Q($x>60$〔cm〕)から観察した場合,Pから観察した場合と比べて,
	細い棒の間隔はどのように変化するか。次の\santaku から,観察結果として正しいものを選び,記号で答えなさい。
		\begin{enumerate}
		\item 狭くなる。
		\item 広くなる。
		\item 変わらない。
		\end{enumerate}
	\end{Enumerate}
	点Pから見て,鏡に映った細い棒の上端の位置(鏡上の位置)に,それぞれ,ペンで目印を付けた。
	\begin{Enumerate*}
	\item 点Pと$x$座標が等しく,より低い位置R($40<y<90$〔cm〕)から観察し,鏡上の細い棒に目印を付けた場合,
	Pから観察した場合と比べて,上端の目印の$y$座標と細い棒の間隔はどのように変化するか。
	次の\rokutaku から,観察結果として正しいものを選び,記号で答えなさい。
		\begin{enumerate}
		\item 上端の目印の$y$座標は増加し,細い棒の間隔は狭くなる。
		\item 上端の目印の$y$座標は増加し,細い棒の間隔は広くなる。
		\item 上端の目印の$y$座標は増加し,細い棒の間隔は変わらない。
		\item 上端の目印の$y$座標は減少し,細い棒の間隔は狭くなる。
		\item 上端の目印の$y$座標は減少し,細い棒の間隔は広くなる。
		\item 上端の目印の$y$座標は減少し,細い棒の間隔は変わらない。
		\end{enumerate}
	\item Pから観察した場合,目印上端の,左右の間隔は何cmか。
	\item Pから観察した場合,目印上端の,$y$座標は何cmか。
\newpage
	\end{Enumerate*}
\end{mawarikomi}
\begin{mawarikomi}(5pt,0pt){350pt}{%WinTpicVersion4.32a
{\unitlength 0.1in%
\begin{picture}(48.4700,33.2800)(6.6000,-34.7800)%
% LINE 2 0 3 0 Black White  
% 2 2600 1563 4369 1563
% 
\special{pn 8}%
\special{pa 2600 1563}%
\special{pa 4369 1563}%
\special{fp}%
% LINE 2 0 3 0 Black White  
% 2 4369 1563 5507 679
% 
\special{pn 8}%
\special{pa 4369 1563}%
\special{pa 5507 679}%
\special{fp}%
% LINE 2 0 3 0 Black White  
% 2 2600 1563 3737 679
% 
\special{pn 8}%
\special{pa 2600 1563}%
\special{pa 3737 679}%
\special{fp}%
% LINE 2 1 3 0 Black White  
% 2 3485 1563 4622 679
% 
\special{pn 8}%
\special{pa 3485 1563}%
\special{pa 4622 679}%
\special{da 0.030}%
% LINE 2 1 3 0 Black White  
% 2 3485 552 3485 1563
% 
\special{pn 8}%
\special{pa 3485 552}%
\special{pa 3485 1563}%
\special{da 0.030}%
% STR 2 0 3 0 Black White  
% 4 3485 552 3485 615 5 0 0 0
% $\times $
\put(34.8500,-6.1500){\makebox(0,0){$\times $}}%
% STR 2 0 3 0 Black White  
% 4 3466 540 3466 603 3 0 0 0
% P
\put(34.6600,-6.0300){\makebox(0,0)[rb]{P}}%
% LINE 2 0 3 0 Black White  
% 2 3710 1360 3710 502
% 
\special{pn 8}%
\special{pa 3710 1360}%
\special{pa 3710 502}%
\special{fp}%
% LINE 2 0 3 1 Black White  
% 2 4342 1360 3712 1360
% 
\special{pn 8}%
\special{pa 4342 1360}%
\special{pa 3712 1360}%
\special{fp}%
% LINE 2 0 3 2 Black White  
% 2 4342 502 4342 1360
% 
\special{pn 8}%
\special{pa 4342 502}%
\special{pa 4342 1360}%
\special{fp}%
% LINE 2 0 3 3 Black White  
% 2 3710 502 4342 502
% 
\special{pn 8}%
\special{pa 3710 502}%
\special{pa 4342 502}%
\special{fp}%
% LINE 3 0 3 4 Black White  
% 48 4340 780 3760 1360 4340 720 3710 1350 4340 660 3710 1290 4340 600 3710 1230 4340 540 3710 1170 4320 500 3710 1110 4260 500 3710 1050 4200 500 3710 990 4140 500 3710 930 4080 500 3710 870 4020 500 3710 810 3960 500 3710 750 3900 500 3710 690 3840 500 3710 630 3780 500 3710 570 4340 840 3820 1360 4340 900 3880 1360 4340 960 3940 1360 4340 1020 4000 1360 4340 1080 4060 1360 4340 1140 4120 1360 4340 1200 4180 1360 4340 1260 4240 1360 4340 1320 4300 1360
% 
\special{pn 4}%
\special{pa 4340 780}%
\special{pa 3760 1360}%
\special{fp}%
\special{pa 4340 720}%
\special{pa 3710 1350}%
\special{fp}%
\special{pa 4340 660}%
\special{pa 3710 1290}%
\special{fp}%
\special{pa 4340 600}%
\special{pa 3710 1230}%
\special{fp}%
\special{pa 4340 540}%
\special{pa 3710 1170}%
\special{fp}%
\special{pa 4320 500}%
\special{pa 3710 1110}%
\special{fp}%
\special{pa 4260 500}%
\special{pa 3710 1050}%
\special{fp}%
\special{pa 4200 500}%
\special{pa 3710 990}%
\special{fp}%
\special{pa 4140 500}%
\special{pa 3710 930}%
\special{fp}%
\special{pa 4080 500}%
\special{pa 3710 870}%
\special{fp}%
\special{pa 4020 500}%
\special{pa 3710 810}%
\special{fp}%
\special{pa 3960 500}%
\special{pa 3710 750}%
\special{fp}%
\special{pa 3900 500}%
\special{pa 3710 690}%
\special{fp}%
\special{pa 3840 500}%
\special{pa 3710 630}%
\special{fp}%
\special{pa 3780 500}%
\special{pa 3710 570}%
\special{fp}%
\special{pa 4340 840}%
\special{pa 3820 1360}%
\special{fp}%
\special{pa 4340 900}%
\special{pa 3880 1360}%
\special{fp}%
\special{pa 4340 960}%
\special{pa 3940 1360}%
\special{fp}%
\special{pa 4340 1020}%
\special{pa 4000 1360}%
\special{fp}%
\special{pa 4340 1080}%
\special{pa 4060 1360}%
\special{fp}%
\special{pa 4340 1140}%
\special{pa 4120 1360}%
\special{fp}%
\special{pa 4340 1200}%
\special{pa 4180 1360}%
\special{fp}%
\special{pa 4340 1260}%
\special{pa 4240 1360}%
\special{fp}%
\special{pa 4340 1320}%
\special{pa 4300 1360}%
\special{fp}%
% LINE 2 0 3 0 Black White  
% 2 3710 500 3881 367
% 
\special{pn 8}%
\special{pa 3710 500}%
\special{pa 3881 367}%
\special{fp}%
% LINE 2 0 3 0 Black White  
% 2 4350 503 4521 370
% 
\special{pn 8}%
\special{pa 4350 503}%
\special{pa 4521 370}%
\special{fp}%
% LINE 2 0 3 0 Black White  
% 2 4350 1361 4521 1228
% 
\special{pn 8}%
\special{pa 4350 1361}%
\special{pa 4521 1228}%
\special{fp}%
% LINE 2 2 3 0 Black White  
% 2 3889 1228 3889 370
% 
\special{pn 8}%
\special{pa 3889 1228}%
\special{pa 3889 370}%
\special{dt 0.045}%
% LINE 2 2 3 0 Black White  
% 2 4521 1228 3889 1228
% 
\special{pn 8}%
\special{pa 4521 1228}%
\special{pa 3889 1228}%
\special{dt 0.045}%
% LINE 2 0 3 0 Black White  
% 2 4520 370 4520 1228
% 
\special{pn 8}%
\special{pa 4520 370}%
\special{pa 4520 1228}%
\special{fp}%
% LINE 2 0 3 0 Black White  
% 2 3880 370 4512 370
% 
\special{pn 8}%
\special{pa 3880 370}%
\special{pa 4512 370}%
\special{fp}%
% STR 2 0 3 0 Black White  
% 4 3497 1519 3497 1582 1 0 0 0
% O
\put(34.9700,-15.8200){\makebox(0,0)[lt]{O}}%
% LINE 2 0 3 0 Black White  
% 2 4350 2878 3718 2878
% 
\special{pn 8}%
\special{pa 4350 2878}%
\special{pa 3718 2878}%
\special{fp}%
% LINE 2 0 3 0 Black White  
% 2 4287 2435 4919 2435
% 
\special{pn 8}%
\special{pa 4287 2435}%
\special{pa 4919 2435}%
\special{fp}%
% LINE 2 0 3 0 Black White  
% 2 3718 2878 4287 2435
% 
\special{pn 8}%
\special{pa 3718 2878}%
\special{pa 4287 2435}%
\special{fp}%
% LINE 2 0 3 0 Black White  
% 2 4350 2878 4919 2435
% 
\special{pn 8}%
\special{pa 4350 2878}%
\special{pa 4919 2435}%
\special{fp}%
% LINE 2 0 3 0 Black White  
% 2 4350 3130 4919 2688
% 
\special{pn 8}%
\special{pa 4350 3130}%
\special{pa 4919 2688}%
\special{fp}%
% LINE 2 0 3 0 Black White  
% 2 4350 3130 3718 3130
% 
\special{pn 8}%
\special{pa 4350 3130}%
\special{pa 3718 3130}%
\special{fp}%
% LINE 2 0 3 0 Black White  
% 2 4919 2435 4919 2688
% 
\special{pn 8}%
\special{pa 4919 2435}%
\special{pa 4919 2688}%
\special{fp}%
% LINE 2 0 3 0 Black White  
% 2 4363 2878 4363 3130
% 
\special{pn 8}%
\special{pa 4363 2878}%
\special{pa 4363 3130}%
\special{fp}%
% LINE 2 0 3 0 Black White  
% 2 3718 2878 3718 3130
% 
\special{pn 8}%
\special{pa 3718 2878}%
\special{pa 3718 3130}%
\special{fp}%
% LINE 2 0 3 0 Black White  
% 2 2600 3459 4369 3459
% 
\special{pn 8}%
\special{pa 2600 3459}%
\special{pa 4369 3459}%
\special{fp}%
% LINE 2 0 3 0 Black White  
% 2 4369 3459 5507 2574
% 
\special{pn 8}%
\special{pa 4369 3459}%
\special{pa 5507 2574}%
\special{fp}%
% LINE 2 0 3 0 Black White  
% 2 2600 3459 3737 2574
% 
\special{pn 8}%
\special{pa 2600 3459}%
\special{pa 3737 2574}%
\special{fp}%
% LINE 2 1 3 0 Black White  
% 2 3485 3459 4622 2574
% 
\special{pn 8}%
\special{pa 3485 3459}%
\special{pa 4622 2574}%
\special{da 0.030}%
% LINE 2 1 3 0 Black White  
% 2 3485 2448 3485 3459
% 
\special{pn 8}%
\special{pa 3485 2448}%
\special{pa 3485 3459}%
\special{da 0.030}%
% STR 2 0 3 0 Black White  
% 4 3485 2448 3485 2511 5 0 0 0
% $\times $
\put(34.8500,-25.1100){\makebox(0,0){$\times $}}%
% STR 2 0 3 0 Black White  
% 4 3466 2435 3466 2498 3 0 0 0
% P
\put(34.6600,-24.9800){\makebox(0,0)[rb]{P}}%
% STR 2 0 3 0 Black White  
% 4 3497 3415 3497 3478 1 0 0 0
% O
\put(34.9700,-34.7800){\makebox(0,0)[lt]{O}}%
% STR 2 0 3 0 Black White  
% 4 4628 590 4628 653 2 0 0 0
% O$'$
\put(46.2800,-6.5300){\makebox(0,0)[lb]{O$'$}}%
% STR 2 0 3 0 Black White  
% 4 4628 2486 4628 2549 2 0 0 0
% O$'$
\put(46.2800,-25.4900){\makebox(0,0)[lb]{O$'$}}%
% LINE 2 2 3 0 Black White  
% 2 4287 2435 4287 2688
% 
\special{pn 8}%
\special{pa 4287 2435}%
\special{pa 4287 2688}%
\special{dt 0.045}%
% LINE 2 2 3 0 Black White  
% 2 4919 2701 4287 2701
% 
\special{pn 8}%
\special{pa 4919 2701}%
\special{pa 4287 2701}%
\special{dt 0.045}%
% LINE 2 2 3 0 Black White  
% 2 3718 3130 4287 2688
% 
\special{pn 8}%
\special{pa 3718 3130}%
\special{pa 4287 2688}%
\special{dt 0.045}%
% LINE 2 1 3 0 Black White  
% 2 3718 3130 3718 3269
% 
\special{pn 8}%
\special{pa 3718 3130}%
\special{pa 3718 3269}%
\special{da 0.030}%
% BOX 2 0 3 0 Black White  
% 2 660 874 2429 2896
% 
\special{pn 8}%
\special{pa 660 874}%
\special{pa 2429 874}%
\special{pa 2429 2896}%
\special{pa 660 2896}%
\special{pa 660 874}%
\special{pa 2429 874}%
\special{fp}%
% LINE 2 1 3 0 Black White  
% 2 1545 2896 1545 874
% 
\special{pn 8}%
\special{pa 1545 2896}%
\special{pa 1545 874}%
\special{da 0.030}%
% STR 2 0 3 0 Black White  
% 4 1545 2926 1545 2989 5 0 0 0
% O
\put(15.4500,-29.8900){\makebox(0,0){O}}%
% STR 2 0 3 0 Black White  
% 4 1543 722 1543 785 5 0 0 0
% O$'$
\put(15.4300,-7.8500){\makebox(0,0){O$'$}}%
% CIRCLE 0 0 3 0 Black White  
% 4 1543 1435 1795 1435 1543 1688 1543 1182
% 
\special{pn 20}%
\special{ar 1543 1435 252 252 4.7123890 1.5707963}%
% CIRCLE 0 0 3 0 Black White  
% 4 1545 1435 1797 1435 1545 1182 1545 1688
% 
\special{pn 20}%
\special{ar 1545 1435 252 252 1.5707963 4.7123890}%
% BOX 2 0 3 0 Black White  
% 2 1545 2240 2240 2461
% 
\special{pn 8}%
\special{pa 1545 2240}%
\special{pa 2240 2240}%
\special{pa 2240 2461}%
\special{pa 1545 2461}%
\special{pa 1545 2240}%
\special{pa 2240 2240}%
\special{fp}%
% ELLIPSE 0 0 3 0 Black White  
% 4 4281 944 4521 1070 4521 1070 4521 1070
% 
\special{pn 20}%
\special{ar 4281 944 240 126 0.0000000 6.2831853}%
% ELLIPSE 0 0 3 0 Black White  
% 4 4281 2840 4521 2966 4521 2966 4521 2966
% 
\special{pn 20}%
\special{ar 4281 2840 240 126 0.0000000 6.2831853}%
% LINE 2 0 3 0 Black White  
% 4 3737 2574 4117 2574 4919 2574 5507 2574
% 
\special{pn 8}%
\special{pa 3737 2574}%
\special{pa 4117 2574}%
\special{fp}%
\special{pa 4919 2574}%
\special{pa 5507 2574}%
\special{fp}%
% LINE 2 2 3 0 Black White  
% 2 4919 2574 4104 2574
% 
\special{pn 8}%
\special{pa 4919 2574}%
\special{pa 4104 2574}%
\special{dt 0.045}%
% LINE 2 0 3 0 Black White  
% 2 5507 679 4521 679
% 
\special{pn 8}%
\special{pa 5507 679}%
\special{pa 4521 679}%
\special{fp}%
% LINE 2 2 3 0 Black White  
% 2 4521 679 3750 679
% 
\special{pn 8}%
\special{pa 4521 679}%
\special{pa 3750 679}%
\special{dt 0.045}%
% STR 2 0 3 0 Black White  
% 4 1898 2313 1898 2376 5 0 0 0
% \scriptsize{�����̃K���X}
\put(18.9800,-23.7600){\makebox(0,0){\scriptsize{�����̃K���X}}}%
% POLYGON 2 0 3 0 Black White  
% 9 4030 1815 4199 1815 4199 1815 4199 2041 4312 2041 4114 2210 3917 2041 4030 2041 4030 1815
% 
\special{pn 8}%
\special{pa 4030 1815}%
\special{pa 4199 1815}%
\special{pa 4199 2041}%
\special{pa 4312 2041}%
\special{pa 4114 2210}%
\special{pa 3917 2041}%
\special{pa 4030 2041}%
\special{pa 4030 1815}%
\special{pa 4199 1815}%
\special{fp}%
% STR 2 0 3 0 Black White  
% 4 4800 1465 4800 1565 2 0 0 0
% ��Ԃ`
\put(48.0000,-15.6500){\makebox(0,0)[lb]{��Ԃ`}}%
% STR 2 0 3 0 Black White  
% 4 4800 3350 4800 3450 2 0 0 0
% ��Ԃa
\put(48.0000,-34.5000){\makebox(0,0)[lb]{��Ԃa}}%
% STR 2 0 3 0 Black White  
% 4 1540 3060 1540 3160 5 0 0 0
% ��Ԃ`�̏�ʐ}
\put(15.4000,-31.6000){\makebox(0,0){��Ԃ`�̏�ʐ}}}%
% LINE 3 0 3 0 Black White  
% 2 4200 440 4310 280
% 
\special{pn 4}%
\special{pa 4200 440}%
\special{pa 4310 280}%
\special{fp}%
% STR 2 0 3 0 Black White  
% 4 4340 180 4340 280 2 0 0 0
% �����̃K���X
\put(43.4000,-2.8000){\makebox(0,0)[lb]{�����̃K���X}}%
% LINE 3 0 3 0 Black White  
% 2 4340 2500 4450 2340
% 
\special{pn 4}%
\special{pa 4340 2500}%
\special{pa 4450 2340}%
\special{fp}%
% STR 2 0 3 0 Black White  
% 4 4480 2240 4480 2340 2 0 0 0
% �����̃K���X
\put(44.8000,-23.4000){\makebox(0,0)[lb]{�����̃K���X}}%
% LINE 3 0 3 0 Black White  
% 44 3740 2870 3759 2832 3785 2870 3832 2776 3830 2870 3905 2719 3875 2870 3979 2663 3920 2870 4052 2606 3965 2870 4125 2549 4010 2870 4199 2493 4055 2870 4272 2436 4100 2870 4320 2430 4145 2870 4365 2430 4190 2870 4410 2430 4235 2870 4455 2430 4280 2870 4500 2430 4325 2870 4545 2430 4377 2857 4590 2430 4452 2797 4635 2430 4527 2737 4680 2430 4602 2677 4725 2430 4677 2617 4770 2430 4752 2557 4815 2430 4827 2497 4860 2430 4902 2437 4905 2430
% 
\special{pn 4}%
\special{pa 3740 2870}%
\special{pa 3759 2832}%
\special{fp}%
\special{pa 3785 2870}%
\special{pa 3832 2776}%
\special{fp}%
\special{pa 3830 2870}%
\special{pa 3905 2719}%
\special{fp}%
\special{pa 3875 2870}%
\special{pa 3979 2663}%
\special{fp}%
\special{pa 3920 2870}%
\special{pa 4052 2606}%
\special{fp}%
\special{pa 3965 2870}%
\special{pa 4125 2549}%
\special{fp}%
\special{pa 4010 2870}%
\special{pa 4199 2493}%
\special{fp}%
\special{pa 4055 2870}%
\special{pa 4272 2436}%
\special{fp}%
\special{pa 4100 2870}%
\special{pa 4320 2430}%
\special{fp}%
\special{pa 4145 2870}%
\special{pa 4365 2430}%
\special{fp}%
\special{pa 4190 2870}%
\special{pa 4410 2430}%
\special{fp}%
\special{pa 4235 2870}%
\special{pa 4455 2430}%
\special{fp}%
\special{pa 4280 2870}%
\special{pa 4500 2430}%
\special{fp}%
\special{pa 4325 2870}%
\special{pa 4545 2430}%
\special{fp}%
\special{pa 4377 2857}%
\special{pa 4590 2430}%
\special{fp}%
\special{pa 4452 2797}%
\special{pa 4635 2430}%
\special{fp}%
\special{pa 4527 2737}%
\special{pa 4680 2430}%
\special{fp}%
\special{pa 4602 2677}%
\special{pa 4725 2430}%
\special{fp}%
\special{pa 4677 2617}%
\special{pa 4770 2430}%
\special{fp}%
\special{pa 4752 2557}%
\special{pa 4815 2430}%
\special{fp}%
\special{pa 4827 2497}%
\special{pa 4860 2430}%
\special{fp}%
\special{pa 4902 2437}%
\special{pa 4905 2430}%
\special{fp}%
\end{picture}}%
}
\begin{Enumerate*}
\item 紙面に〇印を描き,直方体ガラスを状態Aのように紙面に対して垂直に立て,点Pから紙面の〇印を観察する。このとき,〇印の右半分は図のように直方体ガラスの斜線部の面を通して,左半分は直接観察することができた。この状態から,直方体ガラスを90度回転させ,紙面から少し浮かせた状態(状態B)にした。直方体ガラスを状態Aから状態Bに変化させる間,点Pからの観察を続けたとすると,〇印はどのように見え方が変化するか。観察結果として正しいものを,次の\yontaku から選び,記号で答えなさい。
\end{Enumerate*}
\end{mawarikomi}
\begin{center}
%WinTpicVersion4.32a
{\unitlength 0.1in%
\begin{picture}(44.6000,37.5500)(2.9000,-41.1000)%
% LINE 2 0 3 0 Black White  
% 2 552 2092 552 670
% 
\special{pn 8}%
\special{pa 552 2092}%
\special{pa 552 670}%
\special{fp}%
% CIRCLE 0 0 3 0 Black White  
% 4 560 1620 812 1620 560 1873 560 1367
% 
\special{pn 20}%
\special{ar 560 1620 252 252 4.7123890 1.5707963}%
% CIRCLE 0 0 3 0 Black White  
% 4 542 1423 794 1423 542 1170 542 1676
% 
\special{pn 20}%
\special{ar 542 1423 252 252 1.5707963 4.7123890}%
% STR 2 0 3 0 Black White  
% 4 560 320 560 420 5 0 0 0
% {\bf �A}
\put(5.6000,-4.2000){\makebox(0,0){{\bf �A}}}%
% LINE 2 0 3 0 Black White  
% 2 552 4092 552 2670
% 
\special{pn 8}%
\special{pa 552 4092}%
\special{pa 552 2670}%
\special{fp}%
% CIRCLE 0 0 3 0 Black White  
% 4 560 3490 812 3490 560 3743 560 3237
% 
\special{pn 20}%
\special{ar 560 3490 252 252 4.7123890 1.5707963}%
% CIRCLE 0 0 3 0 Black White  
% 4 542 3423 794 3423 542 3170 542 3676
% 
\special{pn 20}%
\special{ar 542 3423 252 252 1.5707963 4.7123890}%
% POLYGON 2 0 3 0 Black White  
% 9 489 2240 595 2240 595 2240 595 2383 667 2383 542 2490 417 2383 489 2383 489 2240
% 
\special{pn 8}%
\special{pa 489 2240}%
\special{pa 595 2240}%
\special{pa 595 2383}%
\special{pa 667 2383}%
\special{pa 542 2490}%
\special{pa 417 2383}%
\special{pa 489 2383}%
\special{pa 489 2240}%
\special{pa 595 2240}%
\special{fp}%
% LINE 2 0 3 0 Black White  
% 2 1752 2092 1752 670
% 
\special{pn 8}%
\special{pa 1752 2092}%
\special{pa 1752 670}%
\special{fp}%
% CIRCLE 0 0 3 0 Black White  
% 4 1760 1620 2012 1620 1760 1873 1760 1367
% 
\special{pn 20}%
\special{ar 1760 1620 252 252 4.7123890 1.5707963}%
% CIRCLE 0 0 3 0 Black White  
% 4 1742 1423 1994 1423 1742 1170 1742 1676
% 
\special{pn 20}%
\special{ar 1742 1423 252 252 1.5707963 4.7123890}%
% STR 2 0 3 0 Black White  
% 4 1760 320 1760 420 5 0 0 0
% {\bf �C}
\put(17.6000,-4.2000){\makebox(0,0){{\bf �C}}}%
% LINE 2 0 3 0 Black White  
% 2 1752 4092 1752 2670
% 
\special{pn 8}%
\special{pa 1752 4092}%
\special{pa 1752 2670}%
\special{fp}%
% CIRCLE 0 0 3 0 Black White  
% 4 1760 3220 2012 3220 1760 3473 1760 2967
% 
\special{pn 20}%
\special{ar 1760 3220 252 252 4.7123890 1.5707963}%
% CIRCLE 0 0 3 0 Black White  
% 4 1742 3423 1994 3423 1742 3170 1742 3676
% 
\special{pn 20}%
\special{ar 1742 3423 252 252 1.5707963 4.7123890}%
% POLYGON 2 0 3 0 Black White  
% 9 1689 2240 1795 2240 1795 2240 1795 2383 1867 2383 1742 2490 1617 2383 1689 2383 1689 2240
% 
\special{pn 8}%
\special{pa 1689 2240}%
\special{pa 1795 2240}%
\special{pa 1795 2383}%
\special{pa 1867 2383}%
\special{pa 1742 2490}%
\special{pa 1617 2383}%
\special{pa 1689 2383}%
\special{pa 1689 2240}%
\special{pa 1795 2240}%
\special{fp}%
% LINE 2 0 3 0 Black White  
% 2 2952 2092 2952 670
% 
\special{pn 8}%
\special{pa 2952 2092}%
\special{pa 2952 670}%
\special{fp}%
% CIRCLE 0 0 3 0 Black White  
% 4 2960 1220 3212 1220 2960 1473 2960 967
% 
\special{pn 20}%
\special{ar 2960 1220 252 252 4.7123890 1.5707963}%
% CIRCLE 0 0 3 0 Black White  
% 4 2942 1423 3194 1423 2942 1170 2942 1676
% 
\special{pn 20}%
\special{ar 2942 1423 252 252 1.5707963 4.7123890}%
% STR 2 0 3 0 Black White  
% 4 2960 320 2960 420 5 0 0 0
% {\bf �E}
\put(29.6000,-4.2000){\makebox(0,0){{\bf �E}}}%
% LINE 2 0 3 0 Black White  
% 2 2952 4092 2952 2670
% 
\special{pn 8}%
\special{pa 2952 4092}%
\special{pa 2952 2670}%
\special{fp}%
% CIRCLE 0 0 3 0 Black White  
% 4 2960 3620 3212 3620 2960 3873 2960 3367
% 
\special{pn 20}%
\special{ar 2960 3620 252 252 4.7123890 1.5707963}%
% CIRCLE 0 0 3 0 Black White  
% 4 2942 3423 3194 3423 2942 3170 2942 3676
% 
\special{pn 20}%
\special{ar 2942 3423 252 252 1.5707963 4.7123890}%
% POLYGON 2 0 3 0 Black White  
% 9 2889 2240 2995 2240 2995 2240 2995 2383 3067 2383 2942 2490 2817 2383 2889 2383 2889 2240
% 
\special{pn 8}%
\special{pa 2889 2240}%
\special{pa 2995 2240}%
\special{pa 2995 2383}%
\special{pa 3067 2383}%
\special{pa 2942 2490}%
\special{pa 2817 2383}%
\special{pa 2889 2383}%
\special{pa 2889 2240}%
\special{pa 2995 2240}%
\special{fp}%
% LINE 2 0 3 0 Black White  
% 2 4152 2092 4152 670
% 
\special{pn 8}%
\special{pa 4152 2092}%
\special{pa 4152 670}%
\special{fp}%
% CIRCLE 0 0 3 0 Black White  
% 4 4160 1220 4412 1220 4160 1473 4160 967
% 
\special{pn 20}%
\special{ar 4160 1220 252 252 4.7123890 1.5707963}%
% CIRCLE 0 0 3 0 Black White  
% 4 4142 1423 4394 1423 4142 1170 4142 1676
% 
\special{pn 20}%
\special{ar 4142 1423 252 252 1.5707963 4.7123890}%
% STR 2 0 3 0 Black White  
% 4 4160 320 4160 420 5 0 0 0
% {\bf �G}
\put(41.6000,-4.2000){\makebox(0,0){{\bf �G}}}%
% LINE 2 0 3 0 Black White  
% 2 4152 4092 4152 2670
% 
\special{pn 8}%
\special{pa 4152 4092}%
\special{pa 4152 2670}%
\special{fp}%
% CIRCLE 0 0 3 0 Black White  
% 4 4160 3360 4412 3360 4160 3613 4160 3107
% 
\special{pn 20}%
\special{ar 4160 3360 252 252 4.7123890 1.5707963}%
% CIRCLE 0 0 3 0 Black White  
% 4 4142 3423 4394 3423 4142 3170 4142 3676
% 
\special{pn 20}%
\special{ar 4142 3423 252 252 1.5707963 4.7123890}%
% POLYGON 2 0 3 0 Black White  
% 9 4089 2240 4195 2240 4195 2240 4195 2383 4267 2383 4142 2490 4017 2383 4089 2383 4089 2240
% 
\special{pn 8}%
\special{pa 4089 2240}%
\special{pa 4195 2240}%
\special{pa 4195 2383}%
\special{pa 4267 2383}%
\special{pa 4142 2490}%
\special{pa 4017 2383}%
\special{pa 4089 2383}%
\special{pa 4089 2240}%
\special{pa 4195 2240}%
\special{fp}%
% LINE 3 0 3 0 Black White  
% 58 550 700 560 680 550 790 605 680 550 880 650 680 550 970 695 680 550 1060 740 680 550 1150 785 680 550 1240 830 680 550 1330 875 680 550 1420 920 680 550 1510 965 680 550 1600 1010 680 550 1690 1055 680 550 1780 1100 680 550 1870 1145 680 550 1960 1150 760 550 2050 1150 850 580 2080 1150 940 625 2080 1150 1030 670 2080 1150 1120 715 2080 1150 1210 760 2080 1150 1300 805 2080 1150 1390 850 2080 1150 1480 895 2080 1150 1570 940 2080 1150 1660 985 2080 1150 1750 1030 2080 1150 1840 1075 2080 1150 1930 1120 2080 1150 2020
% 
\special{pn 4}%
\special{pa 550 700}%
\special{pa 560 680}%
\special{fp}%
\special{pa 550 790}%
\special{pa 605 680}%
\special{fp}%
\special{pa 550 880}%
\special{pa 650 680}%
\special{fp}%
\special{pa 550 970}%
\special{pa 695 680}%
\special{fp}%
\special{pa 550 1060}%
\special{pa 740 680}%
\special{fp}%
\special{pa 550 1150}%
\special{pa 785 680}%
\special{fp}%
\special{pa 550 1240}%
\special{pa 830 680}%
\special{fp}%
\special{pa 550 1330}%
\special{pa 875 680}%
\special{fp}%
\special{pa 550 1420}%
\special{pa 920 680}%
\special{fp}%
\special{pa 550 1510}%
\special{pa 965 680}%
\special{fp}%
\special{pa 550 1600}%
\special{pa 1010 680}%
\special{fp}%
\special{pa 550 1690}%
\special{pa 1055 680}%
\special{fp}%
\special{pa 550 1780}%
\special{pa 1100 680}%
\special{fp}%
\special{pa 550 1870}%
\special{pa 1145 680}%
\special{fp}%
\special{pa 550 1960}%
\special{pa 1150 760}%
\special{fp}%
\special{pa 550 2050}%
\special{pa 1150 850}%
\special{fp}%
\special{pa 580 2080}%
\special{pa 1150 940}%
\special{fp}%
\special{pa 625 2080}%
\special{pa 1150 1030}%
\special{fp}%
\special{pa 670 2080}%
\special{pa 1150 1120}%
\special{fp}%
\special{pa 715 2080}%
\special{pa 1150 1210}%
\special{fp}%
\special{pa 760 2080}%
\special{pa 1150 1300}%
\special{fp}%
\special{pa 805 2080}%
\special{pa 1150 1390}%
\special{fp}%
\special{pa 850 2080}%
\special{pa 1150 1480}%
\special{fp}%
\special{pa 895 2080}%
\special{pa 1150 1570}%
\special{fp}%
\special{pa 940 2080}%
\special{pa 1150 1660}%
\special{fp}%
\special{pa 985 2080}%
\special{pa 1150 1750}%
\special{fp}%
\special{pa 1030 2080}%
\special{pa 1150 1840}%
\special{fp}%
\special{pa 1075 2080}%
\special{pa 1150 1930}%
\special{fp}%
\special{pa 1120 2080}%
\special{pa 1150 2020}%
\special{fp}%
% LINE 3 0 3 0 Black White  
% 58 1750 700 1760 680 1750 790 1805 680 1750 880 1850 680 1750 970 1895 680 1750 1060 1940 680 1750 1150 1985 680 1750 1240 2030 680 1750 1330 2075 680 1750 1420 2120 680 1750 1510 2165 680 1750 1600 2210 680 1750 1690 2255 680 1750 1780 2300 680 1750 1870 2345 680 1750 1960 2350 760 1750 2050 2350 850 1780 2080 2350 940 1825 2080 2350 1030 1870 2080 2350 1120 1915 2080 2350 1210 1960 2080 2350 1300 2005 2080 2350 1390 2050 2080 2350 1480 2095 2080 2350 1570 2140 2080 2350 1660 2185 2080 2350 1750 2230 2080 2350 1840 2275 2080 2350 1930 2320 2080 2350 2020
% 
\special{pn 4}%
\special{pa 1750 700}%
\special{pa 1760 680}%
\special{fp}%
\special{pa 1750 790}%
\special{pa 1805 680}%
\special{fp}%
\special{pa 1750 880}%
\special{pa 1850 680}%
\special{fp}%
\special{pa 1750 970}%
\special{pa 1895 680}%
\special{fp}%
\special{pa 1750 1060}%
\special{pa 1940 680}%
\special{fp}%
\special{pa 1750 1150}%
\special{pa 1985 680}%
\special{fp}%
\special{pa 1750 1240}%
\special{pa 2030 680}%
\special{fp}%
\special{pa 1750 1330}%
\special{pa 2075 680}%
\special{fp}%
\special{pa 1750 1420}%
\special{pa 2120 680}%
\special{fp}%
\special{pa 1750 1510}%
\special{pa 2165 680}%
\special{fp}%
\special{pa 1750 1600}%
\special{pa 2210 680}%
\special{fp}%
\special{pa 1750 1690}%
\special{pa 2255 680}%
\special{fp}%
\special{pa 1750 1780}%
\special{pa 2300 680}%
\special{fp}%
\special{pa 1750 1870}%
\special{pa 2345 680}%
\special{fp}%
\special{pa 1750 1960}%
\special{pa 2350 760}%
\special{fp}%
\special{pa 1750 2050}%
\special{pa 2350 850}%
\special{fp}%
\special{pa 1780 2080}%
\special{pa 2350 940}%
\special{fp}%
\special{pa 1825 2080}%
\special{pa 2350 1030}%
\special{fp}%
\special{pa 1870 2080}%
\special{pa 2350 1120}%
\special{fp}%
\special{pa 1915 2080}%
\special{pa 2350 1210}%
\special{fp}%
\special{pa 1960 2080}%
\special{pa 2350 1300}%
\special{fp}%
\special{pa 2005 2080}%
\special{pa 2350 1390}%
\special{fp}%
\special{pa 2050 2080}%
\special{pa 2350 1480}%
\special{fp}%
\special{pa 2095 2080}%
\special{pa 2350 1570}%
\special{fp}%
\special{pa 2140 2080}%
\special{pa 2350 1660}%
\special{fp}%
\special{pa 2185 2080}%
\special{pa 2350 1750}%
\special{fp}%
\special{pa 2230 2080}%
\special{pa 2350 1840}%
\special{fp}%
\special{pa 2275 2080}%
\special{pa 2350 1930}%
\special{fp}%
\special{pa 2320 2080}%
\special{pa 2350 2020}%
\special{fp}%
% LINE 3 0 3 0 Black White  
% 58 2950 700 2960 680 2950 790 3005 680 2950 880 3050 680 2950 970 3095 680 2950 1060 3140 680 2950 1150 3185 680 2950 1240 3230 680 2950 1330 3275 680 2950 1420 3320 680 2950 1510 3365 680 2950 1600 3410 680 2950 1690 3455 680 2950 1780 3500 680 2950 1870 3545 680 2950 1960 3550 760 2950 2050 3550 850 2980 2080 3550 940 3025 2080 3550 1030 3070 2080 3550 1120 3115 2080 3550 1210 3160 2080 3550 1300 3205 2080 3550 1390 3250 2080 3550 1480 3295 2080 3550 1570 3340 2080 3550 1660 3385 2080 3550 1750 3430 2080 3550 1840 3475 2080 3550 1930 3520 2080 3550 2020
% 
\special{pn 4}%
\special{pa 2950 700}%
\special{pa 2960 680}%
\special{fp}%
\special{pa 2950 790}%
\special{pa 3005 680}%
\special{fp}%
\special{pa 2950 880}%
\special{pa 3050 680}%
\special{fp}%
\special{pa 2950 970}%
\special{pa 3095 680}%
\special{fp}%
\special{pa 2950 1060}%
\special{pa 3140 680}%
\special{fp}%
\special{pa 2950 1150}%
\special{pa 3185 680}%
\special{fp}%
\special{pa 2950 1240}%
\special{pa 3230 680}%
\special{fp}%
\special{pa 2950 1330}%
\special{pa 3275 680}%
\special{fp}%
\special{pa 2950 1420}%
\special{pa 3320 680}%
\special{fp}%
\special{pa 2950 1510}%
\special{pa 3365 680}%
\special{fp}%
\special{pa 2950 1600}%
\special{pa 3410 680}%
\special{fp}%
\special{pa 2950 1690}%
\special{pa 3455 680}%
\special{fp}%
\special{pa 2950 1780}%
\special{pa 3500 680}%
\special{fp}%
\special{pa 2950 1870}%
\special{pa 3545 680}%
\special{fp}%
\special{pa 2950 1960}%
\special{pa 3550 760}%
\special{fp}%
\special{pa 2950 2050}%
\special{pa 3550 850}%
\special{fp}%
\special{pa 2980 2080}%
\special{pa 3550 940}%
\special{fp}%
\special{pa 3025 2080}%
\special{pa 3550 1030}%
\special{fp}%
\special{pa 3070 2080}%
\special{pa 3550 1120}%
\special{fp}%
\special{pa 3115 2080}%
\special{pa 3550 1210}%
\special{fp}%
\special{pa 3160 2080}%
\special{pa 3550 1300}%
\special{fp}%
\special{pa 3205 2080}%
\special{pa 3550 1390}%
\special{fp}%
\special{pa 3250 2080}%
\special{pa 3550 1480}%
\special{fp}%
\special{pa 3295 2080}%
\special{pa 3550 1570}%
\special{fp}%
\special{pa 3340 2080}%
\special{pa 3550 1660}%
\special{fp}%
\special{pa 3385 2080}%
\special{pa 3550 1750}%
\special{fp}%
\special{pa 3430 2080}%
\special{pa 3550 1840}%
\special{fp}%
\special{pa 3475 2080}%
\special{pa 3550 1930}%
\special{fp}%
\special{pa 3520 2080}%
\special{pa 3550 2020}%
\special{fp}%
% LINE 3 0 3 0 Black White  
% 58 4150 700 4160 680 4150 790 4205 680 4150 880 4250 680 4150 970 4295 680 4150 1060 4340 680 4150 1150 4385 680 4150 1240 4430 680 4150 1330 4475 680 4150 1420 4520 680 4150 1510 4565 680 4150 1600 4610 680 4150 1690 4655 680 4150 1780 4700 680 4150 1870 4745 680 4150 1960 4750 760 4150 2050 4750 850 4180 2080 4750 940 4225 2080 4750 1030 4270 2080 4750 1120 4315 2080 4750 1210 4360 2080 4750 1300 4405 2080 4750 1390 4450 2080 4750 1480 4495 2080 4750 1570 4540 2080 4750 1660 4585 2080 4750 1750 4630 2080 4750 1840 4675 2080 4750 1930 4720 2080 4750 2020
% 
\special{pn 4}%
\special{pa 4150 700}%
\special{pa 4160 680}%
\special{fp}%
\special{pa 4150 790}%
\special{pa 4205 680}%
\special{fp}%
\special{pa 4150 880}%
\special{pa 4250 680}%
\special{fp}%
\special{pa 4150 970}%
\special{pa 4295 680}%
\special{fp}%
\special{pa 4150 1060}%
\special{pa 4340 680}%
\special{fp}%
\special{pa 4150 1150}%
\special{pa 4385 680}%
\special{fp}%
\special{pa 4150 1240}%
\special{pa 4430 680}%
\special{fp}%
\special{pa 4150 1330}%
\special{pa 4475 680}%
\special{fp}%
\special{pa 4150 1420}%
\special{pa 4520 680}%
\special{fp}%
\special{pa 4150 1510}%
\special{pa 4565 680}%
\special{fp}%
\special{pa 4150 1600}%
\special{pa 4610 680}%
\special{fp}%
\special{pa 4150 1690}%
\special{pa 4655 680}%
\special{fp}%
\special{pa 4150 1780}%
\special{pa 4700 680}%
\special{fp}%
\special{pa 4150 1870}%
\special{pa 4745 680}%
\special{fp}%
\special{pa 4150 1960}%
\special{pa 4750 760}%
\special{fp}%
\special{pa 4150 2050}%
\special{pa 4750 850}%
\special{fp}%
\special{pa 4180 2080}%
\special{pa 4750 940}%
\special{fp}%
\special{pa 4225 2080}%
\special{pa 4750 1030}%
\special{fp}%
\special{pa 4270 2080}%
\special{pa 4750 1120}%
\special{fp}%
\special{pa 4315 2080}%
\special{pa 4750 1210}%
\special{fp}%
\special{pa 4360 2080}%
\special{pa 4750 1300}%
\special{fp}%
\special{pa 4405 2080}%
\special{pa 4750 1390}%
\special{fp}%
\special{pa 4450 2080}%
\special{pa 4750 1480}%
\special{fp}%
\special{pa 4495 2080}%
\special{pa 4750 1570}%
\special{fp}%
\special{pa 4540 2080}%
\special{pa 4750 1660}%
\special{fp}%
\special{pa 4585 2080}%
\special{pa 4750 1750}%
\special{fp}%
\special{pa 4630 2080}%
\special{pa 4750 1840}%
\special{fp}%
\special{pa 4675 2080}%
\special{pa 4750 1930}%
\special{fp}%
\special{pa 4720 2080}%
\special{pa 4750 2020}%
\special{fp}%
% LINE 3 0 3 0 Black White  
% 58 4150 2700 4160 2680 4150 2790 4205 2680 4150 2880 4250 2680 4150 2970 4295 2680 4150 3060 4340 2680 4150 3150 4385 2680 4150 3240 4430 2680 4150 3330 4475 2680 4150 3420 4520 2680 4150 3510 4565 2680 4150 3600 4610 2680 4150 3690 4655 2680 4150 3780 4700 2680 4150 3870 4745 2680 4150 3960 4750 2760 4150 4050 4750 2850 4180 4080 4750 2940 4225 4080 4750 3030 4270 4080 4750 3120 4315 4080 4750 3210 4360 4080 4750 3300 4405 4080 4750 3390 4450 4080 4750 3480 4495 4080 4750 3570 4540 4080 4750 3660 4585 4080 4750 3750 4630 4080 4750 3840 4675 4080 4750 3930 4720 4080 4750 4020
% 
\special{pn 4}%
\special{pa 4150 2700}%
\special{pa 4160 2680}%
\special{fp}%
\special{pa 4150 2790}%
\special{pa 4205 2680}%
\special{fp}%
\special{pa 4150 2880}%
\special{pa 4250 2680}%
\special{fp}%
\special{pa 4150 2970}%
\special{pa 4295 2680}%
\special{fp}%
\special{pa 4150 3060}%
\special{pa 4340 2680}%
\special{fp}%
\special{pa 4150 3150}%
\special{pa 4385 2680}%
\special{fp}%
\special{pa 4150 3240}%
\special{pa 4430 2680}%
\special{fp}%
\special{pa 4150 3330}%
\special{pa 4475 2680}%
\special{fp}%
\special{pa 4150 3420}%
\special{pa 4520 2680}%
\special{fp}%
\special{pa 4150 3510}%
\special{pa 4565 2680}%
\special{fp}%
\special{pa 4150 3600}%
\special{pa 4610 2680}%
\special{fp}%
\special{pa 4150 3690}%
\special{pa 4655 2680}%
\special{fp}%
\special{pa 4150 3780}%
\special{pa 4700 2680}%
\special{fp}%
\special{pa 4150 3870}%
\special{pa 4745 2680}%
\special{fp}%
\special{pa 4150 3960}%
\special{pa 4750 2760}%
\special{fp}%
\special{pa 4150 4050}%
\special{pa 4750 2850}%
\special{fp}%
\special{pa 4180 4080}%
\special{pa 4750 2940}%
\special{fp}%
\special{pa 4225 4080}%
\special{pa 4750 3030}%
\special{fp}%
\special{pa 4270 4080}%
\special{pa 4750 3120}%
\special{fp}%
\special{pa 4315 4080}%
\special{pa 4750 3210}%
\special{fp}%
\special{pa 4360 4080}%
\special{pa 4750 3300}%
\special{fp}%
\special{pa 4405 4080}%
\special{pa 4750 3390}%
\special{fp}%
\special{pa 4450 4080}%
\special{pa 4750 3480}%
\special{fp}%
\special{pa 4495 4080}%
\special{pa 4750 3570}%
\special{fp}%
\special{pa 4540 4080}%
\special{pa 4750 3660}%
\special{fp}%
\special{pa 4585 4080}%
\special{pa 4750 3750}%
\special{fp}%
\special{pa 4630 4080}%
\special{pa 4750 3840}%
\special{fp}%
\special{pa 4675 4080}%
\special{pa 4750 3930}%
\special{fp}%
\special{pa 4720 4080}%
\special{pa 4750 4020}%
\special{fp}%
% LINE 3 0 3 0 Black White  
% 58 2950 2700 2960 2680 2950 2790 3005 2680 2950 2880 3050 2680 2950 2970 3095 2680 2950 3060 3140 2680 2950 3150 3185 2680 2950 3240 3230 2680 2950 3330 3275 2680 2950 3420 3320 2680 2950 3510 3365 2680 2950 3600 3410 2680 2950 3690 3455 2680 2950 3780 3500 2680 2950 3870 3545 2680 2950 3960 3550 2760 2950 4050 3550 2850 2980 4080 3550 2940 3025 4080 3550 3030 3070 4080 3550 3120 3115 4080 3550 3210 3160 4080 3550 3300 3205 4080 3550 3390 3250 4080 3550 3480 3295 4080 3550 3570 3340 4080 3550 3660 3385 4080 3550 3750 3430 4080 3550 3840 3475 4080 3550 3930 3520 4080 3550 4020
% 
\special{pn 4}%
\special{pa 2950 2700}%
\special{pa 2960 2680}%
\special{fp}%
\special{pa 2950 2790}%
\special{pa 3005 2680}%
\special{fp}%
\special{pa 2950 2880}%
\special{pa 3050 2680}%
\special{fp}%
\special{pa 2950 2970}%
\special{pa 3095 2680}%
\special{fp}%
\special{pa 2950 3060}%
\special{pa 3140 2680}%
\special{fp}%
\special{pa 2950 3150}%
\special{pa 3185 2680}%
\special{fp}%
\special{pa 2950 3240}%
\special{pa 3230 2680}%
\special{fp}%
\special{pa 2950 3330}%
\special{pa 3275 2680}%
\special{fp}%
\special{pa 2950 3420}%
\special{pa 3320 2680}%
\special{fp}%
\special{pa 2950 3510}%
\special{pa 3365 2680}%
\special{fp}%
\special{pa 2950 3600}%
\special{pa 3410 2680}%
\special{fp}%
\special{pa 2950 3690}%
\special{pa 3455 2680}%
\special{fp}%
\special{pa 2950 3780}%
\special{pa 3500 2680}%
\special{fp}%
\special{pa 2950 3870}%
\special{pa 3545 2680}%
\special{fp}%
\special{pa 2950 3960}%
\special{pa 3550 2760}%
\special{fp}%
\special{pa 2950 4050}%
\special{pa 3550 2850}%
\special{fp}%
\special{pa 2980 4080}%
\special{pa 3550 2940}%
\special{fp}%
\special{pa 3025 4080}%
\special{pa 3550 3030}%
\special{fp}%
\special{pa 3070 4080}%
\special{pa 3550 3120}%
\special{fp}%
\special{pa 3115 4080}%
\special{pa 3550 3210}%
\special{fp}%
\special{pa 3160 4080}%
\special{pa 3550 3300}%
\special{fp}%
\special{pa 3205 4080}%
\special{pa 3550 3390}%
\special{fp}%
\special{pa 3250 4080}%
\special{pa 3550 3480}%
\special{fp}%
\special{pa 3295 4080}%
\special{pa 3550 3570}%
\special{fp}%
\special{pa 3340 4080}%
\special{pa 3550 3660}%
\special{fp}%
\special{pa 3385 4080}%
\special{pa 3550 3750}%
\special{fp}%
\special{pa 3430 4080}%
\special{pa 3550 3840}%
\special{fp}%
\special{pa 3475 4080}%
\special{pa 3550 3930}%
\special{fp}%
\special{pa 3520 4080}%
\special{pa 3550 4020}%
\special{fp}%
% LINE 3 0 3 0 Black White  
% 58 1750 2700 1760 2680 1750 2790 1805 2680 1750 2880 1850 2680 1750 2970 1895 2680 1750 3060 1940 2680 1750 3150 1985 2680 1750 3240 2030 2680 1750 3330 2075 2680 1750 3420 2120 2680 1750 3510 2165 2680 1750 3600 2210 2680 1750 3690 2255 2680 1750 3780 2300 2680 1750 3870 2345 2680 1750 3960 2350 2760 1750 4050 2350 2850 1780 4080 2350 2940 1825 4080 2350 3030 1870 4080 2350 3120 1915 4080 2350 3210 1960 4080 2350 3300 2005 4080 2350 3390 2050 4080 2350 3480 2095 4080 2350 3570 2140 4080 2350 3660 2185 4080 2350 3750 2230 4080 2350 3840 2275 4080 2350 3930 2320 4080 2350 4020
% 
\special{pn 4}%
\special{pa 1750 2700}%
\special{pa 1760 2680}%
\special{fp}%
\special{pa 1750 2790}%
\special{pa 1805 2680}%
\special{fp}%
\special{pa 1750 2880}%
\special{pa 1850 2680}%
\special{fp}%
\special{pa 1750 2970}%
\special{pa 1895 2680}%
\special{fp}%
\special{pa 1750 3060}%
\special{pa 1940 2680}%
\special{fp}%
\special{pa 1750 3150}%
\special{pa 1985 2680}%
\special{fp}%
\special{pa 1750 3240}%
\special{pa 2030 2680}%
\special{fp}%
\special{pa 1750 3330}%
\special{pa 2075 2680}%
\special{fp}%
\special{pa 1750 3420}%
\special{pa 2120 2680}%
\special{fp}%
\special{pa 1750 3510}%
\special{pa 2165 2680}%
\special{fp}%
\special{pa 1750 3600}%
\special{pa 2210 2680}%
\special{fp}%
\special{pa 1750 3690}%
\special{pa 2255 2680}%
\special{fp}%
\special{pa 1750 3780}%
\special{pa 2300 2680}%
\special{fp}%
\special{pa 1750 3870}%
\special{pa 2345 2680}%
\special{fp}%
\special{pa 1750 3960}%
\special{pa 2350 2760}%
\special{fp}%
\special{pa 1750 4050}%
\special{pa 2350 2850}%
\special{fp}%
\special{pa 1780 4080}%
\special{pa 2350 2940}%
\special{fp}%
\special{pa 1825 4080}%
\special{pa 2350 3030}%
\special{fp}%
\special{pa 1870 4080}%
\special{pa 2350 3120}%
\special{fp}%
\special{pa 1915 4080}%
\special{pa 2350 3210}%
\special{fp}%
\special{pa 1960 4080}%
\special{pa 2350 3300}%
\special{fp}%
\special{pa 2005 4080}%
\special{pa 2350 3390}%
\special{fp}%
\special{pa 2050 4080}%
\special{pa 2350 3480}%
\special{fp}%
\special{pa 2095 4080}%
\special{pa 2350 3570}%
\special{fp}%
\special{pa 2140 4080}%
\special{pa 2350 3660}%
\special{fp}%
\special{pa 2185 4080}%
\special{pa 2350 3750}%
\special{fp}%
\special{pa 2230 4080}%
\special{pa 2350 3840}%
\special{fp}%
\special{pa 2275 4080}%
\special{pa 2350 3930}%
\special{fp}%
\special{pa 2320 4080}%
\special{pa 2350 4020}%
\special{fp}%
% LINE 3 0 3 0 Black White  
% 58 550 2700 560 2680 550 2790 605 2680 550 2880 650 2680 550 2970 695 2680 550 3060 740 2680 550 3150 785 2680 550 3240 830 2680 550 3330 875 2680 550 3420 920 2680 550 3510 965 2680 550 3600 1010 2680 550 3690 1055 2680 550 3780 1100 2680 550 3870 1145 2680 550 3960 1150 2760 550 4050 1150 2850 580 4080 1150 2940 625 4080 1150 3030 670 4080 1150 3120 715 4080 1150 3210 760 4080 1150 3300 805 4080 1150 3390 850 4080 1150 3480 895 4080 1150 3570 940 4080 1150 3660 985 4080 1150 3750 1030 4080 1150 3840 1075 4080 1150 3930 1120 4080 1150 4020
% 
\special{pn 4}%
\special{pa 550 2700}%
\special{pa 560 2680}%
\special{fp}%
\special{pa 550 2790}%
\special{pa 605 2680}%
\special{fp}%
\special{pa 550 2880}%
\special{pa 650 2680}%
\special{fp}%
\special{pa 550 2970}%
\special{pa 695 2680}%
\special{fp}%
\special{pa 550 3060}%
\special{pa 740 2680}%
\special{fp}%
\special{pa 550 3150}%
\special{pa 785 2680}%
\special{fp}%
\special{pa 550 3240}%
\special{pa 830 2680}%
\special{fp}%
\special{pa 550 3330}%
\special{pa 875 2680}%
\special{fp}%
\special{pa 550 3420}%
\special{pa 920 2680}%
\special{fp}%
\special{pa 550 3510}%
\special{pa 965 2680}%
\special{fp}%
\special{pa 550 3600}%
\special{pa 1010 2680}%
\special{fp}%
\special{pa 550 3690}%
\special{pa 1055 2680}%
\special{fp}%
\special{pa 550 3780}%
\special{pa 1100 2680}%
\special{fp}%
\special{pa 550 3870}%
\special{pa 1145 2680}%
\special{fp}%
\special{pa 550 3960}%
\special{pa 1150 2760}%
\special{fp}%
\special{pa 550 4050}%
\special{pa 1150 2850}%
\special{fp}%
\special{pa 580 4080}%
\special{pa 1150 2940}%
\special{fp}%
\special{pa 625 4080}%
\special{pa 1150 3030}%
\special{fp}%
\special{pa 670 4080}%
\special{pa 1150 3120}%
\special{fp}%
\special{pa 715 4080}%
\special{pa 1150 3210}%
\special{fp}%
\special{pa 760 4080}%
\special{pa 1150 3300}%
\special{fp}%
\special{pa 805 4080}%
\special{pa 1150 3390}%
\special{fp}%
\special{pa 850 4080}%
\special{pa 1150 3480}%
\special{fp}%
\special{pa 895 4080}%
\special{pa 1150 3570}%
\special{fp}%
\special{pa 940 4080}%
\special{pa 1150 3660}%
\special{fp}%
\special{pa 985 4080}%
\special{pa 1150 3750}%
\special{fp}%
\special{pa 1030 4080}%
\special{pa 1150 3840}%
\special{fp}%
\special{pa 1075 4080}%
\special{pa 1150 3930}%
\special{fp}%
\special{pa 1120 4080}%
\special{pa 1150 4020}%
\special{fp}%
% STR 2 0 3 0 Black White  
% 4 652 690 652 790 2 0 1 0
% \scriptsize{�K���X}
\put(6.5200,-7.9000){\makebox(0,0)[lb]{{\colorbox[named]{White}{\scriptsize{�K���X}}}}}%
% STR 2 0 3 1 Black White  
% 4 652 2690 652 2790 2 0 1 0
% \scriptsize{�K���X}
\put(6.5200,-27.9000){\makebox(0,0)[lb]{{\colorbox[named]{White}{\scriptsize{�K���X}}}}}%
% STR 2 0 3 2 Black White  
% 4 1852 690 1852 790 2 0 1 0
% \scriptsize{�K���X}
\put(18.5200,-7.9000){\makebox(0,0)[lb]{{\colorbox[named]{White}{\scriptsize{�K���X}}}}}%
% STR 2 0 3 3 Black White  
% 4 1852 2690 1852 2790 2 0 1 0
% \scriptsize{�K���X}
\put(18.5200,-27.9000){\makebox(0,0)[lb]{{\colorbox[named]{White}{\scriptsize{�K���X}}}}}%
% STR 2 0 3 4 Black White  
% 4 3052 690 3052 790 2 0 1 0
% \scriptsize{�K���X}
\put(30.5200,-7.9000){\makebox(0,0)[lb]{{\colorbox[named]{White}{\scriptsize{�K���X}}}}}%
% STR 2 0 3 5 Black White  
% 4 3052 2690 3052 2790 2 0 1 0
% \scriptsize{�K���X}
\put(30.5200,-27.9000){\makebox(0,0)[lb]{{\colorbox[named]{White}{\scriptsize{�K���X}}}}}%
% STR 2 0 3 6 Black White  
% 4 4252 690 4252 790 2 0 1 0
% \scriptsize{�K���X}
\put(42.5200,-7.9000){\makebox(0,0)[lb]{{\colorbox[named]{White}{\scriptsize{�K���X}}}}}%
% STR 2 0 3 7 Black White  
% 4 4252 2690 4252 2790 2 0 1 0
% \scriptsize{�K���X}
\put(42.5200,-27.9000){\makebox(0,0)[lb]{{\colorbox[named]{White}{\scriptsize{�K���X}}}}}%
% STR 2 0 3 0 Black White  
% 4 650 2140 650 2240 2 0 1 0
% \scriptsize{���A}
\put(6.5000,-22.4000){\makebox(0,0)[lb]{{\colorbox[named]{White}{\scriptsize{���A}}}}}%
% STR 2 0 3 0 Black White  
% 4 650 4140 650 4240 2 0 1 0
% \scriptsize{���B}
\put(6.5000,-42.4000){\makebox(0,0)[lb]{{\colorbox[named]{White}{\scriptsize{���B}}}}}%
% STR 2 0 3 0 Black White  
% 4 1850 2140 1850 2240 2 0 1 0
% \scriptsize{���A}
\put(18.5000,-22.4000){\makebox(0,0)[lb]{{\colorbox[named]{White}{\scriptsize{���A}}}}}%
% STR 2 0 3 0 Black White  
% 4 3050 2140 3050 2240 2 0 1 0
% \scriptsize{���A}
\put(30.5000,-22.4000){\makebox(0,0)[lb]{{\colorbox[named]{White}{\scriptsize{���A}}}}}%
% STR 2 0 3 0 Black White  
% 4 4250 2140 4250 2240 2 0 1 0
% \scriptsize{���A}
\put(42.5000,-22.4000){\makebox(0,0)[lb]{{\colorbox[named]{White}{\scriptsize{���A}}}}}%
% STR 2 0 3 0 Black White  
% 4 1850 4140 1850 4240 2 0 1 0
% \scriptsize{���B}
\put(18.5000,-42.4000){\makebox(0,0)[lb]{{\colorbox[named]{White}{\scriptsize{���B}}}}}%
% STR 2 0 3 0 Black White  
% 4 3050 4140 3050 4240 2 0 1 0
% \scriptsize{���B}
\put(30.5000,-42.4000){\makebox(0,0)[lb]{{\colorbox[named]{White}{\scriptsize{���B}}}}}%
% STR 2 0 3 0 Black White  
% 4 4250 4140 4250 4240 2 0 1 0
% \scriptsize{���B}
\put(42.5000,-42.4000){\makebox(0,0)[lb]{{\colorbox[named]{White}{\scriptsize{���B}}}}}%
\end{picture}}%

\end{center}
