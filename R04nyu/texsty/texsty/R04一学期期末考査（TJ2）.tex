%\documentclass[b5j,10pt]{jsarticle}
\documentclass[b5j,9.5pt]{jsbook}
\usepackage{okumacro}
\usepackage{amsmath,amsthm,amssymb}
\usepackage{enumerate,multicol,}
\usepackage{ascmac,itembbox,emath,hako,scrpage,ulinej,emathP,emathMw,emathEy}
%\usepackage[draft]{graphicx}
\usepackage{graphicx}
\usepackage{picins}
\pagestyle{empty}
\setlength{\textwidth}{162mm}
\setlength{\textheight}{230mm}
\setlength{\oddsidemargin}{-15.4mm}
\setlength{\evensidemargin}{-15.4mm}
\setlength{\topmargin}{-17.4mm}
%\setlength{\columnseprule}{0.4pt}
\renewcommand\labelenumi{\fbox{\bfseries{\sffamily{\theenumi}}}}
\renewcommand{\labelenumii}{(\arabic{enumii})}
\renewcommand{\labelenumiii}{\bfseries{\カタカナ{enumiii}.}}
\def\bunsuushisuu#1#2{\raisebox{0.7zh}{$#1 \over #2$}}
\def\santaku{{\bfseries ア}~{\bfseries ウ}}
\def\yontaku{{\bfseries ア}~{\bfseries エ}}
\def\gotaku{{\bfseries ア}~{\bfseries オ}}
\def\rokutaku{{\bfseries ア}~{\bfseries カ}}
\def\nanataku{{\bfseries ア}~{\bfseries キ}}
\def\sbou#1#2{$_{\sf{#1}}$\kern-0.2pt \ulinej{#2}}%
\def\tanni#1{$〔\mathrm{\sf #1}〕\kern -2pt$}%
\def\sftanni#1{$\kern 2pt{\mathrm{\sf #1}}$}
\begin{document}
\AtBeginDvi{\special{papersize=\the\paperwidth,\the\paperheight}}
%\setcounter{page}{1}
\newpagestyle{custom}{(0mm,0pt)%
{\hfill}
 {%
 \begin{minipage}{16.5cm}
 {\bf {\large 一学期期末考査 物理(科学技術科・情報科)}}\hfill T2A,J2A宮野尾,T2BC,J2B舟橋 R4.6.29(水)1限\\
 \end{minipage}
}%
{\hfill}%
(\textwidth,0pt)}%
{(0mm,0pt)%
{\hfill}{\hfill}{\hfill}%
(0mm,0pt)}
\pagestyle{custom}
\begin{enumerate}
\item 以下の問いに答えよ。
	\begin{enumerate}
		\item ~~
			\begin{mawarikomi}(10pt,0pt){120pt}{%WinTpicVersion4.32a
{\unitlength 0.1in%
\begin{picture}(16.0500,3.7500)(1.2500,-6.9500)%
% LINE 2 0 3 0 Black White  
% 2 400 600 1600 600
% 
\special{pn 8}%
\special{pa 400 600}%
\special{pa 1600 600}%
\special{fp}%
% CIRCLE 2 0 2 0 Black White  
% 4 400 600 450 600 450 600 450 600
% 
\special{sh 0}%
\special{ia 400 600 50 50 0.0000000 6.2831853}%
\special{pn 8}%
\special{ar 400 600 50 50 0.0000000 6.2831853}%
% CIRCLE 2 0 2 0 Black White  
% 4 1600 600 1650 600 1650 600 1650 600
% 
\special{sh 0}%
\special{ia 1600 600 50 50 0.0000000 6.2831853}%
\special{pn 8}%
\special{ar 1600 600 50 50 0.0000000 6.2831853}%
% LINE 3 0 3 0 Black White  
% 4 400 520 400 320 1600 520 1600 320
% 
\special{pn 4}%
\special{pa 400 520}%
\special{pa 400 320}%
\special{fp}%
\special{pa 1600 520}%
\special{pa 1600 320}%
\special{fp}%
% VECTOR 3 0 3 0 Black White  
% 4 1000 410 400 410 1000 410 1600 410
% 
\special{pn 4}%
\special{pa 1000 410}%
\special{pa 400 410}%
\special{fp}%
\special{sh 1}%
\special{pa 400 410}%
\special{pa 467 430}%
\special{pa 453 410}%
\special{pa 467 390}%
\special{pa 400 410}%
\special{fp}%
\special{pa 1000 410}%
\special{pa 1600 410}%
\special{fp}%
\special{sh 1}%
\special{pa 1600 410}%
\special{pa 1533 390}%
\special{pa 1547 410}%
\special{pa 1533 430}%
\special{pa 1600 410}%
\special{fp}%
% STR 2 0 3 0 Black White  
% 4 1000 310 1000 410 5 0 1 0
% 0.80{\sf m}
\put(10.0000,-4.1000){\makebox(0,0){{\colorbox[named]{White}{0.80{\sf m}}}}}%
% STR 2 0 3 0 Black White  
% 4 240 500 240 600 2 0 0 0
% A
\put(2.4000,-6.0000){\makebox(0,0)[lb]{A}}%
% STR 2 0 3 0 Black White  
% 4 1730 500 1730 600 2 0 0 0
% B
\put(17.3000,-6.0000){\makebox(0,0)[lb]{B}}%
% STR 2 0 3 0 Black White  
% 4 400 660 400 760 5 0 0 0
% 2.0{\sf kg}
\put(4.0000,-7.6000){\makebox(0,0){2.0{\sf kg}}}%
% STR 2 0 3 0 Black White  
% 4 1600 660 1600 760 5 0 0 0
% 3.0{\sf kg}
\put(16.0000,-7.6000){\makebox(0,0){3.0{\sf kg}}}%
\end{picture}}%
}
				右図のように長さ0.80{\sf m}の軽い棒の両端に,質量2.0{\sf kg}の小球Aと質量3.0{\sf kg}の小球Bを固定した。小球Aから重心までの距離は何{\sf m}か。ただし,小球の大きさは無視する。	\\
			\end{mawarikomi}
			\item ~~
			\begin{mawarikomi}(10pt,0pt){120pt}{%WinTpicVersion4.32a
{\unitlength 0.1in%
\begin{picture}(15.1500,5.5500)(1.2500,-6.9500)%
% LINE 2 0 3 0 Black White  
% 2 400 600 1600 600
% 
\special{pn 8}%
\special{pa 400 600}%
\special{pa 1600 600}%
\special{fp}%
% CIRCLE 2 0 2 0 Black White  
% 4 400 600 450 600 450 600 450 600
% 
\special{sh 0}%
\special{ia 400 600 50 50 0.0000000 6.2831853}%
\special{pn 8}%
\special{ar 400 600 50 50 0.0000000 6.2831853}%
% CIRCLE 2 0 2 0 Black White  
% 4 1200 600 1250 600 1250 600 1250 600
% 
\special{sh 0}%
\special{ia 1200 600 50 50 0.0000000 6.2831853}%
\special{pn 8}%
\special{ar 1200 600 50 50 0.0000000 6.2831853}%
% LINE 3 0 3 0 Black White  
% 2 400 140 400 540
% 
\special{pn 4}%
\special{pa 400 140}%
\special{pa 400 540}%
\special{fp}%
% LINE 3 0 3 0 Black White  
% 2 1600 140 1600 540
% 
\special{pn 4}%
\special{pa 1600 140}%
\special{pa 1600 540}%
\special{fp}%
% VECTOR 3 0 3 0 Black White  
% 4 1000 230 400 230 1000 230 1600 230
% 
\special{pn 4}%
\special{pa 1000 230}%
\special{pa 400 230}%
\special{fp}%
\special{sh 1}%
\special{pa 400 230}%
\special{pa 467 250}%
\special{pa 453 230}%
\special{pa 467 210}%
\special{pa 400 230}%
\special{fp}%
\special{pa 1000 230}%
\special{pa 1600 230}%
\special{fp}%
\special{sh 1}%
\special{pa 1600 230}%
\special{pa 1533 210}%
\special{pa 1547 230}%
\special{pa 1533 250}%
\special{pa 1600 230}%
\special{fp}%
% STR 2 0 3 0 Black White  
% 4 1000 130 1000 230 5 0 1 0
% 0.60{\sf m}
\put(10.0000,-2.3000){\makebox(0,0){{\colorbox[named]{White}{0.60{\sf m}}}}}%
% STR 2 0 3 0 Black White  
% 4 300 450 300 550 2 0 0 0
% A
\put(3.0000,-5.5000){\makebox(0,0)[lb]{A}}%
% STR 2 0 3 0 Black White  
% 4 1240 450 1240 550 2 0 0 0
% B
\put(12.4000,-5.5000){\makebox(0,0)[lb]{B}}%
% STR 2 0 3 0 Black White  
% 4 400 660 400 760 5 0 0 0
% 2.0{\sf kg}
\put(4.0000,-7.6000){\makebox(0,0){2.0{\sf kg}}}%
% STR 2 0 3 0 Black White  
% 4 1200 660 1200 760 5 0 0 0
% 3.0{\sf kg}
\put(12.0000,-7.6000){\makebox(0,0){3.0{\sf kg}}}%
% CIRCLE 2 0 2 0 Black White  
% 4 1600 600 1630 600 1630 600 1630 600
% 
\special{sh 0}%
\special{ia 1600 600 30 30 0.0000000 6.2831853}%
\special{pn 8}%
\special{ar 1600 600 30 30 0.0000000 6.2831853}%
% STR 2 0 3 0 Black White  
% 4 1600 660 1600 760 5 0 0 0
% 1.0{\sf kg}
\put(16.0000,-7.6000){\makebox(0,0){1.0{\sf kg}}}%
% VECTOR 3 0 3 0 Black White  
% 4 1007 420 407 420 1007 420 1207 420
% 
\special{pn 4}%
\special{pa 1007 420}%
\special{pa 407 420}%
\special{fp}%
\special{sh 1}%
\special{pa 407 420}%
\special{pa 474 440}%
\special{pa 460 420}%
\special{pa 474 400}%
\special{pa 407 420}%
\special{fp}%
\special{pa 1007 420}%
\special{pa 1207 420}%
\special{fp}%
\special{sh 1}%
\special{pa 1207 420}%
\special{pa 1140 400}%
\special{pa 1154 420}%
\special{pa 1140 440}%
\special{pa 1207 420}%
\special{fp}%
% STR 2 0 3 0 Black White  
% 4 810 320 810 420 5 0 1 0
% 0.40{\sf m}
\put(8.1000,-4.2000){\makebox(0,0){{\colorbox[named]{White}{0.40{\sf m}}}}}%
% STR 2 0 3 0 Black White  
% 4 1640 450 1640 550 2 0 0 0
% C
\put(16.4000,-5.5000){\makebox(0,0)[lb]{C}}%
% LINE 3 0 3 0 Black White  
% 2 1200 540 1200 340
% 
\special{pn 4}%
\special{pa 1200 540}%
\special{pa 1200 340}%
\special{fp}%
\end{picture}}%
}
				右図のように小球A,B,C(質量はそれぞれ,2.0{\sf kg},3.0{\sf kg},1.0{\sf kg})を軽い棒で接続する。AB間は0.40{\sf m},AC間は0.60{\sf m}である。小球Aから重心までの距離は何{\sf m}か。ただし,小球の大きさは無視する。\\	
			\end{mawarikomi}
		\item 質量3.0{\sf kg}の物体が,東向きに2.0{\sf m/s}の速さで進んでいる。この物体の運動量の大きさと向きを求めよ。
		\item 質量2.0{\sf kg},速度4.0{\sf m/s}の台車に速度と同じ向きに5.0{\sf N}の力を0.20{\sf s}間加えると,速度は$v'$\tanni{m/s}となった。$v'$を求めよ。
		\item 質量4.0{\sf kg},速さ2.0{\sf m/s}の台車Aが,静止している1.0{\sf kg}の台車Bに衝突して一体となった。衝突後のA,Bの速さ$v'$\tanni{m/s}は何か。
		\item ボールが速さ20m/sで壁に垂直に当たり,速さ12{\sf m/s}で跳ね返った。ボールと壁との間の反発係数$e$を求めよ。
		\item 水平な机の面より60cmの高さのところから,小球を自由落下させた。机の面と小球との間の反発係数を0.50とするとき,小球は衝突後何{\sf cm}の高さまで跳ね上がるか。
		\item 45\Deg は何{\sf rad}か。円周率$\pi $を用いて分数で答えよ。
		\item 半径4.0{\sf m}の円周上を等速円運動する物体が,2.0{\sf s}間に,90\Deg 回転した。物体の角速度$\omega $\tanni{rad/s},速さ$v$\tanni{m/s},周期$T$\tanni{s},回転数$n$\tanni{Hz},加速度の大きさ$a$\tanni{m/s^2}をそれぞれ求めよ。必要ならば,解答には円周率$\pi $を用いてよい。
	\end{enumerate}
	\vfill
\item 次の(1),(2)のような厚さ,密度が一様な板がある。(1)は左端を原点として,重心の$x$座標を,(2)は,左下端を原点として,重心の$x$座標,$y$座標をそれぞれ求めよ。
		\begin{edaenumerate}
			\item ~~\\
				%WinTpicVersion4.32a
{\unitlength 0.1in%
\begin{picture}(22.4000,17.6900)(3.3000,-22.2200)%
% POLYGON 2 0 3 0 Black White  
% 12 885 1108 1366 1108 1366 1108 1366 1589 885 1589 885 2070 2329 2070 2329 2070 2329 626 885 626 885 626 885 1108
% 
\special{pn 8}%
\special{pa 885 1108}%
\special{pa 1366 1108}%
\special{pa 1366 1589}%
\special{pa 885 1589}%
\special{pa 885 2070}%
\special{pa 2329 2070}%
\special{pa 2329 626}%
\special{pa 885 626}%
\special{pa 885 1108}%
\special{pa 1366 1108}%
\special{fp}%
% LINE 3 0 3 0 Black White  
% 98 885 699 921 626 885 807 976 626 885 915 1030 626 885 1023 1084 626 897 1108 1138 626 951 1108 1192 626 1006 1108 1246 626 1060 1108 1300 626 1114 1108 1354 626 885 1673 927 1589 1168 1108 1409 626 885 1781 982 1589 1222 1108 1463 626 885 1890 1036 1589 1276 1108 1517 626 885 1998 1090 1589 1330 1108 1571 626 903 2070 1144 1589 1366 1144 1625 626 957 2070 1198 1589 1366 1252 1679 626 1012 2070 1252 1589 1366 1360 1733 626 1066 2070 1306 1589 1366 1469 1788 626 1120 2070 1360 1589 1366 1577 1842 626 1174 2070 1896 626 1228 2070 1950 626 1282 2070 2004 626 1336 2070 2058 626 1391 2070 2112 626 1445 2070 2167 626 1499 2070 2221 626 1553 2070 2275 626 1607 2070 2329 626 1661 2070 2329 735 1715 2070 2329 843 1770 2070 2329 951 1824 2070 2329 1060 1878 2070 2329 1168 1932 2070 2329 1276 1986 2070 2329 1384 2040 2070 2329 1493 2094 2070 2329 1601 2148 2070 2329 1709 2203 2070 2329 1817 2257 2070 2329 1926 2311 2070 2329 2034
% 
\special{pn 4}%
\special{pa 885 699}%
\special{pa 921 626}%
\special{fp}%
\special{pa 885 807}%
\special{pa 976 626}%
\special{fp}%
\special{pa 885 915}%
\special{pa 1030 626}%
\special{fp}%
\special{pa 885 1023}%
\special{pa 1084 626}%
\special{fp}%
\special{pa 897 1108}%
\special{pa 1138 626}%
\special{fp}%
\special{pa 951 1108}%
\special{pa 1192 626}%
\special{fp}%
\special{pa 1006 1108}%
\special{pa 1246 626}%
\special{fp}%
\special{pa 1060 1108}%
\special{pa 1300 626}%
\special{fp}%
\special{pa 1114 1108}%
\special{pa 1354 626}%
\special{fp}%
\special{pa 885 1673}%
\special{pa 927 1589}%
\special{fp}%
\special{pa 1168 1108}%
\special{pa 1409 626}%
\special{fp}%
\special{pa 885 1781}%
\special{pa 982 1589}%
\special{fp}%
\special{pa 1222 1108}%
\special{pa 1463 626}%
\special{fp}%
\special{pa 885 1890}%
\special{pa 1036 1589}%
\special{fp}%
\special{pa 1276 1108}%
\special{pa 1517 626}%
\special{fp}%
\special{pa 885 1998}%
\special{pa 1090 1589}%
\special{fp}%
\special{pa 1330 1108}%
\special{pa 1571 626}%
\special{fp}%
\special{pa 903 2070}%
\special{pa 1144 1589}%
\special{fp}%
\special{pa 1366 1144}%
\special{pa 1625 626}%
\special{fp}%
\special{pa 957 2070}%
\special{pa 1198 1589}%
\special{fp}%
\special{pa 1366 1252}%
\special{pa 1679 626}%
\special{fp}%
\special{pa 1012 2070}%
\special{pa 1252 1589}%
\special{fp}%
\special{pa 1366 1360}%
\special{pa 1733 626}%
\special{fp}%
\special{pa 1066 2070}%
\special{pa 1306 1589}%
\special{fp}%
\special{pa 1366 1469}%
\special{pa 1788 626}%
\special{fp}%
\special{pa 1120 2070}%
\special{pa 1360 1589}%
\special{fp}%
\special{pa 1366 1577}%
\special{pa 1842 626}%
\special{fp}%
\special{pa 1174 2070}%
\special{pa 1896 626}%
\special{fp}%
\special{pa 1228 2070}%
\special{pa 1950 626}%
\special{fp}%
\special{pa 1282 2070}%
\special{pa 2004 626}%
\special{fp}%
\special{pa 1336 2070}%
\special{pa 2058 626}%
\special{fp}%
\special{pa 1391 2070}%
\special{pa 2112 626}%
\special{fp}%
\special{pa 1445 2070}%
\special{pa 2167 626}%
\special{fp}%
\special{pa 1499 2070}%
\special{pa 2221 626}%
\special{fp}%
\special{pa 1553 2070}%
\special{pa 2275 626}%
\special{fp}%
\special{pa 1607 2070}%
\special{pa 2329 626}%
\special{fp}%
\special{pa 1661 2070}%
\special{pa 2329 735}%
\special{fp}%
\special{pa 1715 2070}%
\special{pa 2329 843}%
\special{fp}%
\special{pa 1770 2070}%
\special{pa 2329 951}%
\special{fp}%
\special{pa 1824 2070}%
\special{pa 2329 1060}%
\special{fp}%
\special{pa 1878 2070}%
\special{pa 2329 1168}%
\special{fp}%
\special{pa 1932 2070}%
\special{pa 2329 1276}%
\special{fp}%
\special{pa 1986 2070}%
\special{pa 2329 1384}%
\special{fp}%
\special{pa 2040 2070}%
\special{pa 2329 1493}%
\special{fp}%
\special{pa 2094 2070}%
\special{pa 2329 1601}%
\special{fp}%
\special{pa 2148 2070}%
\special{pa 2329 1709}%
\special{fp}%
\special{pa 2203 2070}%
\special{pa 2329 1817}%
\special{fp}%
\special{pa 2257 2070}%
\special{pa 2329 1926}%
\special{fp}%
\special{pa 2311 2070}%
\special{pa 2329 2034}%
\special{fp}%
% LINE 2 1 3 0 Black White  
% 10 1366 626 1366 1108 1366 1589 1366 2070 1366 1108 2329 1108 1848 626 1848 2070 2329 1589 1366 1589
% 
\special{pn 8}%
\special{pa 1366 626}%
\special{pa 1366 1108}%
\special{da 0.035}%
\special{pa 1366 1589}%
\special{pa 1366 2070}%
\special{da 0.035}%
\special{pa 1366 1108}%
\special{pa 2329 1108}%
\special{da 0.035}%
\special{pa 1848 626}%
\special{pa 1848 2070}%
\special{da 0.035}%
\special{pa 2329 1589}%
\special{pa 1366 1589}%
\special{da 0.035}%
% VECTOR 2 0 3 0 Black White  
% 2 885 1348 2570 1348
% 
\special{pn 8}%
\special{pa 885 1348}%
\special{pa 2570 1348}%
\special{fp}%
\special{sh 1}%
\special{pa 2570 1348}%
\special{pa 2503 1328}%
\special{pa 2517 1348}%
\special{pa 2503 1368}%
\special{pa 2570 1348}%
\special{fp}%
% STR 2 0 3 0 Black White  
% 4 2642 1228 2642 1348 5 0 0 0
% $x$
\put(26.4200,-13.4800){\makebox(0,0){$x$}}%
% DOT 0 0 3 0 Black White  
% 1 885 1348
% 
\special{pn 4}%
\special{sh 1}%
\special{ar 885 1348 16 16 0 6.2831853}%
% STR 2 0 3 0 Black White  
% 4 850 1220 850 1340 4 0 0 0
% O
\put(8.5000,-13.4000){\makebox(0,0)[rt]{O}}%
% VECTOR 3 0 3 0 Black White  
% 4 1607 518 885 518 1607 518 2329 518
% 
\special{pn 4}%
\special{pa 1607 518}%
\special{pa 885 518}%
\special{fp}%
\special{sh 1}%
\special{pa 885 518}%
\special{pa 952 538}%
\special{pa 938 518}%
\special{pa 952 498}%
\special{pa 885 518}%
\special{fp}%
\special{pa 1607 518}%
\special{pa 2329 518}%
\special{fp}%
\special{sh 1}%
\special{pa 2329 518}%
\special{pa 2262 498}%
\special{pa 2276 518}%
\special{pa 2262 538}%
\special{pa 2329 518}%
\special{fp}%
% STR 2 0 3 0 Black White  
% 4 1607 398 1607 518 5 0 1 0
% 2.4{\sf m}
\put(16.0700,-5.1800){\makebox(0,0){{\colorbox[named]{White}{2.4{\sf m}}}}}%
% VECTOR 3 0 3 0 Black White  
% 2 2401 1348 2401 626
% 
\special{pn 4}%
\special{pa 2401 1348}%
\special{pa 2401 626}%
\special{fp}%
\special{sh 1}%
\special{pa 2401 626}%
\special{pa 2381 693}%
\special{pa 2401 679}%
\special{pa 2421 693}%
\special{pa 2401 626}%
\special{fp}%
% VECTOR 3 0 3 0 Black White  
% 2 2401 1348 2401 2070
% 
\special{pn 4}%
\special{pa 2401 1348}%
\special{pa 2401 2070}%
\special{fp}%
\special{sh 1}%
\special{pa 2401 2070}%
\special{pa 2421 2003}%
\special{pa 2401 2017}%
\special{pa 2381 2003}%
\special{pa 2401 2070}%
\special{fp}%
% STR 2 0 3 0 Black White  
% 4 2377 939 2377 1060 2 0 1 0
% 2.4{\sf m}
\put(23.7700,-10.6000){\makebox(0,0)[lb]{{\colorbox[named]{White}{2.4{\sf m}}}}}%
% VECTOR 3 0 3 0 Black White  
% 4 789 867 789 626 789 867 789 1108
% 
\special{pn 4}%
\special{pa 789 867}%
\special{pa 789 626}%
\special{fp}%
\special{sh 1}%
\special{pa 789 626}%
\special{pa 769 693}%
\special{pa 789 679}%
\special{pa 809 693}%
\special{pa 789 626}%
\special{fp}%
\special{pa 789 867}%
\special{pa 789 1108}%
\special{fp}%
\special{sh 1}%
\special{pa 789 1108}%
\special{pa 809 1041}%
\special{pa 789 1055}%
\special{pa 769 1041}%
\special{pa 789 1108}%
\special{fp}%
% VECTOR 3 0 3 0 Black White  
% 4 789 1830 789 1589 789 1830 789 2070
% 
\special{pn 4}%
\special{pa 789 1830}%
\special{pa 789 1589}%
\special{fp}%
\special{sh 1}%
\special{pa 789 1589}%
\special{pa 769 1656}%
\special{pa 789 1642}%
\special{pa 809 1656}%
\special{pa 789 1589}%
\special{fp}%
\special{pa 789 1830}%
\special{pa 789 2070}%
\special{fp}%
\special{sh 1}%
\special{pa 789 2070}%
\special{pa 809 2003}%
\special{pa 789 2017}%
\special{pa 769 2003}%
\special{pa 789 2070}%
\special{fp}%
% STR 2 0 3 0 Black White  
% 4 620 741 620 861 5 0 1 0
% 0.80{\sf m}
\put(6.2000,-8.6100){\makebox(0,0){{\colorbox[named]{White}{0.80{\sf m}}}}}%
% VECTOR 2 0 3 0 Black White  
% 4 1114 2166 873 2166 1114 2166 1354 2166
% 
\special{pn 8}%
\special{pa 1114 2166}%
\special{pa 873 2166}%
\special{fp}%
\special{sh 1}%
\special{pa 873 2166}%
\special{pa 940 2186}%
\special{pa 926 2166}%
\special{pa 940 2146}%
\special{pa 873 2166}%
\special{fp}%
\special{pa 1114 2166}%
\special{pa 1354 2166}%
\special{fp}%
\special{sh 1}%
\special{pa 1354 2166}%
\special{pa 1287 2146}%
\special{pa 1301 2166}%
\special{pa 1287 2186}%
\special{pa 1354 2166}%
\special{fp}%
% STR 2 0 3 0 Black White  
% 4 1114 2166 1114 2287 5 0 1 0
% 0.80{\sf m}
\put(11.1400,-22.8700){\makebox(0,0){{\colorbox[named]{White}{0.80{\sf m}}}}}%
% STR 2 0 3 0 Black White  
% 4 622 1703 622 1824 5 0 1 0
% 0.80{\sf m}
\put(6.2200,-18.2400){\makebox(0,0){{\colorbox[named]{White}{0.80{\sf m}}}}}%
% CIRCLE 3 0 3 0 Black White  
% 4 1487 1234 1366 1102 1366 1102 1366 1342
% 
\special{pn 4}%
\special{ar 1487 1234 179 179 2.4129022 3.9704417}%
% LINE 2 0 3 0 Black White  
% 2 1282 1234 1342 1234
% 
\special{pn 8}%
\special{pa 1282 1234}%
\special{pa 1342 1234}%
\special{fp}%
% CIRCLE 3 0 3 0 Black White  
% 4 1487 1475 1366 1342 1366 1342 1366 1583
% 
\special{pn 4}%
\special{ar 1487 1475 180 180 2.4129022 3.9741998}%
% LINE 2 0 3 0 Black White  
% 2 1282 1475 1342 1475
% 
\special{pn 8}%
\special{pa 1282 1475}%
\special{pa 1342 1475}%
\special{fp}%
% STR 2 0 3 0 Black White  
% 4 1078 1102 1078 1222 5 0 0 0
% 0.40{\sf m}
\put(10.7800,-12.2200){\makebox(0,0){0.40{\sf m}}}%
% STR 2 0 3 0 Black White  
% 4 1078 1354 1078 1475 5 0 0 0
% 0.40{\sf m}
\put(10.7800,-14.7500){\makebox(0,0){0.40{\sf m}}}%
\end{picture}}%

			\item ~~\\
				%WinTpicVersion4.32a
{\unitlength 0.1in%
\begin{picture}(22.8000,15.9000)(-0.7300,-22.3000)%
% POLYGON 2 0 3 0 Black White  
% 7 407 1005 407 2205 2007 2205 2007 1405 1207 1405 1207 1005 407 1005
% 
\special{pn 8}%
\special{pa 407 1005}%
\special{pa 407 2205}%
\special{pa 2007 2205}%
\special{pa 2007 1405}%
\special{pa 1207 1405}%
\special{pa 1207 1005}%
\special{pa 407 1005}%
\special{pa 407 2205}%
\special{fp}%
% LINE 3 0 3 0 Black White  
% 98 407 1076 442 1005 407 1166 488 1005 407 1256 532 1005 407 1346 578 1005 407 1436 622 1005 407 1526 668 1005 407 1616 712 1005 407 1706 758 1005 407 1796 802 1005 407 1886 848 1005 407 1976 892 1005 407 2066 938 1005 407 2156 982 1005 427 2205 1028 1005 472 2205 1072 1005 517 2205 1118 1005 562 2205 1162 1005 607 2205 1207 1006 652 2205 1207 1096 698 2205 1207 1186 742 2205 1207 1276 788 2205 1207 1366 832 2205 1232 1405 877 2205 1278 1405 922 2205 1322 1405 967 2205 1368 1405 1012 2205 1412 1405 1057 2205 1458 1405 1102 2205 1502 1405 1147 2205 1548 1405 1192 2205 1592 1405 1237 2205 1638 1405 1282 2205 1682 1405 1328 2205 1728 1405 1372 2205 1772 1405 1418 2205 1818 1405 1462 2205 1862 1405 1508 2205 1908 1405 1552 2205 1952 1405 1598 2205 1998 1405 1642 2205 2007 1476 1687 2205 2007 1566 1732 2205 2007 1656 1777 2205 2007 1746 1822 2205 2007 1836 1867 2205 2007 1926 1912 2205 2007 2016 1957 2205 2007 2106 2002 2205 2007 2196
% 
\special{pn 4}%
\special{pa 407 1076}%
\special{pa 442 1005}%
\special{fp}%
\special{pa 407 1166}%
\special{pa 488 1005}%
\special{fp}%
\special{pa 407 1256}%
\special{pa 532 1005}%
\special{fp}%
\special{pa 407 1346}%
\special{pa 578 1005}%
\special{fp}%
\special{pa 407 1436}%
\special{pa 622 1005}%
\special{fp}%
\special{pa 407 1526}%
\special{pa 668 1005}%
\special{fp}%
\special{pa 407 1616}%
\special{pa 712 1005}%
\special{fp}%
\special{pa 407 1706}%
\special{pa 758 1005}%
\special{fp}%
\special{pa 407 1796}%
\special{pa 802 1005}%
\special{fp}%
\special{pa 407 1886}%
\special{pa 848 1005}%
\special{fp}%
\special{pa 407 1976}%
\special{pa 892 1005}%
\special{fp}%
\special{pa 407 2066}%
\special{pa 938 1005}%
\special{fp}%
\special{pa 407 2156}%
\special{pa 982 1005}%
\special{fp}%
\special{pa 427 2205}%
\special{pa 1028 1005}%
\special{fp}%
\special{pa 472 2205}%
\special{pa 1072 1005}%
\special{fp}%
\special{pa 517 2205}%
\special{pa 1118 1005}%
\special{fp}%
\special{pa 562 2205}%
\special{pa 1162 1005}%
\special{fp}%
\special{pa 607 2205}%
\special{pa 1207 1006}%
\special{fp}%
\special{pa 652 2205}%
\special{pa 1207 1096}%
\special{fp}%
\special{pa 698 2205}%
\special{pa 1207 1186}%
\special{fp}%
\special{pa 742 2205}%
\special{pa 1207 1276}%
\special{fp}%
\special{pa 788 2205}%
\special{pa 1207 1366}%
\special{fp}%
\special{pa 832 2205}%
\special{pa 1232 1405}%
\special{fp}%
\special{pa 877 2205}%
\special{pa 1278 1405}%
\special{fp}%
\special{pa 922 2205}%
\special{pa 1322 1405}%
\special{fp}%
\special{pa 967 2205}%
\special{pa 1368 1405}%
\special{fp}%
\special{pa 1012 2205}%
\special{pa 1412 1405}%
\special{fp}%
\special{pa 1057 2205}%
\special{pa 1458 1405}%
\special{fp}%
\special{pa 1102 2205}%
\special{pa 1502 1405}%
\special{fp}%
\special{pa 1147 2205}%
\special{pa 1548 1405}%
\special{fp}%
\special{pa 1192 2205}%
\special{pa 1592 1405}%
\special{fp}%
\special{pa 1237 2205}%
\special{pa 1638 1405}%
\special{fp}%
\special{pa 1282 2205}%
\special{pa 1682 1405}%
\special{fp}%
\special{pa 1328 2205}%
\special{pa 1728 1405}%
\special{fp}%
\special{pa 1372 2205}%
\special{pa 1772 1405}%
\special{fp}%
\special{pa 1418 2205}%
\special{pa 1818 1405}%
\special{fp}%
\special{pa 1462 2205}%
\special{pa 1862 1405}%
\special{fp}%
\special{pa 1508 2205}%
\special{pa 1908 1405}%
\special{fp}%
\special{pa 1552 2205}%
\special{pa 1952 1405}%
\special{fp}%
\special{pa 1598 2205}%
\special{pa 1998 1405}%
\special{fp}%
\special{pa 1642 2205}%
\special{pa 2007 1476}%
\special{fp}%
\special{pa 1687 2205}%
\special{pa 2007 1566}%
\special{fp}%
\special{pa 1732 2205}%
\special{pa 2007 1656}%
\special{fp}%
\special{pa 1777 2205}%
\special{pa 2007 1746}%
\special{fp}%
\special{pa 1822 2205}%
\special{pa 2007 1836}%
\special{fp}%
\special{pa 1867 2205}%
\special{pa 2007 1926}%
\special{fp}%
\special{pa 1912 2205}%
\special{pa 2007 2016}%
\special{fp}%
\special{pa 1957 2205}%
\special{pa 2007 2106}%
\special{fp}%
\special{pa 2002 2205}%
\special{pa 2007 2196}%
\special{fp}%
% VECTOR 3 0 3 0 Black White  
% 4 407 2205 407 805 407 2205 2207 2205
% 
\special{pn 4}%
\special{pa 407 2205}%
\special{pa 407 805}%
\special{fp}%
\special{sh 1}%
\special{pa 407 805}%
\special{pa 387 872}%
\special{pa 407 858}%
\special{pa 427 872}%
\special{pa 407 805}%
\special{fp}%
\special{pa 407 2205}%
\special{pa 2207 2205}%
\special{fp}%
\special{sh 1}%
\special{pa 2207 2205}%
\special{pa 2140 2185}%
\special{pa 2154 2205}%
\special{pa 2140 2225}%
\special{pa 2207 2205}%
\special{fp}%
% STR 2 0 3 0 Black White  
% 4 2450 2100 2450 2200 5 0 0 0
% $x$\!\!\!〔{\sf m}〕
\put(24.5000,-22.0000){\makebox(0,0){$x$\!\!\!〔{\sf m}〕}}%
% STR 2 0 3 0 Black White  
% 4 407 605 407 705 5 0 0 0
% $y$\!\!\!〔{\sf m}〕
\put(4.0700,-7.0500){\makebox(0,0){$y$\!\!\!〔{\sf m}〕}}%
% LINE 2 1 3 0 Black White  
% 10 1207 1405 407 1405 807 1005 807 2205 407 1805 2007 1805 1207 1405 1207 2205 1607 2205 1607 1405
% 
\special{pn 8}%
\special{pa 1207 1405}%
\special{pa 407 1405}%
\special{da 0.035}%
\special{pa 807 1005}%
\special{pa 807 2205}%
\special{da 0.035}%
\special{pa 407 1805}%
\special{pa 2007 1805}%
\special{da 0.035}%
\special{pa 1207 1405}%
\special{pa 1207 2205}%
\special{da 0.035}%
\special{pa 1607 2205}%
\special{pa 1607 1405}%
\special{da 0.035}%
% STR 2 0 3 0 Black White  
% 4 407 2105 407 2205 4 0 0 0
% O
\put(4.0700,-22.0500){\makebox(0,0)[rt]{O}}%
% STR 2 0 3 0 Black White  
% 4 2007 2195 2007 2295 5 0 0 0
% 4.0
\put(20.0700,-22.9500){\makebox(0,0){4.0}}%
% STR 2 0 3 0 Black White  
% 4 1200 2190 1200 2290 5 0 0 0
% 2.0
\put(12.0000,-22.9000){\makebox(0,0){2.0}}%
% STR 2 0 3 0 Black White  
% 4 1600 2190 1600 2290 5 0 0 0
% 3.0
\put(16.0000,-22.9000){\makebox(0,0){3.0}}%
% STR 2 0 3 0 Black White  
% 4 800 2190 800 2290 5 0 0 0
% 1.0
\put(8.0000,-22.9000){\makebox(0,0){1.0}}%
% STR 2 0 3 0 Black White  
% 4 270 1700 270 1800 5 0 0 0
% 1.0
\put(2.7000,-18.0000){\makebox(0,0){1.0}}%
% STR 2 0 3 0 Black White  
% 4 270 1300 270 1400 5 0 0 0
% 2.0
\put(2.7000,-14.0000){\makebox(0,0){2.0}}%
% STR 2 0 3 0 Black White  
% 4 270 900 270 1000 5 0 0 0
% 3.0
\put(2.7000,-10.0000){\makebox(0,0){3.0}}%
\end{picture}}%

		\end{edaenumerate}
\vfill
\newpage
\item ~~
		\begin{mawarikomi}{160pt}{%WinTpicVersion4.32a
{\unitlength 0.1in%
\begin{picture}(20.0000,10.9800)(10.0000,-20.0000)%
% POLYGON 2 0 3 0 Black White  
% 5 1766 1770 1594 1195 2360 966 2533 1540 1766 1770
% 
\special{pn 8}%
\special{pa 1766 1770}%
\special{pa 1594 1195}%
\special{pa 2360 966}%
\special{pa 2533 1540}%
\special{pa 1766 1770}%
\special{pa 1594 1195}%
\special{fp}%
% LINE 2 0 3 0 Black White  
% 2 1000 2000 2916 1425
% 
\special{pn 8}%
\special{pa 1000 2000}%
\special{pa 2916 1425}%
\special{fp}%
% CIRCLE 3 0 3 0 Black White  
% 4 2140 1658 2351 968 2351 968 1584 1198
% 
\special{pn 4}%
\special{ar 2140 1658 722 722 3.8327823 5.0091557}%
% CIRCLE 3 0 3 0 Black White  
% 4 2054 1371 2351 968 2523 1543 2351 968
% 
\special{pn 4}%
\special{ar 2054 1371 501 501 5.3475004 0.3515074}%
% LINE 2 0 3 0 Black White  
% 2 1000 2000 3000 2000
% 
\special{pn 8}%
\special{pa 1000 2000}%
\special{pa 3000 2000}%
\special{fp}%
% CIRCLE 2 0 3 0 Black White  
% 4 1000 2000 1500 2000 3200 2000 3000 1400
% 
\special{pn 8}%
\special{ar 1000 2000 500 500 5.9917285 6.2831853}%
% STR 2 0 3 0 Black White  
% 4 1550 1860 1550 1960 2 0 0 0
% $\theta $
\put(15.5000,-19.6000){\makebox(0,0)[lb]{$\theta $}}%
% STR 2 0 3 0 Black White  
% 4 1933 867 1933 967 5 0 1 0
% $b$
\put(19.3300,-9.6700){\makebox(0,0){{\colorbox[named]{White}{$b$}}}}%
% STR 2 0 3 0 Black White  
% 4 2533 1087 2533 1187 5 0 1 0
% $a$
\put(25.3300,-11.8700){\makebox(0,0){{\colorbox[named]{White}{$a$}}}}%
% DOT 0 0 3 0 Black White  
% 1 1760 1770
% 
\special{pn 4}%
\special{sh 1}%
\special{ar 1760 1770 16 16 0 6.2831853}%
% STR 2 0 3 0 Black White  
% 4 1700 1620 1700 1720 5 0 0 0
% A
\put(17.0000,-17.2000){\makebox(0,0){A}}%
% STR 2 0 3 0 Black White  
% 4 2200 1100 2200 1200 2 0 0 0
% $m$
\put(22.0000,-12.0000){\makebox(0,0)[lb]{$m$}}%
\end{picture}}%
}
		右図のように,縦$a$\tanni{m},横$b$\tanni{m}で質量$m$\tanni{kg}の一様な物質でできている直方体の物体を,水平面と$\theta $の角をなすあらい斜面上に静止させる。物体と斜面との間の静止摩擦係数を$\mu $,重力加速度の大きさを$g$\tanni{m/s^2}とする。
			\begin{enumerate}
			\item 図の点Aから物体にはたらく垂直抗力の作用線までの距離$x$を求めよ。
			\item 斜面の角$\theta $をしだいに大きくしていくと,物体はすべることなく傾き始めた。このときの$\theta$を$\theta _0$とするとき,$\tan{\theta _0}$はいくらか。
			\item (2)が起きるために,静止摩擦係数$\mu $は$\mu_0$より大きくなければならない。$\mu_0 $を,$a$,$b$を用いて表せ。
			\end{enumerate}
		\end{mawarikomi}
\vfill
%\item ~~
%		\begin{mawarikomi}{160pt}{%WinTpicVersion4.32a
{\unitlength 0.1in%
\begin{picture}(20.5000,8.3500)(3.5000,-13.9500)%
% BOX 2 0 3 0 Black White  
% 2 800 595 1400 795
% 
\special{pn 8}%
\special{pa 800 595}%
\special{pa 1400 595}%
\special{pa 1400 795}%
\special{pa 800 795}%
\special{pa 800 595}%
\special{pa 1400 595}%
\special{fp}%
% STR 2 0 3 1 Black White  
% 4 1100 595 1100 695 5 0 0 0
% B
\put(11.0000,-6.9500){\makebox(0,0){B}}%
% LINE 2 0 3 0 Black White  
% 6 1400 595 1600 595 1600 795 1400 795 1400 795 1400 595
% 
\special{pn 8}%
\special{pa 1400 595}%
\special{pa 1600 595}%
\special{fp}%
\special{pa 1600 795}%
\special{pa 1400 795}%
\special{fp}%
\special{pa 1400 795}%
\special{pa 1400 595}%
\special{fp}%
% CIRCLE 2 0 3 1 Black White  
% 4 1600 695 1600 595 1600 995 1600 595
% 
\special{pn 8}%
\special{ar 1600 695 100 100 4.7123890 1.5707963}%
% STR 2 0 3 2 Black White  
% 4 1540 595 1540 695 5 0 0 0
% A
\put(15.4000,-6.9500){\makebox(0,0){A}}%
% BOX 2 0 3 0 Black White  
% 2 1400 1195 2000 1395
% 
\special{pn 8}%
\special{pa 1400 1195}%
\special{pa 2000 1195}%
\special{pa 2000 1395}%
\special{pa 1400 1395}%
\special{pa 1400 1195}%
\special{pa 2000 1195}%
\special{fp}%
% STR 2 0 3 1 Black White  
% 4 1700 1195 1700 1295 5 0 0 0
% B
\put(17.0000,-12.9500){\makebox(0,0){B}}%
% LINE 2 0 3 0 Black White  
% 6 2200 1195 2400 1195 2400 1395 2200 1395 2200 1395 2200 1195
% 
\special{pn 8}%
\special{pa 2200 1195}%
\special{pa 2400 1195}%
\special{fp}%
\special{pa 2400 1395}%
\special{pa 2200 1395}%
\special{fp}%
\special{pa 2200 1395}%
\special{pa 2200 1195}%
\special{fp}%
% CIRCLE 2 0 3 1 Black White  
% 4 2400 1295 2400 1195 2400 1595 2400 1195
% 
\special{pn 8}%
\special{ar 2400 1295 100 100 4.7123890 1.5707963}%
% STR 2 0 3 2 Black White  
% 4 2340 1195 2340 1295 5 0 0 0
% A
\put(23.4000,-12.9500){\makebox(0,0){A}}%
% VECTOR 2 0 3 0 Black White  
% 2 1800 695 2200 695
% 
\special{pn 8}%
\special{pa 1800 695}%
\special{pa 2200 695}%
\special{fp}%
\special{sh 1}%
\special{pa 2200 695}%
\special{pa 2133 675}%
\special{pa 2147 695}%
\special{pa 2133 715}%
\special{pa 2200 695}%
\special{fp}%
% STR 2 0 3 0 Black White  
% 4 2000 525 2000 625 5 0 0 0
% $v$
\put(20.0000,-6.2500){\makebox(0,0){$v$}}%
% STR 2 0 3 0 Black White  
% 4 350 660 350 760 2 0 0 0
% 分離前
\put(3.5000,-7.6000){\makebox(0,0)[lb]{分離前}}%
% STR 2 0 3 0 Black White  
% 4 350 1260 350 1360 2 0 0 0
% 分離後
\put(3.5000,-13.6000){\makebox(0,0)[lb]{分離後}}%
\end{picture}}%
}
%			質量$m$\tanni{kg}の頭部Aと質量$M$\tanni{kg}の尾部Bからなるロケットが,速度$v$\tanni{m/s}で進んでいるとき,頭部Aを尾部Bに対する相対的な速さ$u$\tanni{m/s}で一瞬のうちに分離した。分離前のロケットの進行方向を正の向きとして以下の問いに答えよ。
%			\begin{enumerate}
%				\item 分離後の頭部Aの地面に対する速度を$v_\mathrm{A}$,尾部Bの地面に対する速度を$v_\mathrm{B}$として,$v_\mathrm{A}$,$v_\mathrm{B}$,$u$の関係を示せ。
%				\item 頭部Aの地面に対する速度$v_\mathrm{A}$\tanni{m/s}を求めよ。
%			\end{enumerate}
%		\end{mawarikomi}
%\vfill
\item ~~
		\begin{mawarikomi}(20pt,0pt){160pt}{%WinTpicVersion4.32a
{\unitlength 0.1in%
\begin{picture}(19.0500,15.3500)(4.9500,-25.5500)%
% LINE 2 1 3 0 Black Black  
% 2 1567 2001 2149 2547
% 
\special{pn 8}%
\special{pa 1567 2001}%
\special{pa 2149 2547}%
\special{da 0.035}%
% VECTOR 2 0 3 0 Black White  
% 2 600 2000 2400 2000
% 
\special{pn 8}%
\special{pa 600 2000}%
\special{pa 2400 2000}%
\special{fp}%
\special{sh 1}%
\special{pa 2400 2000}%
\special{pa 2333 1980}%
\special{pa 2347 2000}%
\special{pa 2333 2020}%
\special{pa 2400 2000}%
\special{fp}%
% VECTOR 2 0 3 0 Black White  
% 2 600 2200 1000 2200
% 
\special{pn 8}%
\special{pa 600 2200}%
\special{pa 1000 2200}%
\special{fp}%
\special{sh 1}%
\special{pa 1000 2200}%
\special{pa 933 2180}%
\special{pa 947 2200}%
\special{pa 933 2220}%
\special{pa 1000 2200}%
\special{fp}%
% STR 2 0 3 0 Black White  
% 4 800 2220 800 2320 5 0 0 0
% 3.0{\sf m/s}
\put(8.0000,-23.2000){\makebox(0,0){3.0{\sf m/s}}}%
% STR 2 0 3 0 Black White  
% 4 800 1730 800 1830 5 0 0 0
% 0.50{\sf kg}
\put(8.0000,-18.3000){\makebox(0,0){0.50{\sf kg}}}%
% CIRCLE 2 0 1 0 Black Black  
% 4 1640 2050 1720 2050 1720 2050 1720 2050
% 
\special{sh 0.300}%
\special{ia 1640 2050 80 80 0.0000000 6.2831853}%
\special{pn 8}%
\special{ar 1640 2050 80 80 0.0000000 6.2831853}%
% CIRCLE 2 0 2 0 Black White  
% 4 800 2000 880 2000 880 2000 880 2000
% 
\special{sh 0}%
\special{ia 800 2000 80 80 0.0000000 6.2831853}%
\special{pn 8}%
\special{ar 800 2000 80 80 0.0000000 6.2831853}%
% LINE 2 1 3 0 Black White  
% 2 1560 2000 2080 1100
% 
\special{pn 8}%
\special{pa 1560 2000}%
\special{pa 2080 1100}%
\special{da 0.035}%
% CIRCLE 2 1 2 0 Black White  
% 4 1910 1400 1990 1400 1990 1400 1990 1400
% 
\special{sh 0}%
\special{ia 1910 1400 80 80 0.0000000 6.2831853}%
\special{pn 8}%
\special{pn 8}%
\special{pa 1990 1400}%
\special{pa 1990 1409}%
\special{pa 1989 1409}%
\special{pa 1989 1415}%
\special{pa 1988 1416}%
\special{pa 1987 1423}%
\special{pa 1986 1424}%
\special{pa 1985 1429}%
\special{pa 1984 1429}%
\special{fp}%
\special{pa 1971 1451}%
\special{pa 1971 1452}%
\special{pa 1967 1456}%
\special{pa 1967 1457}%
\special{pa 1966 1457}%
\special{pa 1962 1461}%
\special{pa 1961 1461}%
\special{pa 1961 1462}%
\special{pa 1960 1462}%
\special{pa 1960 1463}%
\special{pa 1957 1464}%
\special{pa 1957 1465}%
\special{pa 1956 1465}%
\special{pa 1956 1466}%
\special{pa 1953 1467}%
\special{pa 1953 1468}%
\special{pa 1951 1468}%
\special{pa 1951 1469}%
\special{pa 1951 1469}%
\special{fp}%
\special{pa 1924 1479}%
\special{pa 1919 1479}%
\special{pa 1919 1480}%
\special{pa 1901 1480}%
\special{pa 1901 1479}%
\special{pa 1895 1479}%
\special{pa 1894 1478}%
\special{fp}%
\special{pa 1868 1468}%
\special{pa 1867 1468}%
\special{pa 1867 1467}%
\special{pa 1864 1466}%
\special{pa 1864 1465}%
\special{pa 1863 1465}%
\special{pa 1863 1464}%
\special{pa 1860 1463}%
\special{pa 1860 1462}%
\special{pa 1859 1462}%
\special{pa 1859 1461}%
\special{pa 1858 1461}%
\special{pa 1854 1457}%
\special{pa 1853 1457}%
\special{pa 1853 1456}%
\special{pa 1849 1452}%
\special{pa 1849 1451}%
\special{pa 1848 1451}%
\special{pa 1848 1450}%
\special{fp}%
\special{pa 1835 1427}%
\special{pa 1834 1424}%
\special{pa 1833 1423}%
\special{pa 1833 1420}%
\special{pa 1832 1420}%
\special{pa 1832 1416}%
\special{pa 1831 1415}%
\special{pa 1831 1409}%
\special{pa 1830 1409}%
\special{pa 1830 1398}%
\special{fp}%
\special{pa 1836 1369}%
\special{pa 1837 1366}%
\special{pa 1838 1366}%
\special{pa 1838 1364}%
\special{pa 1839 1364}%
\special{pa 1839 1362}%
\special{pa 1840 1362}%
\special{pa 1840 1360}%
\special{pa 1841 1360}%
\special{pa 1841 1359}%
\special{pa 1842 1359}%
\special{pa 1842 1357}%
\special{pa 1843 1357}%
\special{pa 1844 1354}%
\special{pa 1845 1354}%
\special{pa 1845 1353}%
\special{pa 1846 1353}%
\special{pa 1847 1350}%
\special{pa 1848 1350}%
\special{pa 1848 1349}%
\special{pa 1849 1349}%
\special{pa 1849 1348}%
\special{pa 1849 1348}%
\special{fp}%
\special{pa 1871 1330}%
\special{pa 1872 1330}%
\special{pa 1873 1329}%
\special{pa 1874 1329}%
\special{pa 1874 1328}%
\special{pa 1876 1328}%
\special{pa 1876 1327}%
\special{pa 1881 1326}%
\special{pa 1881 1325}%
\special{pa 1886 1324}%
\special{pa 1887 1323}%
\special{pa 1890 1323}%
\special{pa 1890 1322}%
\special{pa 1895 1322}%
\special{pa 1895 1321}%
\special{pa 1896 1321}%
\special{fp}%
\special{pa 1926 1322}%
\special{pa 1933 1323}%
\special{pa 1933 1324}%
\special{pa 1936 1324}%
\special{pa 1937 1325}%
\special{pa 1939 1325}%
\special{pa 1939 1326}%
\special{pa 1944 1327}%
\special{pa 1944 1328}%
\special{pa 1946 1328}%
\special{pa 1946 1329}%
\special{pa 1948 1329}%
\special{pa 1948 1330}%
\special{pa 1950 1330}%
\special{pa 1950 1331}%
\special{pa 1951 1331}%
\special{pa 1951 1331}%
\special{fp}%
\special{pa 1972 1349}%
\special{pa 1972 1349}%
\special{pa 1972 1350}%
\special{pa 1973 1350}%
\special{pa 1974 1353}%
\special{pa 1975 1353}%
\special{pa 1975 1354}%
\special{pa 1976 1354}%
\special{pa 1977 1357}%
\special{pa 1978 1357}%
\special{pa 1978 1359}%
\special{pa 1979 1359}%
\special{pa 1980 1362}%
\special{pa 1981 1363}%
\special{pa 1981 1364}%
\special{pa 1982 1364}%
\special{pa 1982 1366}%
\special{pa 1983 1366}%
\special{pa 1984 1371}%
\special{pa 1985 1371}%
\special{pa 1985 1371}%
\special{fp}%
% VECTOR 2 0 3 0 Black White  
% 2 1710 1400 1930 1020
% 
\special{pn 8}%
\special{pa 1710 1400}%
\special{pa 1930 1020}%
\special{fp}%
\special{sh 1}%
\special{pa 1930 1020}%
\special{pa 1879 1068}%
\special{pa 1903 1066}%
\special{pa 1914 1088}%
\special{pa 1930 1020}%
\special{fp}%
% STR 2 0 3 0 Black White  
% 4 1400 1120 1400 1220 2 0 0 0
% 3.0{\sf m/s}
\put(14.0000,-12.2000){\makebox(0,0)[lb]{3.0{\sf m/s}}}%
% CIRCLE 2 0 3 0 Black White  
% 4 1560 2000 1800 2000 2400 2000 2120 1030
% 
\special{pn 8}%
\special{ar 1560 2000 240 240 5.2359647 6.2831853}%
% STR 2 0 3 0 Black White  
% 4 1840 1820 1840 1920 2 0 0 0
% 60\Deg
\put(18.4000,-19.2000){\makebox(0,0)[lb]{60\Deg}}%
% STR 2 0 3 0 Black White  
% 4 650 1990 650 2090 5 0 0 0
% A
\put(6.5000,-20.9000){\makebox(0,0){A}}%
% STR 2 0 3 0 Black White  
% 4 2050 1300 2050 1400 2 0 0 0
% A
\put(20.5000,-14.0000){\makebox(0,0)[lb]{A}}%
% STR 2 0 3 0 Black White  
% 4 1450 1980 1450 2080 1 0 0 0
% B
\put(14.5000,-20.8000){\makebox(0,0)[lt]{B}}%
% STR 2 0 3 0 Black White  
% 4 2490 1900 2490 2000 5 0 0 0
% $x$
\put(24.9000,-20.0000){\makebox(0,0){$x$}}%
% STR 2 0 3 0 Black White  
% 4 1640 2180 1640 2280 5 0 0 0
% 1.5{\sf kg}
\put(16.4000,-22.8000){\makebox(0,0){1.5{\sf kg}}}%
% CIRCLE 2 1 1 0 Black Black  
% 4 2063 2475 2141 2493 2141 2493 2141 2493
% 
\special{sh 0.300}%
\special{ia 2063 2475 80 80 0.0000000 6.2831853}%
\special{pn 8}%
\special{pn 8}%
\special{pa 2143 2475}%
\special{pa 2143 2484}%
\special{pa 2142 2484}%
\special{pa 2142 2490}%
\special{pa 2141 2491}%
\special{pa 2141 2495}%
\special{pa 2140 2495}%
\special{pa 2140 2498}%
\special{pa 2139 2499}%
\special{pa 2138 2504}%
\special{pa 2137 2504}%
\special{fp}%
\special{pa 2123 2527}%
\special{pa 2123 2527}%
\special{pa 2123 2528}%
\special{pa 2121 2530}%
\special{pa 2121 2531}%
\special{pa 2120 2531}%
\special{pa 2119 2532}%
\special{pa 2119 2533}%
\special{pa 2118 2533}%
\special{pa 2118 2534}%
\special{pa 2115 2535}%
\special{pa 2115 2536}%
\special{pa 2114 2536}%
\special{pa 2114 2537}%
\special{pa 2113 2537}%
\special{pa 2113 2538}%
\special{pa 2112 2538}%
\special{pa 2112 2539}%
\special{pa 2110 2539}%
\special{pa 2110 2540}%
\special{pa 2109 2540}%
\special{pa 2109 2541}%
\special{pa 2106 2542}%
\special{pa 2106 2543}%
\special{pa 2104 2543}%
\special{pa 2104 2544}%
\special{fp}%
\special{pa 2078 2554}%
\special{pa 2078 2554}%
\special{pa 2072 2554}%
\special{pa 2072 2555}%
\special{pa 2054 2555}%
\special{pa 2054 2554}%
\special{pa 2048 2554}%
\special{pa 2048 2553}%
\special{fp}%
\special{pa 2023 2544}%
\special{pa 2020 2543}%
\special{pa 2020 2542}%
\special{pa 2017 2541}%
\special{pa 2017 2540}%
\special{pa 2016 2540}%
\special{pa 2016 2539}%
\special{pa 2014 2539}%
\special{pa 2014 2538}%
\special{pa 2013 2538}%
\special{pa 2013 2537}%
\special{pa 2012 2537}%
\special{pa 2012 2536}%
\special{pa 2011 2536}%
\special{pa 2010 2535}%
\special{pa 2010 2534}%
\special{pa 2007 2533}%
\special{pa 2007 2532}%
\special{pa 2006 2532}%
\special{pa 2006 2531}%
\special{pa 2005 2530}%
\special{pa 2004 2530}%
\special{pa 2003 2527}%
\special{pa 2002 2527}%
\special{fp}%
\special{pa 1989 2504}%
\special{pa 1988 2504}%
\special{pa 1988 2501}%
\special{pa 1987 2501}%
\special{pa 1987 2498}%
\special{pa 1986 2498}%
\special{pa 1986 2495}%
\special{pa 1985 2495}%
\special{pa 1985 2491}%
\special{pa 1984 2490}%
\special{pa 1984 2484}%
\special{pa 1983 2484}%
\special{pa 1983 2476}%
\special{fp}%
\special{pa 1988 2447}%
\special{pa 1988 2446}%
\special{pa 1989 2446}%
\special{pa 1989 2443}%
\special{pa 1990 2443}%
\special{pa 1990 2441}%
\special{pa 1991 2441}%
\special{pa 1991 2439}%
\special{pa 1992 2439}%
\special{pa 1992 2437}%
\special{pa 1993 2437}%
\special{pa 1994 2434}%
\special{pa 1995 2434}%
\special{pa 1995 2432}%
\special{pa 1996 2432}%
\special{pa 1997 2429}%
\special{pa 1998 2429}%
\special{pa 1999 2426}%
\special{pa 2000 2426}%
\special{pa 2000 2425}%
\special{pa 2001 2425}%
\special{pa 2001 2425}%
\special{fp}%
\special{pa 2022 2406}%
\special{pa 2023 2406}%
\special{pa 2024 2405}%
\special{pa 2025 2405}%
\special{pa 2025 2404}%
\special{pa 2027 2404}%
\special{pa 2027 2403}%
\special{pa 2029 2403}%
\special{pa 2029 2402}%
\special{pa 2031 2402}%
\special{pa 2031 2401}%
\special{pa 2034 2401}%
\special{pa 2034 2400}%
\special{pa 2039 2399}%
\special{pa 2040 2398}%
\special{pa 2043 2398}%
\special{pa 2043 2397}%
\special{pa 2047 2397}%
\special{pa 2048 2396}%
\special{fp}%
\special{pa 2079 2397}%
\special{pa 2079 2397}%
\special{pa 2083 2397}%
\special{pa 2083 2398}%
\special{pa 2086 2398}%
\special{pa 2086 2399}%
\special{pa 2089 2399}%
\special{pa 2090 2400}%
\special{pa 2092 2400}%
\special{pa 2092 2401}%
\special{pa 2097 2402}%
\special{pa 2097 2403}%
\special{pa 2099 2403}%
\special{pa 2099 2404}%
\special{pa 2101 2404}%
\special{pa 2101 2405}%
\special{pa 2104 2406}%
\special{pa 2104 2406}%
\special{fp}%
\special{pa 2124 2423}%
\special{pa 2124 2424}%
\special{pa 2125 2424}%
\special{pa 2125 2425}%
\special{pa 2126 2425}%
\special{pa 2127 2428}%
\special{pa 2128 2428}%
\special{pa 2128 2429}%
\special{pa 2129 2429}%
\special{pa 2130 2432}%
\special{pa 2131 2432}%
\special{pa 2132 2435}%
\special{pa 2133 2435}%
\special{pa 2133 2437}%
\special{pa 2134 2437}%
\special{pa 2134 2439}%
\special{pa 2135 2439}%
\special{pa 2135 2441}%
\special{pa 2136 2441}%
\special{pa 2137 2446}%
\special{pa 2137 2446}%
\special{fp}%
\end{picture}}%
}
		右図のように,$x$軸上を正の方向に3.0{\sf m/s}で進む質量0.50{\sf kg}の小球Aが,$x$軸上で静止している質量1.5{\sf kg}の小球Bに衝突した。小球Aは$x$軸となす角が60\Deg の方向へ3.0{\sf m/s}で進んだ。以下の問いに答えよ。ただし,向きは$x$軸となす角で表し,反時計回りを正とし,$-180\Deg $から$180\Deg $の間の値で答えよ。
			\begin{enumerate}
				\item 小球Aが小球Bから受けた力積の大きさと向きを答えよ。
				\item 小球Bが小球Aから受けた力積の大きさと向きを答えよ。
				\item 衝突後の小球Bの速さを求めよ。
			\end{enumerate}
		\end{mawarikomi}
\vfill
\item ~~
		\begin{mawarikomi}(10pt,0pt){180pt}{%WinTpicVersion4.32a
{\unitlength 0.1in%
\begin{picture}(24.0000,11.8000)(4.0000,-18.0000)%
% LINE 2 0 3 0 Black White  
% 2 400 1800 2800 1800
% 
\special{pn 8}%
\special{pa 400 1800}%
\special{pa 2800 1800}%
\special{fp}%
% LINE 2 1 3 0 Black White  
% 2 1600 1800 920 620
% 
\special{pn 8}%
\special{pa 1600 1800}%
\special{pa 920 620}%
\special{da 0.035}%
% LINE 2 1 3 0 Black White  
% 2 1600 1800 2600 800
% 
\special{pn 8}%
\special{pa 1600 1800}%
\special{pa 2600 800}%
\special{da 0.035}%
% CIRCLE 2 0 2 0 Black White  
% 4 1050 850 1100 850 1100 850 1100 850
% 
\special{sh 0}%
\special{ia 1050 850 50 50 0.0000000 6.2831853}%
\special{pn 8}%
\special{ar 1050 850 50 50 0.0000000 6.2831853}%
% CIRCLE 2 0 2 0 Black White  
% 4 1600 1750 1650 1750 1650 1750 1650 1750
% 
\special{sh 0}%
\special{ia 1600 1750 50 50 0.0000000 6.2831853}%
\special{pn 8}%
\special{ar 1600 1750 50 50 0.0000000 6.2831853}%
% CIRCLE 2 0 2 0 Black White  
% 4 2250 1150 2300 1150 2300 1150 2300 1150
% 
\special{sh 0}%
\special{ia 2250 1150 50 50 0.0000000 6.2831853}%
\special{pn 8}%
\special{ar 2250 1150 50 50 0.0000000 6.2831853}%
% VECTOR 1 0 3 0 Black White  
% 2 1050 850 1380 1420
% 
\special{pn 13}%
\special{pa 1050 850}%
\special{pa 1380 1420}%
\special{fp}%
\special{sh 1}%
\special{pa 1380 1420}%
\special{pa 1364 1352}%
\special{pa 1353 1374}%
\special{pa 1329 1372}%
\special{pa 1380 1420}%
\special{fp}%
% VECTOR 1 0 3 0 Black White  
% 2 2250 1150 2580 820
% 
\special{pn 13}%
\special{pa 2250 1150}%
\special{pa 2580 820}%
\special{fp}%
\special{sh 1}%
\special{pa 2580 820}%
\special{pa 2519 853}%
\special{pa 2542 858}%
\special{pa 2547 881}%
\special{pa 2580 820}%
\special{fp}%
% CIRCLE 2 0 3 0 Black White  
% 4 1600 1800 1400 1800 770 360 770 1800
% 
\special{pn 8}%
\special{ar 1600 1800 200 200 3.1415927 4.1895115}%
% CIRCLE 2 0 3 0 Black White  
% 4 1600 1800 1800 1800 2600 1800 2600 800
% 
\special{pn 8}%
\special{ar 1600 1800 200 200 5.4977871 6.2831853}%
% STR 2 0 3 0 Black White  
% 4 1230 1620 1230 1720 2 0 0 0
% 60\Deg
\put(12.3000,-17.2000){\makebox(0,0)[lb]{60\Deg}}%
% STR 2 0 3 0 Black White  
% 4 1830 1670 1830 1770 2 0 0 0
% 45\Deg
\put(18.3000,-17.7000){\makebox(0,0)[lb]{45\Deg}}%
% STR 2 0 3 0 Black White  
% 4 1320 1070 1320 1170 2 0 0 0
% $v$
\put(13.2000,-11.7000){\makebox(0,0)[lb]{$v$}}%
% STR 2 0 3 0 Black White  
% 4 2460 1020 2460 1120 2 0 0 0
% $v'$
\put(24.6000,-11.2000){\makebox(0,0)[lb]{$v'$}}%
% STR 2 0 3 0 Black White  
% 4 1140 700 1140 800 2 0 0 0
% $m$
\put(11.4000,-8.0000){\makebox(0,0)[lb]{$m$}}%
\end{picture}}%
}
		水平でなめらかな床面に対して,60\Deg をなす角で質量$m$の小球が速さ$v$で衝突したところ,床面に対して45\Deg をなす角で跳ね返った。以下の問いに答えよ。ただし,$\sqrt{ } $はそのままでよい。
			\begin{enumerate}
				\item 小球と床面との反発係数$e$を求めよ。
				\item 跳ね返った後の小球の速さ$v'$を$v$を用いて表せ。
				\item 小球が床面から受けた力積の大きさを$m$,$v$を用いて表せ。
				\item 小球の床面との衝突における力学的エネルギーの変化$\varDelta E$を$m$,$v$を用いて表せ。
			\end{enumerate}
		\end{mawarikomi}
\vfill
\newpage
\item ~~
		\begin{mawarikomi}(20pt,0pt){160pt}{%WinTpicVersion4.32a
{\unitlength 0.1in%
\begin{picture}(18.7500,4.5200)(4.9500,-13.9700)%
% CIRCLE 2 0 3 0 Black White  
% 4 800 1200 880 1200 880 1200 880 1200
% 
\special{pn 8}%
\special{ar 800 1200 80 80 0.0000000 6.2831853}%
% CIRCLE 2 0 3 0 Black White  
% 4 2200 1200 2310 1200 2310 1200 2310 1200
% 
\special{pn 8}%
\special{ar 2200 1200 110 110 0.0000000 6.2831853}%
% VECTOR 2 0 3 0 Black White  
% 2 600 1330 1000 1330
% 
\special{pn 8}%
\special{pa 600 1330}%
\special{pa 1000 1330}%
\special{fp}%
\special{sh 1}%
\special{pa 1000 1330}%
\special{pa 933 1310}%
\special{pa 947 1330}%
\special{pa 933 1350}%
\special{pa 1000 1330}%
\special{fp}%
% VECTOR 2 0 3 0 Black White  
% 2 2200 1370 2000 1370
% 
\special{pn 8}%
\special{pa 2200 1370}%
\special{pa 2000 1370}%
\special{fp}%
\special{sh 1}%
\special{pa 2000 1370}%
\special{pa 2067 1390}%
\special{pa 2053 1370}%
\special{pa 2067 1350}%
\special{pa 2000 1370}%
\special{fp}%
% STR 2 0 3 0 Black White  
% 4 800 930 800 1030 5 0 0 0
% 1.0{\sf kg}
\put(8.0000,-10.3000){\makebox(0,0){1.0{\sf kg}}}%
% STR 2 0 3 0 Black White  
% 4 2200 910 2200 1010 5 0 0 0
% 3.0{\sf kg}
\put(22.0000,-10.1000){\makebox(0,0){3.0{\sf kg}}}%
% STR 2 0 3 0 Black White  
% 4 800 1360 800 1460 5 0 0 0
% 8.0{\sf m/s}
\put(8.0000,-14.6000){\makebox(0,0){8.0{\sf m/s}}}%
% STR 2 0 3 0 Black White  
% 4 2200 1360 2200 1460 5 0 0 0
% 2.0{\sf m/s}
\put(22.0000,-14.6000){\makebox(0,0){2.0{\sf m/s}}}%
% STR 2 0 3 0 Black White  
% 4 600 1100 600 1200 5 0 0 0
% A
\put(6.0000,-12.0000){\makebox(0,0){A}}%
% STR 2 0 3 0 Black White  
% 4 2400 1100 2400 1200 5 0 0 0
% B
\put(24.0000,-12.0000){\makebox(0,0){B}}%
\end{picture}}%
}
		一直線上を,質量1.0{\sf kg},8.0{\sf m/s}の速さで右向きに進む小球Aと,
		同一直線上を,質量3.0{\sf kg},2.0{\sf m/s}の速さで左向きに進む小球Bとが衝突する。
		小球Aと小球Bとの間の反発係数が以下の(1),(2)場合,衝突後の小球Aの向きと,速さ$v_\mathrm{A}$\tanni{m/s}を答えよ。
			\begin{enumerate}
				\item $e=0$(完全非弾性衝突)のとき。
				\item $e=0.40$のとき。
			\end{enumerate}
		\end{mawarikomi}
\vfill
\item ~~
		\begin{mawarikomi}(10pt,0pt){200pt}{%WinTpicVersion4.32a
{\unitlength 0.1in%
\begin{picture}(22.0000,29.0500)(4.8000,-38.7000)%
% CIRCLE 2 0 3 0 Black White  
% 4 800 2000 900 2066 900 2066 699 2066
% 
\special{pn 8}%
\special{ar 800 2000 120 120 2.5627852 0.5833730}%
% LINE 2 0 3 1 Black White  
% 2 900 2066 699 2066
% 
\special{pn 8}%
\special{pa 900 2066}%
\special{pa 699 2066}%
\special{fp}%
% CIRCLE 2 0 3 0 Black White  
% 4 800 2000 900 2066 699 2066 900 2066
% 
\special{pn 8}%
\special{ar 800 2000 120 120 0.5833730 2.5627852}%
% LINE 2 0 3 1 Black White  
% 2 900 2066 699 2066
% 
\special{pn 8}%
\special{pa 900 2066}%
\special{pa 699 2066}%
\special{fp}%
% CIRCLE 2 0 3 0 Black White  
% 4 2270 1630 2370 1696 2370 1696 2169 1696
% 
\special{pn 8}%
\special{ar 2270 1630 120 120 2.5627852 0.5833730}%
% LINE 2 0 3 1 Black White  
% 2 2370 1696 2169 1696
% 
\special{pn 8}%
\special{pa 2370 1696}%
\special{pa 2169 1696}%
\special{fp}%
% CIRCLE 2 0 3 0 Black White  
% 4 2260 3100 2360 3166 2159 3166 2360 3166
% 
\special{pn 8}%
\special{ar 2260 3100 120 120 0.5833730 2.5627852}%
% LINE 2 0 3 1 Black White  
% 2 2360 3166 2159 3166
% 
\special{pn 8}%
\special{pa 2360 3166}%
\special{pa 2159 3166}%
\special{fp}%
% VECTOR 2 0 3 0 Black White  
% 2 1600 2000 2600 2000
% 
\special{pn 8}%
\special{pa 1600 2000}%
\special{pa 2600 2000}%
\special{fp}%
\special{sh 1}%
\special{pa 2600 2000}%
\special{pa 2533 1980}%
\special{pa 2547 2000}%
\special{pa 2533 2020}%
\special{pa 2600 2000}%
\special{fp}%
% LINE 2 1 3 0 Black White  
% 2 1600 2000 2310 1590
% 
\special{pn 8}%
\special{pa 1600 2000}%
\special{pa 2310 1590}%
\special{da 0.035}%
% CIRCLE 2 0 3 0 Black White  
% 4 1600 2000 1900 2000 2500 2000 2640 1400
% 
\special{pn 8}%
\special{ar 1600 2000 300 300 5.7599070 6.2831853}%
% STR 2 0 3 0 Black White  
% 4 1920 1870 1920 1970 2 0 0 0
% 30\Deg
\put(19.2000,-19.7000){\makebox(0,0)[lb]{30\Deg}}%
% LINE 2 1 3 0 Black White  
% 2 1600 2000 2270 3160
% 
\special{pn 8}%
\special{pa 1600 2000}%
\special{pa 2270 3160}%
\special{da 0.035}%
% VECTOR 1 0 3 0 Black White  
% 2 800 2000 1200 2000
% 
\special{pn 13}%
\special{pa 800 2000}%
\special{pa 1200 2000}%
\special{fp}%
\special{sh 1}%
\special{pa 1200 2000}%
\special{pa 1133 1980}%
\special{pa 1147 2000}%
\special{pa 1133 2020}%
\special{pa 1200 2000}%
\special{fp}%
% VECTOR 2 0 3 0 Black White  
% 2 1600 2000 1600 1000
% 
\special{pn 8}%
\special{pa 1600 2000}%
\special{pa 1600 1000}%
\special{fp}%
\special{sh 1}%
\special{pa 1600 1000}%
\special{pa 1580 1067}%
\special{pa 1600 1053}%
\special{pa 1620 1067}%
\special{pa 1600 1000}%
\special{fp}%
% LINE 2 0 3 0 Black White  
% 2 1600 2000 1600 2800
% 
\special{pn 8}%
\special{pa 1600 2000}%
\special{pa 1600 2800}%
\special{fp}%
% LINE 2 0 3 0 Black White  
% 2 1600 2000 1000 2000
% 
\special{pn 8}%
\special{pa 1600 2000}%
\special{pa 1000 2000}%
\special{fp}%
% STR 2 0 3 0 Black White  
% 4 1570 1940 1570 2040 4 0 0 0
% O
\put(15.7000,-20.4000){\makebox(0,0)[rt]{O}}%
% STR 2 0 3 0 Black White  
% 4 2540 1980 2540 2080 5 0 0 0
% $x$
\put(25.4000,-20.8000){\makebox(0,0){$x$}}%
% STR 2 0 3 0 Black White  
% 4 1690 930 1690 1030 5 0 0 0
% $y$
\put(16.9000,-10.3000){\makebox(0,0){$y$}}%
% CIRCLE 2 0 3 0 Black White  
% 4 1600 2000 1840 2000 2300 3210 2300 2000
% 
\special{pn 8}%
\special{ar 1600 2000 240 240 6.2831853 1.0463264}%
% STR 2 0 3 0 Black White  
% 4 1850 2020 1850 2120 1 0 0 0
% 60\Deg
\put(18.5000,-21.2000){\makebox(0,0)[lt]{60\Deg}}%
% STR 2 0 3 0 Black White  
% 4 800 2110 800 2210 5 0 0 0
% 4.0{\sf kg}
\put(8.0000,-22.1000){\makebox(0,0){4.0{\sf kg}}}%
% STR 2 0 3 0 Black White  
% 4 2360 1280 2360 1380 5 0 0 0
% 3.0{\sf kg}
\put(23.6000,-13.8000){\makebox(0,0){3.0{\sf kg}}}%
% STR 2 0 3 0 Black White  
% 4 2430 2920 2430 3020 5 0 0 0
% 1.0{\sf kg}
\put(24.3000,-30.2000){\makebox(0,0){1.0{\sf kg}}}%
% VECTOR 1 0 3 0 Black White  
% 2 2280 1610 2680 1380
% 
\special{pn 13}%
\special{pa 2280 1610}%
\special{pa 2680 1380}%
\special{fp}%
\special{sh 1}%
\special{pa 2680 1380}%
\special{pa 2612 1396}%
\special{pa 2634 1407}%
\special{pa 2632 1431}%
\special{pa 2680 1380}%
\special{fp}%
% VECTOR 1 0 3 0 Black White  
% 2 2270 3180 2670 3870
% 
\special{pn 13}%
\special{pa 2270 3180}%
\special{pa 2670 3870}%
\special{fp}%
\special{sh 1}%
\special{pa 2670 3870}%
\special{pa 2654 3802}%
\special{pa 2643 3824}%
\special{pa 2619 3822}%
\special{pa 2670 3870}%
\special{fp}%
% STR 2 0 3 0 Black White  
% 4 480 1740 480 1840 2 0 0 0
% 小物体
\put(4.8000,-18.4000){\makebox(0,0)[lb]{小物体}}%
% STR 2 0 3 0 Black White  
% 4 2510 1570 2510 1670 2 0 0 0
% A
\put(25.1000,-16.7000){\makebox(0,0)[lb]{A}}%
% STR 2 0 3 0 Black White  
% 4 2390 3170 2390 3270 2 0 0 0
% B
\put(23.9000,-32.7000){\makebox(0,0)[lb]{B}}%
% STR 2 0 3 0 Black White  
% 4 940 1820 940 1920 2 0 0 0
% 2.0{\sf m/s}
\put(9.4000,-19.2000){\makebox(0,0)[lb]{2.0{\sf m/s}}}%
% STR 2 0 3 0 Black White  
% 4 2620 1190 2620 1290 2 0 0 0
% $v_1$\!\!\!〔{\sf m/s}〕
\put(26.2000,-12.9000){\makebox(0,0)[lb]{$v_1$\!\!\!〔{\sf m/s}〕}}%
% STR 2 0 3 0 Black White  
% 4 2540 3510 2540 3610 2 0 0 0
% $v_2$\!\!\!〔{\sf m/s}〕
\put(25.4000,-36.1000){\makebox(0,0)[lb]{$v_2$\!\!\!〔{\sf m/s}〕}}%
\end{picture}}%
}
		右図のように,なめらかな水平面の$x$軸上を,質量4.0{\sf kg}の小物体が速さ$2.0${\sf m/s}で$x$軸の正の向きに進んできて,原点において3.0{\sf kg}の物体Aと1.0{\sf kg}の物体Bに瞬間的に分裂した。物体Aは$x$軸となす角30\Deg の向きに速さ$v_1$\tanni{m/s},物体Bは$x$軸となす角60\Deg の向きに速さ$v_2$\tanni{m/s}に進んだ。以下の問いに答えよ。
			\begin{enumerate}
				\item $x$方向,$y$方向について,運動量保存則の式を立てよ。ただし,sin,cosはそのままでよい。
				\item $v_2$\tanni{m/s}を求めよ。
				\item 物体Aが物体Bから受けた力積の大きさと向きを答えよ。ただし,向きは$x$軸となす角で表し,反時計回りを正とし,$-180\Deg $から$180\Deg $の間の値で答えよ。$\sqrt{ }$はそのままでよい。
			\end{enumerate}
		\end{mawarikomi}
\vfill
\item ~~
		\begin{mawarikomi}(10pt,0pt){160pt}{%WinTpicVersion4.32a
{\unitlength 0.1in%
\begin{picture}(20.0000,14.0000)(2.6000,-28.1000)%
% LINE 2 0 3 0 Black White  
% 10 680 1610 480 1610 480 1610 480 2810 480 2810 2000 2810 2000 2810 2000 2610 2000 2610 1660 2610
% 
\special{pn 8}%
\special{pa 680 1610}%
\special{pa 480 1610}%
\special{fp}%
\special{pa 480 1610}%
\special{pa 480 2810}%
\special{fp}%
\special{pa 480 2810}%
\special{pa 2000 2810}%
\special{fp}%
\special{pa 2000 2810}%
\special{pa 2000 2610}%
\special{fp}%
\special{pa 2000 2610}%
\special{pa 1660 2610}%
\special{fp}%
% CIRCLE 2 0 3 1 Black White  
% 4 1680 1610 1680 2610 680 1610 1680 2610
% 
\special{pn 8}%
\special{ar 1680 1610 1000 1000 1.5707963 3.1415927}%
% LINE 2 0 3 0 Black White  
% 2 2260 2810 260 2810
% 
\special{pn 8}%
\special{pa 2260 2810}%
\special{pa 260 2810}%
\special{fp}%
% CIRCLE 2 0 3 0 Black White  
% 4 730 1610 780 1610 780 1610 780 1610
% 
\special{pn 8}%
\special{ar 730 1610 50 50 0.0000000 6.2831853}%
% LINE 2 1 3 0 Black White  
% 2 780 1610 1860 1610
% 
\special{pn 8}%
\special{pa 780 1610}%
\special{pa 1860 1610}%
\special{da 0.035}%
% VECTOR 3 0 3 0 Black White  
% 2 1800 2010 1800 1610
% 
\special{pn 4}%
\special{pa 1800 2010}%
\special{pa 1800 1610}%
\special{fp}%
\special{sh 1}%
\special{pa 1800 1610}%
\special{pa 1780 1677}%
\special{pa 1800 1663}%
\special{pa 1820 1677}%
\special{pa 1800 1610}%
\special{fp}%
% VECTOR 3 0 3 0 Black White  
% 2 1800 2210 1800 2610
% 
\special{pn 4}%
\special{pa 1800 2210}%
\special{pa 1800 2610}%
\special{fp}%
\special{sh 1}%
\special{pa 1800 2610}%
\special{pa 1820 2543}%
\special{pa 1800 2557}%
\special{pa 1780 2543}%
\special{pa 1800 2610}%
\special{fp}%
% STR 2 0 3 0 Black White  
% 4 1800 2000 1800 2100 5 0 0 0
% $h$
\put(18.0000,-21.0000){\makebox(0,0){$h$}}%
% STR 2 0 3 0 Black White  
% 4 800 1440 800 1540 2 0 0 0
% A
\put(8.0000,-15.4000){\makebox(0,0)[lb]{A}}%
% STR 2 0 3 0 Black White  
% 4 2050 2500 2050 2600 2 0 0 0
% B
\put(20.5000,-26.0000){\makebox(0,0)[lb]{B}}%
\end{picture}}%
}
		右図のように,曲面と水平面からなる質量$M$の台Bが,なめらかな床の上に置かれている。ここで,台Bの水平面から高さ$h$の曲面上から,質量$m$の小球Aを静かにすべらせた。この後,小球Aが台Bの水平面上を動いているときの,床に対する小球Aと台の速さをそれぞれ$v$,$V$とする。小球Aが台Bとの間に摩擦はなく,運動の前後で力学的エネルギーが保存されているとしてよい。また,重力加速度の大きさを$g$とする。以下の問いに答えよ。
			\begin{enumerate}
				\item 小球Aと台Bについて,運動量保存則の式を立てよ。
				\item 小球Aと台Bについて,力学的エネルギー保存則の式を立てよ。
				\item $v$を,$m$,$M$,$g$,$h$を用いて表せ。
			\end{enumerate}
		\end{mawarikomi}
\end{enumerate}
\vfill
\newpage
〔計算用紙〕
\vfill
\end{document}
