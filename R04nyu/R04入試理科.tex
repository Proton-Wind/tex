%\documentclass[b5j,10pt]{jsarticle}
\documentclass[b5j,10pt,dvipdfmx]{jsbook}
\usepackage{amsmath,amsthm,amssymb}
\usepackage{enumerate,multicol,tabularx}
\usepackage{ascmac,itembbox,emath,emathMw,emathEy,hako,scrpage,ulinej}
%\usepackage[draft]{graphicx}
%\usepackage[dvipdfmx]{graphicx}
%\usepackage[dvipdfmx]{color}
\usepackage[dvipdfmx]{graphicx}
\usepackage[dvipdfmx]{color}
%\usepackage[dvips]{graphicx}
%\usepackage[dvips]{color}
\usepackage{picins}
\pagestyle{empty}
\setlength{\textwidth}{162mm}
\setlength{\textheight}{230mm}
\setlength{\oddsidemargin}{-15.4mm}
\setlength{\evensidemargin}{-15.4mm}
\setlength{\topmargin}{-20.4mm}
%\setlength{\columnseprule}{0.4pt}
\def\dvrule{\vbox to15mm{%
           \cleaders
              \hbox{\vrule height2.9pt depth-2.3pt \kern-.4pt
                    \vrule height.6pt}%
              \vfil}}
\def\bunsuushisuu#1#2{\raisebox{0.7zh}{$#1 \over #2$}}
\def\kaibox#1#2#3{\vrule\vbox{\hbox{\vrule width 0mm height 5mm depth 2mm \hspace{1mm}#2\hfill}\hbox to #1{\vrule width 0mm height 3mm depth 4mm \hspace{1mm}#3}\hrule}}
\def\dkai#1#2#3{\dvrule\vbox{\hbox{\vrule width 0mm height 5mm depth 2mm \hspace{1mm}#2\hfill}\hbox to #1{\vrule width 0mm height 3mm depth 4mm \hspace{1mm}#3}\hrule}}
\def\wkaibox#1#2#3{\vrule\vbox{\hbox{\vrule width 0mm height 5mm depth 25mm \hspace{1mm}#2\hfill}\hbox to #1{\vrule width 0mm height 3mm depth 4mm \hspace{1mm}#3}\hrule}}
\def\wwkaibox#1#2#3{\vrule\vbox{\hbox{\vrule width 0mm height 5mm depth 2mm \hspace{1mm}#2\hfill}\hbox to #1{\vrule width 0mm height 3mm depth 4mm \hspace{1mm}#3}\hrule}}
\def\zeroh{\hrule height 0mm}
\def\zerov{\vrule height 0mm}

\renewcommand\labelenumi{\fbox{\bfseries{\sffamily{\theenumi}}}}
\renewcommand{\labelenumii}{問\arabic{enumii}}
\renewcommand{\labelenumiii}{\bfseries{\カタカナ{enumiii}.}}
\newcounter{daimon}
\newcounter{shomon}[daimon]
\newcounter{aiueo}[daimon]
\newcounter{shoshomon}[shomon]
\newcommand{\ICHI}{\stepcounter{daimon}\fbox{\bfseries{\sffamily{\arabic{daimon}}}}}
\newcommand{\ichi}{\stepcounter{shomon}問\arabic{shomon}}
\newcommand{\michi}{\stepcounter{shoshomon}\maru{\arabic{shoshomon}}}
\newcommand{\aiu}{\stepcounter{aiueo}{\bfseries{\カタカナ{aiueo}}}}

\newcolumntype{C}{>{\centering\arraybackslash}X}
\newcolumntype{R}{>{\raggedright\arraybackslash}X}
\newcolumntype{L}{>{\raggedleft\arraybackslash}X}

\def\dvrule{\vbox to10.5mm{%
           \cleaders
              \hbox{\vrule height2.9pt depth-2.3pt \kern-.4pt
                    \vrule height.6pt}%
              \vfil}}

\def\bunsuushisuu#1#2{\raisebox{0.7zh}{$#1 \over #2$}}
\def\santaku{{\bfseries ア}~{\bfseries ウ}}
\def\yontaku{{\bfseries ア}~{\bfseries エ}}
\def\gotaku{{\bfseries ア}~{\bfseries オ}}
\def\rokutaku{{\bfseries ア}~{\bfseries カ}}
\def\nanataku{{\bfseries ア}~{\bfseries キ}}
\def\hachitaku{{\bfseries ア}~{\bfseries ク}}
\def\kyutaku{{\bfseries ア}~{\bfseries ケ}}
\def\jyutaku{{\bfseries ア}~{\bfseries コ}}
\def\sbou#1#2{$_{\maru{\sf{#1}}}$\kern-0.2pt \ulinej{#2}}%
\def\emaru#1{\kern-1pt\raisebox{-0.5zh}{
\unitlength 0.1in
\begin{picture}(  1.2000,  2.0000)(  1.4000, -3.0000)
\put(2.0000,-2.0000){\makebox(0,0){#1}}%
\special{pn 8}%
\special{ar 200 200 60 90  0.0000000 6.2831853}%
\end{picture}}\kern+2pt%
}
\begin{document}
\AtBeginDvi{\special{papersize=\the\paperwidth,\the\paperheight}}
\setcounter{page}{0}
\newpagestyle{custom}{(0mm,0pt)%
%{\hfill{\bf 奨学生 理科}}{{\bf 奨学生 理科}\hfill}{\hfill}%
%{\hfill{\bf 平成28年度 一学期中間考査(物理)}}{{\bf 平成28年度 一学期中間考査(物理)}\hfill}{\hfill}%
(\textwidth,0pt)}%
{(0mm,0pt)%
{~\hfill -~\thepage ~- \hfill~}{~\hfill -~\thepage~- \hfill~}{\hfill-\pagemark -\hfill}%
(0mm,0pt)}
\hakosyokika
\pagestyle{empty}
    \begin{enumerate}
    \item ~~
        \begin{mawarikomi}{270pt}{%WinTpicVersion4.32a
{\unitlength 0.1in%
\begin{picture}(39.1000,17.1000)(4.0000,-22.0000)%
% CIRCLE 2 0 3 0 Black White  
% 4 800 1800 1200 1800 1200 1800 1200 1800
% 
\special{pn 8}%
\special{ar 800 1800 400 400 0.0000000 6.2831853}%
% LINE 3 0 3 0 Black White  
% 2 800 1800 4000 1800
% 
\special{pn 4}%
\special{pa 800 1800}%
\special{pa 4000 1800}%
\special{fp}%
% CIRCLE 2 0 3 0 Black White  
% 4 4000 1800 4200 1800 4200 1800 4200 1800
% 
\special{pn 8}%
\special{ar 4000 1800 200 200 0.0000000 6.2831853}%
% CIRCLE 2 0 3 0 Black White  
% 4 3200 1800 3300 1800 3300 1800 3300 1800
% 
\special{pn 8}%
\special{ar 3200 1800 100 100 0.0000000 6.2831853}%
% DOT 0 0 3 0 Black White  
% 2 3200 1800 4000 1800
% 
\special{pn 4}%
\special{sh 1}%
\special{ar 3200 1800 16 16 0 6.2831853}%
\special{sh 1}%
\special{ar 4000 1800 16 16 0 6.2831853}%
% CIRCLE 2 0 3 0 Black White  
% 4 800 1800 4000 1800 4000 1800 4000 600
% 
\special{pn 8}%
\special{ar 800 1800 3200 3200 5.9244146 6.2831853}%
% SARROW 2 0 3 1 Black White  
% 2 3801 690 3796 676
% 
\special{pn 8}%
\special{pa 3801 690}%
\special{pa 3796 676}%
\special{fp}%
\special{sh 1}%
\special{pa 3796 676}%
\special{pa 3800 746}%
\special{pa 3814 726}%
\special{pa 3837 732}%
\special{pa 3796 676}%
\special{fp}%
% CIRCLE 2 0 3 0 Black White  
% 4 3810 690 4010 690 4010 690 4010 690
% 
\special{pn 8}%
\special{ar 3810 690 200 200 0.0000000 6.2831853}%
% CIRCLE 2 2 3 0 Black White  
% 4 3010 690 3110 690 3110 690 3110 690
% 
\special{pn 8}%
\special{pn 8}%
\special{pa 3110 690}%
\special{pa 3110 698}%
\special{fp}%
\special{pa 3101 731}%
\special{pa 3098 737}%
\special{fp}%
\special{pa 3081 761}%
\special{pa 3076 765}%
\special{fp}%
\special{pa 3049 782}%
\special{pa 3043 785}%
\special{fp}%
\special{pa 3009 790}%
\special{pa 3001 790}%
\special{fp}%
\special{pa 2970 782}%
\special{pa 2964 779}%
\special{fp}%
\special{pa 2940 761}%
\special{pa 2936 757}%
\special{fp}%
\special{pa 2919 731}%
\special{pa 2916 724}%
\special{fp}%
\special{pa 2910 691}%
\special{pa 2910 683}%
\special{fp}%
\special{pa 2918 651}%
\special{pa 2921 644}%
\special{fp}%
\special{pa 2939 620}%
\special{pa 2943 616}%
\special{fp}%
\special{pa 2970 598}%
\special{pa 2977 596}%
\special{fp}%
\special{pa 3010 590}%
\special{pa 3019 590}%
\special{fp}%
\special{pa 3050 598}%
\special{pa 3057 602}%
\special{fp}%
\special{pa 3082 620}%
\special{pa 3086 624}%
\special{fp}%
\special{pa 3102 651}%
\special{pa 3105 657}%
\special{fp}%
\special{pa 3110 690}%
\special{pa 3110 690}%
\special{fp}%
% DOT 0 0 3 0 Black White  
% 1 3010 690
% 
\special{pn 4}%
\special{sh 1}%
\special{ar 3010 690 16 16 0 6.2831853}%
% DOT 0 0 3 0 Black White  
% 1 3810 690
% 
\special{pn 4}%
\special{sh 1}%
\special{ar 3810 690 16 16 0 6.2831853}%
% LINE 3 0 3 0 Black White  
% 2 3010 690 3810 690
% 
\special{pn 4}%
\special{pa 3010 690}%
\special{pa 3810 690}%
\special{fp}%
% CIRCLE 2 0 3 0 Black White  
% 4 3810 690 4610 690 3010 690 3010 1290
% 
\special{pn 8}%
\special{ar 3810 690 800 800 2.4980915 3.1415927}%
% SARROW 2 0 3 1 Black White  
% 2 3161 1158 3170 1170
% 
\special{pn 8}%
\special{pa 3161 1158}%
\special{pa 3170 1170}%
\special{fp}%
\special{sh 1}%
\special{pa 3170 1170}%
\special{pa 3146 1105}%
\special{pa 3138 1127}%
\special{pa 3114 1129}%
\special{pa 3170 1170}%
\special{fp}%
% CIRCLE 2 0 3 0 Black White  
% 4 3170 1160 3270 1160 3270 1160 3270 1160
% 
\special{pn 8}%
\special{ar 3170 1160 100 100 0.0000000 6.2831853}%
% DOT 0 0 3 0 Black White  
% 1 3170 1160
% 
\special{pn 4}%
\special{sh 1}%
\special{ar 3170 1160 16 16 0 6.2831853}%
% LINE 3 0 3 0 Black White  
% 2 800 1800 3800 690
% 
\special{pn 4}%
\special{pa 800 1800}%
\special{pa 3800 690}%
\special{fp}%
% LINE 3 0 3 0 Black White  
% 2 3800 690 3160 1160
% 
\special{pn 4}%
\special{pa 3800 690}%
\special{pa 3160 1160}%
\special{fp}%
% STR 2 0 3 0 Black White  
% 4 3800 1430 3800 1530 5 0 0 0
% 地球
\put(38.0000,-15.3000){\makebox(0,0){地球}}%
% STR 2 0 3 0 Black White  
% 4 800 1910 800 2010 5 0 0 0
% 太陽
\put(8.0000,-20.1000){\makebox(0,0){太陽}}%
% DOT 0 0 3 0 Black White  
% 1 800 1800
% 
\special{pn 4}%
\special{sh 1}%
\special{ar 800 1800 16 16 0 6.2831853}%
% STR 2 0 3 0 Black White  
% 4 800 1590 800 1690 5 0 0 0
% {\sf S}
\put(8.0000,-16.9000){\makebox(0,0){{\sf S}}}%
% STR 2 0 3 0 Black White  
% 4 4000 1790 4000 1890 5 0 0 0
% {\sf E}
\put(40.0000,-18.9000){\makebox(0,0){{\sf E}}}%
% STR 2 0 3 0 Black White  
% 4 3810 510 3810 610 5 0 0 0
% {\sf E'}
\put(38.1000,-6.1000){\makebox(0,0){{\sf E'}}}%
% STR 2 0 3 0 Black White  
% 4 2810 590 2810 690 5 0 0 0
% {\sf M}
\put(28.1000,-6.9000){\makebox(0,0){{\sf M}}}%
% STR 2 0 3 0 Black White  
% 4 3010 1190 3010 1290 5 0 0 0
% {\sf M'}
\put(30.1000,-12.9000){\makebox(0,0){{\sf M'}}}%
% STR 2 0 3 0 Black White  
% 4 3030 1830 3030 1930 5 0 0 0
% 月
\put(30.3000,-19.3000){\makebox(0,0){月}}%
% CIRCLE 1 0 3 0 Black White  
% 4 800 1800 1400 1800 3810 1800 3820 690
% 
\special{pn 13}%
\special{ar 800 1800 600 600 5.9309624 6.2831853}%
% STR 2 0 3 0 Black White  
% 4 1460 1640 1460 1740 2 0 0 0
% $x$
\put(14.6000,-17.4000){\makebox(0,0)[lb]{$x$}}%
% CIRCLE 2 0 3 0 Black White  
% 4 3800 690 4200 690 3400 690 3160 1160
% 
\special{pn 8}%
\special{ar 3800 690 400 400 2.5081668 3.1415927}%
% CIRCLE 2 0 3 0 Black White  
% 4 3800 690 3370 690 3000 690 3150 1170
% 
\special{pn 8}%
\special{ar 3800 690 430 430 2.5055172 3.1415927}%
% STR 2 0 3 0 Black White  
% 4 3290 680 3290 780 5 0 0 0
% $y$
\put(32.9000,-7.8000){\makebox(0,0){$y$}}%
% CIRCLE 2 0 3 0 Black White  
% 4 4000 1800 4310 1800 4310 2000 4310 1600
% 
\special{pn 8}%
\special{ar 4000 1800 310 310 5.7102192 0.5729661}%
% SARROW 2 0 3 1 Black White  
% 2 4268 1645 4260 1632
% 
\special{pn 8}%
\special{pa 4268 1645}%
\special{pa 4260 1632}%
\special{fp}%
\special{sh 1}%
\special{pa 4260 1632}%
\special{pa 4278 1699}%
\special{pa 4288 1677}%
\special{pa 4312 1678}%
\special{pa 4260 1632}%
\special{fp}%
% SARROW 2 0 3 2 Black White  
% 2 4268 1955 4260 1968
% 
\special{pn 8}%
\special{pa 4268 1955}%
\special{pa 4260 1968}%
\special{fp}%
\special{sh 1}%
\special{pa 4260 1968}%
\special{pa 4312 1922}%
\special{pa 4288 1923}%
\special{pa 4278 1901}%
\special{pa 4260 1968}%
\special{fp}%
% STR 2 0 3 0 Black White  
% 4 4310 1460 4310 1560 5 0 0 0
% {\sf A}
\put(43.1000,-15.6000){\makebox(0,0){{\sf A}}}%
% STR 2 0 3 0 Black White  
% 4 4320 1930 4320 2030 5 0 0 0
% {\sf B}
\put(43.2000,-20.3000){\makebox(0,0){{\sf B}}}%
% STR 2 0 3 0 Black White  
% 4 4010 1480 4010 1580 2 0 0 0
% {\sf P}
\put(40.1000,-15.8000){\makebox(0,0)[lb]{{\sf P}}}%
% STR 2 0 3 0 Black White  
% 4 3990 1920 3990 2020 1 0 0 0
% {\sf Q}
\put(39.9000,-20.2000){\makebox(0,0)[lt]{{\sf Q}}}%
% DOT 0 0 3 0 Black White  
% 2 4000 1600 4000 2000
% 
\special{pn 4}%
\special{sh 1}%
\special{ar 4000 1600 16 16 0 6.2831853}%
\special{sh 1}%
\special{ar 4000 2000 16 16 0 6.2831853}%
\end{picture}}%
}
            月が地球のまわりを公転する際の通り道は,真円ではないため,月の満ち欠けの周期はわずかに変動するが,平均すると,29.5日になる。しかし,月が地球のまわりを1周する,つまり360度公転するのに要する日数は,月の満ち欠けの周期と異なる。右図は,太陽系を地球の北極側から見た模式図である。以下の問いに答えなさい。
            \begin{enumerate}
                \item 以下の表から,地球の自転方向と,地球上の明け方の地点の組み合わせとして正しい番号をマークせよ。\hako{1}\\
                    \begin{tabularx}{250pt}{|C|C|C|}
                        \hline
                        {\sf 番号}  &  自転方向    &  明け方の地点 \\ \hline
                        {\sf 1}    &  {\sf A}     &  {\sf P} \\ \hline
                        {\sf 2}    &  {\sf A}     &  {\sf Q} \\ \hline
                        {\sf 3}    &  {\sf B}     &  {\sf P} \\ \hline
                        {\sf 4}    &  {\sf B}     &  {\sf Q} \\ \hline
                    \end{tabularx}
                    \\
                    \item 以下の表から,日食と月食において,太陽,月の欠け始める方角として正しい番号をマークせよ。\hako{2}\\
                    \begin{tabularx}{250pt}{|C|C|C|}
                        \hline
                        {\sf 番号}  &  日食の太陽    &  月食の月 \\ \hline
                        {\sf 1}    &  東側     &  東側 \\ \hline
                        {\sf 2}    &  東側     &  西側 \\ \hline
                        {\sf 3}    &  西側     &  東側 \\ \hline
                        {\sf 4}    &  西側     &  西側 \\ \hline
                    \end{tabularx}
                    \\            
                \item 図において,{\sf E}の位置に地球があるときに,太陽と同じ方向にあった月は,地球が公転により{\sf E'}の位置にくると,{\sf M}から{\sf M'}の位置に公転により移動している。太陽の位置{\sf S}は動かないものとして,$\angle {\sf ESE'}=x$〔度〕,$\angle {\sf ME'M'}=y$〔度〕とする。地球が{\sf E}から{\sf E'}の位置に移動する間に,月は地球から見て,どのように移動したように見えるか。次の{\sf 1}~{\sf 6}から正しい番号を1つ選び,記号で答えよ。\hako{3}
                    \begin{edaenumerate}[{\sf 1.}]
                        \item $x$〔度〕だけ東側に移動
                        \item $x$〔度〕だけ西側に移動
                        \item $y$〔度〕だけ東側に移動
                        \item $y$〔度〕だけ西側に移動
                        \item $y-x$〔度〕だけ東側に移動
                        \item $y-x$〔度〕だけ西側に移動
                    \end{edaenumerate}
                \item 地球が{\sf E}から{\sf E'}まで公転する時間を1日としたとき,$y-x$の値を小数第2位を四捨五入して,小数第1位まで求め,\hako{4}~\hako{6}に適切な数字をマークせよ。\\
                    $y-x$=\hako{4}\hako{5}$.$\hako{6}〔度〕
                \item 1年を365日と考えると,1日あたりの$x$の値は約1.0度である。$x=1.0$〔度〕として,月の公転周期を小数第2位を四捨五入して小数第1位まで求め,\hako{7}~\hako{9}に適切な数字をマークせよ。\\
                    月の公転周期$=$\hako{7}\hako{8}$.$\hako{9}〔日〕
                \item 地球の自転周期を24時間とすると,月の南中時刻は1日で何分進むか,または遅れるか。\hako{10}には,進む場合は+,遅れる場合は-をマークし,月の南中時刻からのずれの時間を小数第2位を四捨五入して,小数第1位まで求め,\hako{11}~\hako{13}に適切な数字をマークせよ。\\
                    南中時刻のずれ$=$\hako{10}\hako{11}\hako{12}$.$\hako{13}〔分〕
                \end{enumerate}
        \end{mawarikomi}
    \end{enumerate}
\end{document}